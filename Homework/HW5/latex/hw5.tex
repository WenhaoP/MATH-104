\documentclass[12pt,lettersize]{article}

\usepackage[margin=1in]{geometry}
\usepackage{amsmath,physics}
\usepackage{amsfonts}
\usepackage{amssymb}
\usepackage{enumerate}
\usepackage{fancyhdr}
\usepackage{chngcntr}

\newcommand{\R}{\mathbb{R}}
\newcommand{\Q}{\mathbb{Q}}
\newcommand{\N}{\mathbb{N}}
\newcommand{\Z}{\mathbb{Z}}
\newcommand{\C}{\mathbb{C}}

\DeclareMathOperator{\dom}{dom}

\pagestyle{fancy}
\fancyhf{}
\lhead{MATH 104}
\chead{Homework 5}
\rhead{Wenhao Pan}
\cfoot{\thepage}

\setlength\parindent{0pt}

\begin{document}
	
	\subsection*{Q1}
	\begin{enumerate}[(a)]
		\item 
		\begin{itemize}
			\item Positive Definiteness: $\forall (x_1,y_1),(x_2,y_2)\in(X\times Y)\ d_X(x_1,x_2)\geq 0\text{ and }d_Y(y_1,y_2)\geq 0\implies d((x_1,y_1),(x_2,y_2))=\max\{d_X(x_1,x_2),d_Y(y_1,y_2)\}\geq 0$ since $d_X$ and $d_Y$ are proper metrics. Also because
			\begin{align*}
				\max\{d_X(x_1,x_2),d_Y(y_1,y_2)\}=0 &\iff 0\leq d_X(x_1,x_2)\leq0\text{ and }0\leq d_Y(y_1,y_2)\leq0\\
													&\iff d_X(x_1,x_2)=0\text{ and }d_Y(y_1,y_2)=0\\
													&\iff x_1=x_2\text{ and }y_1=y_2\\
													&\iff (x_1,y_1)=(x_2,y_2)
			\end{align*} 
			$d$ is positive definite.
			\item Symmetry: Since $d_X$ and $d_Y$ are proper metrics, it is clear that 
			\begin{displaymath}
				\max\{d_X(x_1,x_2),d_Y(y_1,y_2)\}=\max\{d_X(x_2,x_1),d_Y(y_2,y_1)\}.
			\end{displaymath}
			Thus $d$ is symmetric.
			\item Triangular Inequality: Since $d_X$ and $d_Y$ are proper metrics, it is clear that for each $(x_1,y_1),(x_2,y_2),(x_3,y_3)\in(X\times Y)$ we have
			\begin{align*}
				d_X(x_1,x_2) &\leq d_X(x_1,x_3)+d_X(x_3,x_2),\\
				d_Y(y_1,y_2) &\leq d_Y(y_1,y_3)+d_Y(y_3,y_2).\\
			\end{align*}
			If $d_X(x_1,x_2)\leq d_Y(y_1,y_2)$, then
			\begin{align*}
				\max\{d_X(x_1,x_2),d_Y(y_1,y_2)\}&=d_Y(y_1,y_2)\\
												 &\leq d_Y(y_1,y_3)+d_Y(y_3,y_2)\\
											     &\leq \max\{d_X(x_1,x_3),d_Y(y_1,y_3)\}+\max\{d_X(x_3,x_2),d_Y(y_3,y_2)\}.
			\end{align*}
			If $d_X(x_1,x_2)> d_Y(y_1,y_2)$, then
			\begin{align*}
				\max\{d_X(x_1,x_2),d_Y(y_1,y_2)\}&=d_X(x_1,x_2)\\
				&\leq d_X(x_1,x_3)+d_X(x_3,x_2)\\
				&\leq \max\{d_X(x_1,x_3),d_Y(y_1,y_3)\}+\max\{d_X(x_3,x_2),d_Y(y_3,y_2)\}.
			\end{align*}
			Thus combining two cases, we have
			\begin{displaymath}
				d((x_1,y_1),(x_2,y_2))\leq d((x_1,y_1),(x_3,y_3))+d((x_3,y_3),(x_2,y_2)),
			\end{displaymath}
			completing the proof.
		\end{itemize}
	
		\item Consider an arbitrary sequence $(x_n,y_n)$ in $E\times F$. Note that $(x_n)\in E$ and $(y_n)\in F$. Since $E$ is compact, $(x_n)$ has a subsequence $(x_{n_k})$ converging to $x_0$ in $E$. Moreover, since $F$ is compact, $(y_{n_k})$ has a subsequence $(y_{n_{k_l}})$ converging to $y_0$ in $F$. Now $(x_n,y_n)$ has a subsequence $(x_{n_{k_l}},y_{n_{k_l}})$ converging to $(x_0,y_0)$ in $E\times F$, so $E\times F$ is compact.
	\end{enumerate}
	\newpage
	
	\subsection*{Q2}
	Since $\sum a_n$ is convergent, $\lim a_n=0$. i.e. Let $\epsilon=1\ \exists N_1\in\N\ n\geq N_1\implies |a_n|<1\implies \abs{\frac{1}{a_n}}=\frac{1}{|a_n|}>1$. Thus if $\lim\frac{1}{a_n}$ exists, $\lim\frac{1}{a_n}\neq 0$, implying that $\sum \frac{1}{a_n}$ does not converge, and hence diverges.
	\newpage
	
	\subsection*{Q3}
	Since $\sum(a_n+b_n)=\sum a_n+\sum b_n$, it is clear that $\sum(a_n+b_n)$ converges when both $\sum a_n$ and $\sum b_n$ converge. Observe that when for each $n\in N$ $a_n$ and $b_n$ are nonnegative, we have
	\begin{align*}
		\abs{\sqrt{a_nb_n}}=\sqrt{a_nb_n}\leq\sqrt{2a_nb_n}\leq\sqrt{a_n^2+b_n^2+2a_nb_n}=\sqrt{(a_n+b_n)^2}=a_n+b_n.
	\end{align*}
	By comparison test, $\sum\sqrt{a_nb_n}$ converges.
	\newpage
	
	\subsection*{Q4}
	Since $\lim\inf|a_n|=0$, there exists a subsequence $|a_{n_k}|$ such that $\lim|a_{n_k}|=0$, i.e. $\forall\epsilon>0\ \exists K\in\N\ k\geq K\implies \abs{a_{n_k}}<\epsilon$. Thus select $\epsilon=1$ and according to the definition of limit, we can select $n_{k_1}$ such that $\abs{a_{n_{k_1}}}<1$. Having already found $n_{k_1}<n_{k_2}<\cdots<n_{k_l}$ such that $\abs{a_{n_{k_l}}}<\frac{1}{l^2}$, we can choose $n_{k_{l+1}}>n_{k_l}$ such that $\abs{a_{n_{k_{l+1}}}}<\frac{1}{(l+1)^2}$ since we have infinitely many terms smaller than $\frac{1}{(l+1)^2}$.\smallskip
	
	Now we have a series $\sum a_{n_{k_{l}}}$ such that $\abs{a_{n_{k_l}}}<\frac{1}{l^2}$ for each $l\in\N$. Then by comparison test, since $\sum\frac{1}{l^2}$ converges, $\sum a_{n_{k_l}}$ also converges. Note $(a_{n_{k_l}})$ is a subsequence of $(a_n)$.
	\newpage
	
	\subsection*{Q5}
	Let $\sum a_n=\sum\frac{(-1)^n}{\sqrt{n}}$. It is an alternating series because 
	\begin{displaymath}
		1>\frac{1}{\sqrt{2}}>\frac{1}{\sqrt{3}}>\cdots>0\text{ and }\lim\frac{1}{\sqrt{n}}=0.
	\end{displaymath}
	Thus $\sum a_n$ converges. Then $\sum a_n^2=\sum(\frac{(-1)^n}{\sqrt{n}})^2=\sum \frac{1}{n}$ is the harmonic series and hence $\sum a_n^2$ diverges.
	\newpage
	
	\subsection*{Q6}
	Since $\sum a_n$ converges, it satisfies the Cauchy Criterion, i.e.
	\begin{displaymath}
		\forall\epsilon>0\ \exists N\in\N\ n\geq m\geq N\implies \abs{\sum_{k=m}^{n}a_n}<\frac{\epsilon}{2}.
	\end{displaymath}
	Let $m=N$, then we have $\forall\epsilon>0$,
	\begin{align*}
		\frac{\epsilon}{2} &> \abs{a_N+a_{N+1}+\cdots+a_n}\\
				 &\geq (n-N+1)\abs{a_n}\quad\text{since $(a_n)$ is nonincreasing.}
	\end{align*}
	If $n\geq 2N$, then $n\leq 2(n-N)<2(n-N+1)$. Thus
	\begin{align*}
		|na_n|=n|a_n|<2(n-N+1)|a_n|<2\cdot\frac{\epsilon}{2}=\epsilon.
	\end{align*}
	i.e. $\lim na_n=0$.
	\newpage
	
	\subsection*{Q7}
	\begin{enumerate}[(a)]
		\item First observe that all terms are nonzero. Using ratio test, we have
		\begin{displaymath}
			\lim\abs{\frac{a_{n+1}}{a_n}}=\abs{\frac{n+1}{(n+2)!}\cdot\frac{(n+1)!}{n}}=\lim\abs{\frac{n+1}{n^2+2n}}=0.
		\end{displaymath}
		Thus $\lim\sup\abs{\frac{a_{n+1}}{a_n}}=0<1$, and hence the series converges (absolutely). 
		\item If $a=0$, then the series converges trivially. If $a\neq 0$, first observe all terms are nonzero, and then we have
		\begin{displaymath}
			\lim\abs{\frac{a_{n+1}}{a_n}}=\lim\abs{\frac{a^{n+1}}{(n+1)!}\cdot\frac{n!}{a^n}}=\lim\abs{\frac{a}{n+1}}=0.
		\end{displaymath}
		Thus $\lim\sup\abs{\frac{a_{n+1}}{a_n}}=0<1$, and hence the series converges (absolutely) by the ratio test.
		\item Let $(a_n)=\frac{1}{\sqrt{n}}$. Note that we have
		\begin{displaymath}
		\frac{1}{\sqrt{2}}>\frac{1}{\sqrt{3}}>\cdots>0\text{ and }\lim\frac{1}{\sqrt{n}}=0.
		\end{displaymath}
		Thus by alternating series test, the series converges.
		\item Observe that
		\begin{displaymath}
			\sum_{n=2}^{\infty}\frac{1}{n\log n}>\int_{2}^{\infty}\frac{1}{n\log n}\ dn=\infty.
		\end{displaymath}
		Thus by the integral test, the series diverges.
		
	\end{enumerate}
	\newpage
	
	\subsection*{Q8}
	\begin{enumerate}[(a)]
		\item If $x=0$, then $f(x)=0^n=0$ is continuous trivially. Now show that $g(x)=x$ is continuous. For each $x_0\in\R$ and each $\epsilon>0$, let $\delta=\epsilon$, then it is clear that $x\in\dom(f)\text{ and }|x-x_0|<\delta=\epsilon\implies \abs{f(x)-f(x_0)}=\abs{x-x_0}<\epsilon$. Thus $g(x)=x$ is continuous. Then for each $x_0\in\R$ and each $n>2$ in $\N$, suppose $f^\prime(x)=x^{n-1}$ is continuous at $x_0$, since $f(x)=x^n=x^{n-1}\cdot x=f^\prime(x)\cdot g(x)$, $f(x)$ is continuous at $x_0$ by theorem 17.4(ii). Moreover, we've shown $f(x)=x^n$ is continuous for each $n\in\N$, completing the proof.
		
		\item By part (a), $1,x,x^2,\dots,x^n$ are all continuous for each $n\in N$. Then by theorem 17.3, $a_0,a_1x,\dots,a_nx^n$ are continuous for each $a_0,a_1,\dots,a_n\in\R$. Thus using theorem 17.4(i) inductively, $p(x)=a_0+a_1x+\dots+a_nx^n$ is continuous.
	\end{enumerate}
	\newpage
	
	\subsection*{Q9}
	\begin{enumerate}[(a)]
		\item Consider $(x_n)=-\frac{1}{n}$. Obviously, $(x_n)\rightarrow0$. Let $x_0=0$. For each $n\in N$, $x_n<0\implies f(x_n)=0$, and hence $f(x_n)\rightarrow0$. However, since $f(x_0)=1$, $f(x_n)\nrightarrow f(x_0)$, $f(x)$ is discontinuous at $0$.
		\item Consider $(x_n)=\frac{2}{(4n-3)\pi}$. Obviously, $(x_n)\rightarrow0$. Let $x_0=0$. For each $n\in N$, $f(x_n)=\sin(\frac{(4n-3)\pi}{2})=1$, and hence $f(x_n)\rightarrow1$. However, since $f(x_0)=0$, $f(x_n)\nrightarrow f(x_0)$, $f(x)$ is discontinuous at $0$.
	\end{enumerate}
	\newpage
	
	\subsection*{Q10}
	\begin{itemize}
		\item If $r\in\Q$, then consider the sequence $(x_n)=r+\frac{\sqrt{2}}{n}$. Obviously, $(x_n)\rightarrow r\in\Q\implies f(r)=1$. However, $\forall n\in\N\ x_n\in\R\backslash\Q\implies f(x_n)=0$. Thus $f(x_n)\rightarrow 0\neq f(r)$, and hence $f(x)$ is discontinuous at $r$.
		\item If $r\in\R\backslash\Q$, then there exists a sequence $(q_n)$ of rational numbers such that $(q_n)\rightarrow r$. Then $\forall n\in\N\ f(q_n)=1$, but $f(r)=0$. Thus $f(q_n)\nrightarrow f(r)$, and hence $f$ is discontinuous at $r$.\smallskip
		
		Combining two cases above, we've shown that $f$ is discontinuous at every $r\in\R$.
	\end{itemize}
	\newpage
	
\end{document}
