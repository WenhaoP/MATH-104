\documentclass[12pt,lettersize]{article}

\usepackage[margin=1in]{geometry}
\usepackage{amsmath}
\usepackage{amsthm}
\usepackage{amsfonts}
\usepackage{amssymb}
\usepackage{enumerate}
\usepackage{fancyhdr}
\usepackage{chngcntr}

\newcommand{\R}{\mathbb{R}}
\newcommand{\Q}{\mathbb{Q}}
\newcommand{\N}{\mathbb{N}}
\newcommand{\Z}{\mathbb{Z}}
\newcommand{\C}{\mathbb{C}}
\newcommand{\com}{\mathsf{C}}


\newtheorem{lem*}{Lemma}

\pagestyle{fancy}
\fancyhf{}
\lhead{MATH 104}
\chead{Homework 3}
\rhead{Wenhao Pan}
\cfoot{\thepage}

\setlength\parindent{0pt}

\begin{document}
	The following lemma will be used in Q2 and Q4:
	\begin{lem*}
	For sequences $s_n\rightarrow s$ and $t_n\rightarrow t$, if there exists $N\in\N$ such that $n\geq N\implies s_n\leq t_n$, then $s\leq t$.
	\end{lem*}
	\begin{proof}
	We will use proof by contradiction. Suppose $s-t=\lim s_n-\lim t_n=\lim(s_n-t_n)>0$, then let $\epsilon=s-t>0$, so $\exists N\in\N\ |s_n-t_n-(s-t)|<s-t\implies s_n-t_n-(s-t)>-(s-t)\implies s_n-t_n>0\implies s_n>t_n$. i.e. there are infinitely $n\in\N$ such that $s_n>t_n$. Thus we have a contradiction, completing the proof.
	\end{proof}
	
	\newpage
	
	\subsection*{Q1}
	We need to show both directions:
	\begin{itemize}
		\item[$\implies$:] We will show the contrapositive of the forward direction which is "If $(s_n)$ does not converge to $s$, then there exists a subsequence of $(s_n)$ such that all of its subsequences do not converge to $s$." 
		
		Since $s_n\nrightarrow s$, then $\exists\epsilon_0>0$ such that $\forall N\in N\ \exists n\geq N\ |s_n-s|\geq\epsilon_0$. Then we can construct a subsequence of $(s_n)$ of which each term is at least $\epsilon_0$ away from $s$:
		\begin{itemize}
			\item[Base Case:] Let $N=1$, then there exists $n_1\in\N$ and $n_1>1$ such that $|s_{n_1}-s|\geq\epsilon_0$.
			\item[Induction step:] Given $n_1<\cdots<n_k\in\N$ such that $|s_{n_j}-s|\geq\epsilon_0$ for $j=1,\dots,k$, there exists $n_{k+1}\in\N$ and  $n_{k+1}>n_k$ such that $|s_{k+1}-s|\geq\epsilon_0$ by the condition $s_n\nrightarrow s$.
		\end{itemize}
		Now since every term of $s_{n_k}$ is $\epsilon_0>0$ away from $s$, all of its subsequences still have every term at least $\epsilon_0>0$ away from $s$, and hence cannot converge to $s$.
		\item[$\impliedby$:] Since $s_n\rightarrow s$, then every subsequence $(s_{n_k})$ of $(s_n)$ converges to $s$. Since each $(s_{n_k})$ itself is also a sequence and converges, $(s_{n_k})$ is bounded. Thus by Bolzano-Weierstrass Theorem, $(s_{n_k})$ has a convergent subsequence which converges to $s$ since $s_{n_k}\rightarrow s$.
	\end{itemize}
	\newpage
	
	\subsection*{Q2}
	We know for $N\in\N$, $n\geq N$ implies $s_n\leq\sup\{s_n: n\geq N\}\ \text{and}\ t_n\leq\sup\{t_n: n\geq N\}$, so
	$s_n+t_n\leq\sup\{s_n: n\geq N\}+\sup\{t_n: n\geq N\}$ and hence $\sup\{s_n+t_n: n\geq N\}\leq\sup\{s_n: n\geq N\}+\sup\{t_n: n\geq N\}$. Then we have
	\begin{align}
		\lim\sup\{s_n+t_n: n\geq N\} &\leq\lim(\sup\{s_n: n\geq N\}+\sup\{t_n: n\geq N\})\\
									 &= \lim\sup\{s_n: n\geq N\}+\lim\sup\{t_n: n\geq N\}.
	\end{align}
	(1) comes from Lemma 1. (2) comes from theorem 9.3 when $(s_n)$ and $(t_n)$ are bounded.
	\newpage
	
	\subsection*{Q3}
	\begin{enumerate}[(a)]
		\item Let's show both $\sup(-S)\leq-\inf S$ and $\sup(-S)\geq-\inf S$:
		\begin{itemize}
			\item[$\leq$:] Let $\inf S=u$, then $\forall s\in S$
			\begin{align*}
				s\geq u &\implies -u\geq -s\\
						&\implies -u \geq \sup(-S)\quad\text{since $-u$ is an upper bound of $-S$}\\
						&\implies \sup(-S)\leq-\inf S
			\end{align*}
			Thus $\sup(-S)\leq-\inf S$.
			\item[$\geq$:] Let $\sup(-S)=v$, then $\forall s\in S$
			\begin{align*}
				-s\leq v &\implies -v\leq s\\
						 &\implies -v\leq\inf S\quad\text{since $-v$ is a lower bound of $S$,}\\
						 &\implies -\inf S\leq v=\sup(-S)
			\end{align*} 
			Thus $\sup(-S)\geq-\inf S$, concluding $\sup(-S)=-\inf S$.
		\end{itemize}
	
		\item If $k=0$, then $\lim\sup(0\cdot s_n)=\lim\sup(0)=0=0\cdot\lim\sup(s_n)$. Thus $\lim\sup(ks_n)=k\cdot\lim\sup(s_n)$.
		
		If $k>0$, let $v^\prime_N=\sup\{ks_n: n\geq N\}$ and $v_N=\sup\{s_n: n\geq N\}$, then we have
		\begin{align*}
			n\geq N &\implies ks_n\leq v^\prime_N\\
				&\implies s_n\leq\frac{v^\prime_N}{k}\\
				&\implies v_N\leq\frac{v^\prime_N}{k}\\
				&\implies k\cdot v_N\leq v^\prime_N,
		\end{align*}
		and 
		\begin{align*}
			n\geq N &\implies s_n\leq v_N\\
					&\implies k\cdot s_n\leq k\cdot v_N\\
					&\implies v_N^\prime\leq k\cdot v_N
		\end{align*}
		Thus $ v_N^\prime=k\cdot v_N\implies \lim\sup(ks_n)=k\cdot\lim\sup(s_n)$, completing the proof.
		
		\item Since $k<0$, $-k>0$. Then we have 
		\begin{align*}
			\lim\sup(ks_n) &= \lim\sup((-k)(-s_n))\\
						   &= (-k)\cdot\lim\sup(-s_n)\quad\text{by (b)}\\
						   &= (-k)\cdot\lim-\inf(s_n)\quad\text{by (a)}\\
						   &= k\cdot\lim\inf(s_n).
		\end{align*}
		
		
	\end{enumerate}
	\newpage
	
	\setcounter{equation}{0}
	\subsection*{Q4}
	\begin{enumerate}[(a)]
		\item Consider $N\in\N$, then $n\geq N \implies s_n \leq \sup\{s_n: n\geq N\}$ and $t_n\leq\sup\{t_n: n\geq N\}$. Then we have
		\begin{align*}
			n\geq N \implies s_nt_n &\leq \sup\{s_n: n\geq N\}\cdot t_n\\
								    &\leq \sup\{s_n: n\geq N\}\cdot\sup\{t_n: n\geq N\}
		\end{align*}
		Thus $\sup\{s_n: n\geq N\}\cdot\sup\{t_n: n\geq N\}$ is an upper bound of $\{s_nt_n: n\geq N\}$ and hence $\sup\{s_nt_n: n\geq N\}\leq\sup\{s_n: n\geq N\}\cdot\sup\{t_n: n\geq N\}$.
		
		Since $(s_n)$ and $(t_n)$ are bounded, we have
		\begin{align}
			\lim\sup s_nt_n &\leq \lim_N(\sup\{s_n: n\geq N\}\cdot\sup\{t_n: n\geq N\})\\
							&= \lim\sup s_n\cdot\lim\sup t_n
		\end{align}
		(1) comes from Lemma 1. (2) comes from theorem 9.4 when $(s_n)$ and $(t_n)$ are bounded.
		\item Let $s_n = (-1)^n$ and $t_n=-1$ for $n\in N$. Then $s_nt_n=(-1)^{n+1}$ for $n\in\N$. Thus $\lim\sup s_nt_n=1$ , $\lim\sup s_n=1$, and $\lim\sup t_n=-1$. Now we have $\lim\sup s_nt_n=1>-1=(\lim\sup s_n)(\lim\sup t_n)$.
	\end{enumerate}
	\newpage
	
	\subsection*{Q5}
	\begin{enumerate}[(a)]
		\item First show the first inequality $\lim\sup\bar{s}_n\leq\lim\sup s_n$. There are three cases regarding to the value of $\lim\sup s_n$.
		\begin{itemize}
			\item[Case 1:] If $\lim\sup s_n=\infty$, then for any value $\lim\sup\bar{s}_n\in\R\cup\{+\infty,-\infty\}$, $\lim\sup\bar{s}_n\leq\lim\sup s_n$.
			
			\item[Case 2:] If $\lim\sup s_n=-\infty$, since $\lim\inf s_n\leq\lim\sup s_n$, we have $\lim\inf s_n=-\infty=\lim\sup s_n\implies \lim s_n=-\infty$. Intuitively, $\lim\bar{s}_n=-\infty$.  Because $\lim s_n=-\infty$, for $M<0$ and $M-1<0$, $\exists N\in\ n\geq N\implies s_n<M-1$, then we have $n\geq N$ implies
			\begin{align*}
				\bar{s}_n = \frac{s_1+\cdots+s_{N-1}+s_N+\cdots+s_n}{n} &= \frac{s_1+\cdots+s_{N-1}}{n}+\frac{s_N+\cdots+s_n}{n}\\
				&< \frac{s_1+\cdots+s_{N-1}}{n} + \frac{(n-N+1)(M-1)}{n}\\
				&= \frac{s_1+\cdots+s_{N-1}}{n} + \frac{n}{n}(M-1)+\frac{-N+1}{n}(M-1)\\
				&= \frac{s_1+\cdots+s_{N-1}+(-N+1)(M-1)}{n} + (M-1)
			\end{align*}
			Since for fixed $N$ and $M$, $F(n)=\frac{s_1+\cdots+s_{N-1}+(-N+1)(M-1)}{n}\rightarrow0$, $\exists N^\prime\geq N\ F(N^\prime)<1$. Because $F(n)$ is nonincreasing, we have $n\geq N^\prime\implies F(n)\leq F(N^\prime)<1$.
			\begin{displaymath}
				n\geq N^\prime\implies \bar{s}_n<\frac{s_1+\cdots+s_{N-1}+(-N+1)(M-1)}{n} + (M-1)<1+(M-1)=M
			\end{displaymath}
			Thus $\lim\bar{s}_n=-\infty$, completing the case.
			
			\item[Case 3:] If $\lim\sup s_n=\alpha\in\R$, then for each $\frac{\epsilon}{2}>0$, $\exists N\in\N\ v_N<\alpha+\frac{\epsilon}{2}$. Notice $v_N$ is nonincreasing. Observe that for fixed $N$, $F(n)=\frac{s_1+\cdots+s_{N-1}-(N-1)v_N}{n}\rightarrow 0$ as $n\rightarrow0$, so for each $\frac{\epsilon}{2}>0$, $\exists N^\prime\geq N\ n\geq N^\prime\implies F(n)\leq F(N^\prime)<\frac{\epsilon}{2}$ since $F(n)$ is nonincreasing. Thus for each $\epsilon>0$, we have $n\geq N^\prime\implies \bar{s}_n\leq F(N^\prime)+v_{N^\prime}\leq F(N^\prime)+v_N<\frac{\epsilon}{2}+(\alpha+\frac{\epsilon}{2})=\alpha+\epsilon\implies \bar{s}_n<\alpha\implies \sup\{\bar{s}_n: n\geq N^\prime\geq N\}\leq\alpha$. Thus $\lim\sup\bar{s}_n\leq\lim\alpha=\alpha=\lim\sup s_n$, completing the proof of the first inequality.
		\end{itemize}\bigskip
		
		The proof of the second inequality mirrors the proof of the first.
		\item If $\lim s_n$ exists, then $\lim\inf s_n=\lim\sup s_n$. It is clear that $\lim\inf\bar{s}_n\leq\lim\sup\bar{s}_n$, then $\lim\inf s_n\leq\lim\inf\bar{s}_n\leq\lim\sup\bar{s}_n\leq\lim\sup s_n$ achieves equality every where, so $\lim\inf\bar{s}_n=\lim\sup\bar{s}_n$ and hence $\lim\bar{s}_n$ exists. Then $\lim\bar{s}_n=\lim\inf\bar{s}_n=\lim\inf s_n=\lim s_n$, completing the proof.
		\item Let $s_n=(-1)^n$. Obviously $(s_n)$ does not converge since its set of subsequential limit has elements $-1$ and $1$. However $\bar{s}_n=\frac{(-1)^n}{n}$ converges to $0$.
		\item First such a sequence is not monotonic. Consider a sequence whose terms are in $\{-1, -1\}$. Each group of $1$'s or $-1$'s is followed by a longer alternative group of $-1$'s or $1$'s, like $\{1,-1,-1,1,1,1,-1,-1,-1,-1,1,1,1,1,1,\dots\}$. Thus, its first $n$ terms average $s_n$ will fluctuate between $-1$ and $1$ though slower and slower but never converge to some point. 
	\end{enumerate}
	\newpage
	
	\subsection*{Q6}
	\begin{enumerate}[(a)]
		\item We need to show positive definiteness, symmetry, and triangular inequality of this metric:
		\begin{itemize}
			\item Positive Definiteness: $\forall \mathbf{x,y}\in\R^k\ d(\mathbf{x},\mathbf{y})=\sum_{j=1}^{k}|y_j-x_j|\geq\sum_{j=1}^{k}0=0$. Also if $\mathbf{x}=\mathbf{y}$, then $\forall j=1,\dots,k\ x_j=y_j\implies y_j-x_j=0\implies\sum_{j=1}^{k}|y_j-x_j|=0$. If $d(\mathbf{x},\mathbf{y})=\sum_{j=1}^{k}|y_j-x_j|=0$, then $\forall j=1,\dots,k\ y_j-x_j=0\implies x_j=y_j\implies \mathbf{x}=\mathbf{y}$.
			\item Symmetry: Since $|y_j-x_j|=|(-1)(x_j-y_j)|=|x_j-y_j|$, it is clear that $\forall \mathbf{x,y}\in\R^k\ \sum_{j=1}^{k}|y_j-x_j|=\sum_{j=1}^{k}|x_j-y_j|\implies d(\mathbf{x},\mathbf{y})=d(\mathbf{y},\mathbf{x})$.
			\item Triangular Inequality: $\forall \mathbf{x,y,z}\in\R^k$
			\begin{align*}
				d(\mathbf{x,z})=\sum_{j=1}^{k}|z_j-x_j| &= \sum_{j=1}^{k}|z_j-y_j+y_j-x_j|\\
														&\leq \sum_{j=1}^{k}(|z_j-y_j|+|y_j-x_j|)\\
														&= \sum_{j=1}^{k}|z_j-y_j|+\sum_{j=1}^{k}|y_j-x_j|\\
														&= d(\mathbf{y,z})+d(\mathbf{x,y})
			\end{align*}
			Thus $d(\mathbf{x,z})\leq d(\mathbf{x,y})+d(\mathbf{y,z})$, completing the proof.
		\end{itemize}
		\item Consider a Cauchy sequence $(\mathbf{x}^{(n)})\in\R^k$. By Lemma 13.3, $\forall j=1,\dots,k\ \mathbf{x}^{(n)}_j$ is a Cauchy sequence in $\R$. By the completeness of $\R$, $\forall j=1,\dots,k\ \mathbf{x}^{(n)}_j$ is convergent in $\R$. Then by Lemma 13.3 again $(\mathbf{x}^{(n)})$ is convergent in $\R^k$ and hence $(\R^k,d)$ is complete.
	\end{enumerate}
	\newpage
	
	\subsection*{Q7}
	We will show both directions:
	\begin{itemize}
		\item[$\implies$:] Suppose $x$ is a limit point of $E$, then $\forall r>0\ (B_r(x)\backslash\{x\})\cap E\neq\emptyset$. We will use inductive construction to build a sequence $(x_n)$ of points in $E\backslash\{x\}$ such that $(x_n)$ converges to $x$:
		\begin{itemize}
			\item[Base case:] Let $r=1$, then $\exists s\in(B_1(x)\backslash\{x\})\cap E \implies s\in E\backslash\{x\}$ and $d(x,s)<1$. Let $s_1=s$.
			\item[Induction Step:] Given $s_1,\dots,s_k\in E\backslash\{x\}$ such that $d(x,s_j)<\frac{1}{j}$ for $j=1,\dots,k$. Since $x$ is a limit point of $E$, $\exists s\in(B_{\frac{1}{k+1}}(x)\backslash\{x\})\cap E\implies s\in E\backslash\{x\}$ and $d(x,s)<\frac{1}{k+1}$. Let $s_{k+1}=s$.
		\end{itemize}
		Thus we've built a $(x_n)$ of points in $E\backslash\{x\}$ such that $d(x,s_n)<\frac{1}{n}$ for $n\in\N$. Since $0\leq d(x,s_n)$ for $n\in\N$, by Squeeze Lemma $\lim_nd(x,s_n)=0\implies x_n\rightarrow x$.
		\item[$\impliedby$:] Suppose there exists a sequence $(x_n)$ of points in $E\backslash\{x\}$ such that $(x_n)$ converges to $x$. In other words, $\forall r>0\ \exists N\in\N\ n\geq N\implies (x_n\in E\backslash\{x\})\land (d(x,x_n)<r)\implies \forall n\geq N\ x_n\in (B_r(x)\backslash\{x\})\cap E\implies (B_r(x)\backslash\{x\})\cap E\neq\emptyset$. Thus $x$ is a limiting point.
	\end{itemize}
	\newpage
	
	\setcounter{equation}{0}
	\subsection*{Q8}
	Consider $x\in E^\prime$. Then we have $\forall r>0\ (B_r(x)\backslash\{x\})\cap E\neq\emptyset$. Now $\forall s\in(B_r(x)\backslash\{x\})\cap E$
	\begin{align}
		(s\in(B_r(x)\backslash\{x\}))\land(s\in E) &\implies (s\in(B_r(x)\backslash\{x\}))\land(s\in F)\\
												   &\implies s\in(B_r(x)\backslash\{x\})\cap F
	\end{align}
	(1) comes from $E\subseteq F$, and (2) comes from the definition of intersection. Thus $(B_r(x)\backslash\{x\})\cap E\subseteq (B_r(x)\backslash\{x\})\cap F$, and hence $(B_r(x)\backslash\{x\})\cap F\neq\emptyset$. This implies $x$ is also a limit point of $F$, so $x\in F^\prime$. Thus $E^\prime\subseteq F^\prime$.
	\newpage
	
	\subsection*{Q9}
	\begin{enumerate}[(a)]
		\item If we can show $\overline{E}^\com$ is open, then $\overline{E}$ is closed. Consider $x\in\overline{E}^\com$, then
		\begin{align*}
			\forall x\in(E\cup E^\prime)^\com &\implies (x\notin E)\land(x\notin E^\prime)\\
								   &\implies \exists r_1>0\ B_{r_1}(x)\cap E=\emptyset\\
								   &\implies \exists r_1>0\ B_{r_1}(x)\subseteq E^\com
		\end{align*}
		
		Since $x\notin E^\prime$, $x\in(E^\prime)^\com$. Also we know $E^\prime$ is closed, so $(E^\prime)^\com$ is open, and hence $\exists r_2>0\ B_{r_2}(x)\subseteq (E^\prime)^\com$. Take $r=\min\{r_1,r_2\}$ then
		\begin{displaymath}
			(B_r(x)\subseteq E^\com)\land(B_r(x)\subseteq (E^\prime)^\com)\implies B_r(x)\subseteq(E\cup E^\prime)^\com=\overline{E}^\com
		\end{displaymath}
		Since $\forall x\in\overline{E}^\com\ \exists r_x>0\ B_{r_x}(x)\subseteq\overline{E}^\com$,
		\begin{displaymath}
			\bigcup_{x\in\overline{E}^\com}^{}B_{r_x}(x)\subseteq \overline{E}^\com.
		\end{displaymath} 
		It is clear that $\overline{E}^\com\subseteq\bigcup_{x\in\overline{E}^\com}B_{r_x}(x)$ because every point in $\overline{E}^\com$ is a center of an open ball. Now since $\overline{E}^\com=\bigcup_{x\in\overline{E}^\com}B_{r_x}(x)$ and union of open balls (sets) is still open, $\overline{E}^\com$ is open. 
		
		\item We will show both directions:
		\begin{itemize}
			\item[$\implies$:] From (a) we know $\overline{E}$ is closed, so $E$ is closed.
			\item[$\impliedby$:] If $E$ is closed, by definition $E^\prime\subseteq E$. Thus $\overline{E}=E\cup E^\prime=E$.
		\end{itemize}
	
		\item From (b) we know $\overline{F}=F\cup F^\prime=F$. From Q8 we have $E\subseteq F$ implies $E^\prime\subseteq F^\prime$. Then it is clear that $\overline{E}=E\cup E^\prime\subseteq F\cup F^\prime=\overline{F}=F$, completing the proof.
	\end{enumerate}
	\newpage
	
	\subsection*{Q10}
	\begin{enumerate}[(a)]
		\item $\forall x\in E^\circ\ \exists r>0\ B_r(x)\subseteq E$. Since $B_r(x)$ itself is open, $\forall y\in B_r(x)$
		\begin{align*}
			\exists s>0\ B_s(y)\subseteq B_r(x)\subseteq E &\implies y\in E^\circ\\
														   &\implies B_r(x)\subseteq E^\circ.
		\end{align*}
		Thus $x\in(E^\circ)^\circ$, and hence $E^\circ$ is open by definition.
		\item We will show both directions:
		\begin{itemize}
			\item[$\implies$:] From (a) we know $E^\circ$ is open, so $E$ is open.
			\item[$\impliedby$:] If $E$ is open, by definition $\forall x\in E\ x\in E^\circ\implies E\subseteq E^\circ$. It is clear that $E^\circ\subseteq E$ since any interior point of a set is in the set. Thus $E=E^\circ$.
		\end{itemize}
		\item Since $F$ is open, by (b) $F^\circ=F$. $\forall x\in F^\circ\ \exists r>0\ B_r(x)\subseteq F\subseteq E$, so $x\in E^\circ$ and hence $F^\circ\subseteq E^\circ$. Thus $F=F^\circ\subseteq E^\circ$, completing the proof.
	\end{enumerate}
	\newpage
	
\end{document}
