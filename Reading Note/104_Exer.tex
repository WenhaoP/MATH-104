\documentclass[12pt, lettersize]{article}

% format setting
\usepackage[margin=1in]{geometry}

\usepackage{amsmath}
\usepackage{amsthm}
\usepackage{amssymb}
\usepackage{physics}
\usepackage{enumerate}
\usepackage{hyperref}
\usepackage{chngcntr}

% Theorem declaration
\newtheorem{thm}{Theorem}[section]
\newtheorem{dfn}[thm]{Definition}
\newtheorem{nte}[thm]{Notation}
\newtheorem{lem}[thm]{Lemma}
\newtheorem{eg}[thm]{Example}
\newtheorem{cor}{Corollary}[thm]

% Define new commands
\renewcommand\qedsymbol{\hfill $\blacksquare$}
\newcommand{\inner}[2]{\left\langle #1, #2 \right\rangle}
\newcommand{\R}{\mathbf{R}}
\newcommand{\C}{\mathbf{C}}
\newcommand{\N}{\mathbf{N}}
\newcommand{\LL}{\mathcal{L}}
\newcommand{\PP}{\mathcal{P}}
\newcommand{\MM}{\mathcal{M}}
\newcommand{\F}{\mathbf{F}}
\newcommand{\NU}{\text{null}\,}
\newcommand{\RG}{\text{range}\,}
\newcommand{\SP}{\text{span}\,}
\newcommand{\com}{\mathsf{C}}

\newcommand{\lline}{\noindent\rule{\textwidth}{1pt}}

\title{MATH 104 Exercise Solutions}
\author{Wenhao Pan}
\date{July 3rd, 2021}

\begin{document}
	
	\maketitle
	
	\section*{Introduction}
	This note is a collection of the solutions to the recommended exercises of \textit{Elementary Analysis} by Kenneth A. Ross.	
	
	\section*{Chapter 8}
	\subsection*{Q6a}
	
	\lline
	
	\section*{Chapter 9}
	\subsection*{Q4}
	\begin{enumerate}[(a)]
		\item $s_1=1, s_2=\sqrt{2}, s_3=\sqrt{\sqrt{2}+1}, s_4=\sqrt{\sqrt{\sqrt{2}+1}+1}$.
		\item Since $(s_n)$ converges, $\lim (s_n)=\lim (s_{n+1})=s$ which implies
		\begin{align*}
			\lim s_n=\lim\sqrt{s_n+1} &= s\\
					 \lim s_n+1 &= s^2\\
					 \lim s_n &= s^2-1\\
					 s &= s^2-1
		\end{align*}
		Thus solve the last equation for $s$ to get $s=\frac{1+\sqrt{5}}{2}$ since $s_n>0$ for all $n$.
	\end{enumerate}
	
	\lline
	
	\subsection*{Q9}
	\begin{enumerate}
		\item[(c):] Let $s=\lim(s_n-t_n)$. Suppose $s>0$, then $\exists N_1\ n>N_1\implies |s_n-t_n-s|<s\implies s_n>t_n$. This contradicts to the condition that there exists $N_0$ such that $n>N_0\implies s_n\leq t_n$.  
	\end{enumerate}
	
	\lline
	
	\subsection*{Q10}
	\begin{enumerate}[(a)]
		\item Since $\lim s_n=+\infty$ and $k<0$, for each $\frac{M}{k}>0$, there exists $N$ such that $n>N\implies s_n>\frac{M}{k}$. Thus for each $M>0$, $n>N\implies ks_n>k\cdot\frac{M}{k}=M$, so $\lim ks_n=+\infty$.
	\end{enumerate}

	\lline
	
	\subsection*{Q11}
	\begin{enumerate}[(a)]
		\item Suppose $\inf\{t_n: n\in\N\}=m$. For each $M>0$, consider two cases of $M-m$:
		\begin{itemize}
			\item[Case 1:] If $M-m\leq0$ then $M>M-m$. Thus we have there exists $N_1$ such that $n>N_1\implies s_n>M-m\implies s_n+m>M\implies s_n+t_n\geq s_n+m>M$, so $\lim(s_n+t_n)=+\infty$.
			\item[Case 2:] If $M-m>0$, then there exists $N_1$ such that $n>N_1\implies s_n>M-m\implies s_n+m>M\implies s_n+t_n\geq s_n+m>M$, so $\lim(s_n+t_n)=+\infty$.  
		\end{itemize}
		\item We want to show that $\lim t_n>-\infty\implies \inf\{t_n: n\in\N\}>-\infty$. Since $\lim t_n\neq-\infty$, there exists $M$ with $-\infty<M<0$ such that $\forall N\in\N\ \exists n>N\ t_n>M$. This implies $\inf\{t_n: n\in \N\}\geq M>-\infty$. Then we can apply (a).
		\item Since $(t_n)$ is bounded, $\exists M\in\R$ such that $\forall n\in\N\ |t_n|\leq M$. This implies for all $n$, $t_n\geq -M\implies -M\leq\inf\{t_n: n\in\N\}\implies \inf{t_n: n\in\N}>-\infty$. Then we can apply (a).
	\end{enumerate}
	
	\lline
	
	\subsection*{Q18}
	\begin{enumerate}[(a)]
		\item Let $S=1+a+a^2+\cdots+a^n$, then $a\cdot S=a+a^2+a^3+\cdots+a^{n+1}$. Then subtract $aS$ from $S$ to get $S-aS=1-a^{n+1}\implies S=\frac{1-a^{n+1}}{1-a}$.
		\item $\lim_n(1+a+a^2+\cdots+a^n)=\lim_n\frac{1-a^{n+1}}{1-a}=\frac{1}{1-a}\lim(1-a^{n+1})=\frac{1}{1-a}(1-\lim a^{n})=\frac{1}{1-a}(1-0)=\frac{1}{1-a}$ when $|a|<1$.
		\item $\frac{1}{1-1/3}=\frac{3}{2}$.
		\item If $a\geq1$, then $\lim_n(1+a+a^2+\cdots+a^n)\geq\lim(1+1+1+\cdots+1)=\lim n=+\infty$. Thus $\lim_n(1+a+a^2+\cdots+a^n)=\infty$.
	\end{enumerate}
	
	\lline
	\section*{Chapter 10}
	
	\subsection*{Q9}
	\begin{enumerate}[(a)]
		\item $s_2 = (\frac{1}{2})\cdot1^2=\frac{1}{2}$; $s_3=(\frac{2}{3})\cdot(\frac{1}{2})^2=\frac{1}{2\cdot3}$; $s_4=\frac{3}{4}\cdot(\frac{1}{2\cdot3})^2=\frac{1}{2^2\cdot3\cdot4}$
		\item Observe that $s_n$ is nonincreasing(monotone) and bounded by 1, so $s_n$ converges and hence $\lim s_n$ exists.
		\item Since $\lim s_n$ exists, assume $\lim s_n=s$. Then $s=\lim s_{n+1}=\lim(\frac{n}{n+1})s_n^2=\lim(\frac{n}{n+1})s^2=s^2$. Then solve the equation for $s$ to get $s=1$ or $s=0$. Since $s_2<1$ and $s_n$ is strictly decreasing, $s=0$.
	\end{enumerate}
	\lline
	
	\section*{Chapter 11}
	\subsection*{Q8}
	First we want to show that $\inf\{s_n: n>N\}=-\sup\{-s_n: n>N\}$:
	\begin{itemize}
		\item[$\leq$:] Let $\inf\{s_n: n>N\}=m$, then we have
		\begin{align*}
			\forall n>N\ s_n\geq m &\implies \forall n>N\ -s_n\geq -m\\
								   &\implies \sup\{-s_n: n>N\}\leq -m\\
								   &\implies m \leq -\sup\{-s_n: n>N\}.
		\end{align*}
		Thus $\inf\{s_n: n>N\}\leq-\sup\{-s_n: n>N\}$.
		\item[$\geq$:] Let $-\sup\{-s_n: n>N\}=M$, then we have
		\begin{align*}
			\sup\{-s_n: n>N\}=-M &\implies \forall n>N\ -s_n\leq -M\\
							   &\implies \forall n>N\ M\leq s_n\\
							   &\implies M\leq\inf\{s_n: n>N\}.
		\end{align*}
		Thus $\inf\{s_n: n>N\}\geq-\sup\{-s_n: n>N\}$.
	\end{itemize}
	Thus $\inf\{s_n: n>N\}=-\sup\{-s_n: n>N\}$. Then $\lim_N\inf\{s_n: n>N\}=\lim_N(-\sup\{-s_n: n>N\})=-\lim_N\sup\{-s_n: n>N\}=-\lim_N\sup(-s_n)$.
	
	\lline
	
	\section*{Chapter 12}
	\subsection*{Q2}
	\begin{itemize}
		\item[]
		\item[$\implies$:] Since $0\geq\lim\inf|s_n|\leq\lim\sup|s_n|=0$, we have $\\lim\inf|s_n|=\lim\sup|s_n|=0\implies \lim|s_n|=0$. Since $\forall n\in\N\ -|s_n|\leq s_n\leq|s_n|$, by Squeeze Formula $\lim s_n=0$.
		\item[$\impliedby$:] From $\lim s_n=0$, we know $\forall\epsilon>0\ \exists N\in\N$
		\begin{align*}
			n\geq N &\implies |s_n-0|<\epsilon\\
					&\implies \big||s_n|-|0|\big|\leq|s_n-0|<\epsilon\\
					&\implies \big||s_n|-0\big|<\epsilon\\
					&\implies \lim|s_n|=0.
		\end{align*}
	\end{itemize}
	
	\lline
	
	\subsection*{Q7}
	
	\lline
	
	\subsection*{Q9b}
	Since $\lim\inf t_n>0$, $\exists N_1\ m=\inf\{t_n: n\geq N_1\}>0$. From $\lim\sup s_n=+\infty$, we know $\forall \frac{M}{m}>0\ \exists N_2\ \sup\{s_n: n\geq N_1\}>\frac{M}{m}$. Now take $N=\max\{N_1,N_2\}$, we have $\forall n\geq N$
	\begin{displaymath}
		\sup\{s_nt_n: n\geq N\} \geq s_nt_n\geq s_n\cdot m.						
	\end{displaymath}
	This implies 
	\begin{align*}
		\sup\{s_nt_n: n\geq N\} &\geq \sup\{s_n\cdot m: n\geq N\}\\
								&=m\cdot\sup\{s_n: n\geq N\}\quad\text{since $m>0$}\\
								&>m\cdot\frac{M}{m}\\
								&=M,
	\end{align*}
	completing the proof.
	
	\lline
	
	\subsection*{Q13}
	Let's prove $\lim\inf s_n=\sup A$. Consider $N\in\N$ then $\forall n\geq N\ u_N=\inf\{s_n: n\geq N\}\leq s_n$. i.e. $\{n\in\N: s_n<u_n\}\subseteq\{1,\dots,N-1\}$. Thus $u_N\in A\implies u_N\leq\sup A$, and hence $\lim_Nu_N=\lim\inf s_n\leq\sup A$.
	
	Now consider arbitrary $a\in A$. Let $N_0=\max\{n\in\N: s_n<a\}<\infty$ since all but finitely many $s_n<a$. Then $s_n\geq a$ for $n>N_0$. Thus for $N>N_0$ we have $u_N=\inf\{s_n: n\geq N\}\geq a$. It follows that $\lim_Nu_N=\lim\inf s_n\geq a$. Since $a$ is arbitrary, $\lim\inf s_n$ is an upper bound of $A$, and hence $\lim\inf s_n\geq\sup A$.
	
	\lline
	
	\section*{Chapter 13}
	\subsection*{Q3b}
	No, since $d^\ast((1,1,1,\dots),(0,0,0,\dots))=\sum_{j=1}^{\infty}|1-0|=\sum_{j=1}^{\infty}1=\infty$, and we don't allow distance function to be valued as $\infty$.
	
	\lline
	
	\setcounter{equation}{0}
	\subsection*{Q5a}
	We want to show both directions.
	\begin{itemize}
		\item[$\subseteq$:] Consider $u\in\bigcap_{\alpha\in\mathcal{A}}\mathcal{U}_{\alpha}^{\com}$, then we have
		\begin{align}
			\forall\alpha\in\mathcal{A}\ u\in\mathcal{U}_{\alpha}^{\com}&\implies\forall\alpha\in\mathcal{A}\ u\notin\mathcal{U}_{\alpha}\\
			&\implies u\notin \bigcup_{\alpha\in\mathcal{A}}\mathcal{U}_\alpha\\
			&\implies u\in\left(\bigcup_{\alpha\in\mathcal{A}}\mathcal{U}_\alpha\right)^\com.
		\end{align}
		(1) $\implies$ (2) because
		\begin{displaymath}
			\left(\neg\left(u\in\mathcal{U}_1\right)\right)\land\left(\neg\left(u\in\mathcal{U}_2\right)\right)\land\cdots=\neg\left((u\in\mathcal{U}_1\lor(u\in\mathcal{U}_2)\lor\cdots\right)=\neg\left(u\in\bigcup\mathcal{U}_i\right)
		\end{displaymath} 
		Thus $\bigcap_{\alpha\in\mathcal{A}}\mathcal{U}_{\alpha}^{\com}\subseteq\left(\bigcup_{\alpha\in\mathcal{A}}\mathcal{U}_\alpha\right)^\com$.
		\item[$\supseteq$:] Consider $u\in\left(\bigcup_{\alpha\in\mathcal{A}}\mathcal{U}_\alpha\right)^\com$, then we have
		\begin{align*}
			u\notin\bigcup_{\alpha\in\mathcal{A}}\mathcal{U}_\alpha &\implies\forall\alpha\in\mathcal{A}\ u\notin\mathcal{U}_\alpha\\
			&\implies\forall\alpha\in\mathcal{A}\ u\in\mathcal{U}_{\alpha}^{\com}\\
			&\implies u\in\bigcap_{\alpha\in\mathcal{A}}\mathcal{U}_{\alpha}^{\com}
		\end{align*}
		Thus $\bigcap_{\alpha\in\mathcal{A}}\mathcal{U}_{\alpha}^{\com}\supseteq\left(\bigcup_{\alpha\in\mathcal{A}}\mathcal{U}_\alpha\right)^\com$, and hence $\bigcap_{\alpha\in\mathcal{A}}\mathcal{U}_{\alpha}^{\com}=\left(\bigcup_{\alpha\in\mathcal{A}}\mathcal{U}_\alpha\right)^\com$.
	\end{itemize}
	

	
\end{document}