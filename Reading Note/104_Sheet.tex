\documentclass[12pt, lettersize]{book}

% format setting
\usepackage[margin=1in]{geometry}

\usepackage{amsmath}
\usepackage{amsthm}
\usepackage{amssymb}
\usepackage{physics}
\usepackage{enumerate}
\usepackage{hyperref}
\usepackage{chngcntr}
\usepackage{xcolor}
\usepackage{tcolorbox}


% Theorem declaration
\theoremstyle{plain}
\newtheorem{thm}{Theorem}[section]
\newtheorem{nte}[thm]{Notation}
\newtheorem{lem}[thm]{Lemma}
\newtheorem{cor}{Corollary}[thm]

\theoremstyle{definition}
\newtheorem{dfn}[thm]{Definition}
\newtheorem*{eg}{Example}

\theoremstyle{remark}
\newtheorem*{rem}{Remark}

\tcolorboxenvironment{thm}{
	colframe=cyan, colback=cyan!5, before skip=10pt,after skip=10pt}

\tcolorboxenvironment{dfn}{
	colframe=orange, colback=orange!5, before skip=10pt,after skip=10pt}

\tcolorboxenvironment{lem}{
	colframe=blue, colback=blue!5, before skip=10pt,after skip=10pt}

\tcolorboxenvironment{cor}{
	colframe=blue, colback=blue!5, before skip=10pt,after skip=10pt}

\tcolorboxenvironment{eg}{
	colframe=red, colback=red!5, before skip=10pt,after skip=10pt}

\renewcommand\qedsymbol{\hfill $\blacksquare$}
\newcommand{\R}{\mathbb{R}}
\newcommand{\N}{\mathbb{N}}
\newcommand{\Q}{\mathbb{Q}}
\newcommand{\Z}{\mathbb{Z}}
\newcommand{\dom}{\text{dom}\,}
\newcommand{\com}{\mathsf{C}}
\newcommand{\lline}{\noindent\rule{\textwidth}{1pt}}

\let\oldemptyset\emptyset
\let\emptyset\varnothing

\counterwithout{equation}{chapter}

\title{MATH 104 Cheat Sheet}
\author{Wenhao Pan}
\date{\today}

\begin{document}
	\maketitle
	
	This document is a collection of all mentioned definitions, theorems, and corollaries from \emph{Elementary Analysis} by Kenneth A. Ross or Theodore Zhu's lectures of MATH 104 Summer 2021.
	
	\tableofcontents
	
	\chapter{Introduction}
	\newpage
	\section{The Set $\N$ of Natural Numbers}
	We denote the set $\{1,2,3,\dots\}$ of all \emph{positive integers} by $\N$. Each positive integer $n$ has a successor, namely $n+1$. The following is 5 properties of $\N$:
	\begin{itemize}
		\item[\textbf{N1.}] $1$ belongs to $\N$.
		\item[\textbf{N2.}] If $n\in\N$, then its successor $n+1\in\N$.
		\item[\textbf{N3.}] $1$ is not the successor of any element in $\N$.
		\item[\textbf{N4.}] If $n$ and $m$ in $\N$ have the same successor, then $n=m$.
		\item[\textbf{N5.}] A subset of $\N$ which contains $1$, and which contains $n+1$ whenever it contains $n$, must equal $\N$.
	\end{itemize}
	Axiom \textbf{N5} is the basis of mathematical induction, which asserts all the statements $P_1,P_2,P_3,\dots$ are true provided
	\begin{itemize}
		\item[(\textbf{$I_1$})] $P_1$ is true,
		\item[(\textbf{$I_2$})] $P_{n+1}$ is true whenever $P_n$ is true.
	\end{itemize}
	\newpage
	\section{The Set $\Q$ of Rational Numbers}
	\begin{dfn}
		A number is called an \emph{algebraic number} if it satisfies a polynomial equation
		\begin{displaymath}
			c_nx^n+c_{n-1}x^{n-1}+\cdots+c_1x+c_0=0
		\end{displaymath}
		where the coefficients $c_0,c_1,\dots,c_n$ are integers, $c_n\neq0$ and $n\geq1$.
	\end{dfn}
	Rational numbers are always algebraic numbers. If $r=\frac{m}{n}$ is a rational number [$m,n\in\Z$ and $n\neq0$], then it satisfies the equation $nx-m=0$.
	
	\setcounter{equation}{0}
	\begin{thm}[Rational Zeros Theorem]\label{thm:2.2}
		Suppose $c_0,c_1,\dots,c_n$ are integers and $r$ is a rational number satisfying the polynomial equation
		\begin{equation}
			c_nx^n+c_{n-1}x^{n-1}+\cdots+c_1x+c_0=0
		\end{equation}
		where $n\geq1$, $c_n\neq0$ and $c_0\neq0$. Let $r=\frac{c}{d}$ where $c,d$ are integers having no common factors and $d\neq 0$. Then $c\,|\, c_0$ and $d\,|\,c_n$.
	\end{thm}
	In other words, the only rational candidates for solutions of (1) have the form $\frac{c}{d}$ where $c$ divides $c_0$ and $d$ divides $c_n$.
	\begin{proof}
		We are given
		\begin{displaymath}
			c_n\left(\frac{c}{d}\right)^n+c_{n-1}\left(\frac{c}{d}\right)^{n-1}+\cdots+c_1\left(\frac{c}{d}\right)+c_0=0
		\end{displaymath}
		Multiply both sides by $d^n$ and obtain
		\begin{displaymath}
			c_nc^n+c_{n-1}c^{n-1}d+c_{n-2}c^{n-2}d^2+\cdots+c_2c^2d^{n-2}+c_1cd^{n-1}+c_0d^n=0
		\end{displaymath}
		Solve for $c_0d^n$ and obtain
		\begin{displaymath}
			c_0d^n=-c[c_nc^{n-1}+c_{n-1}c^{n-2}d+\cdots+c_2cd^{n-2}+c_1d^{n-1}]
		\end{displaymath}
		Since $c$ and $d^n$ have no common factors, $c$ divides $c_0$. Do the same thing and solve for $c_nc^n$ and we will see $d$ divides $c_n$.
	\end{proof}
	
	\begin{cor}\label{cor:2.3}
		Consider the polynomial equation
		\begin{displaymath}
			x^n+c_{n-1}x^{n-1}+\cdots+c_1x+c_0=0
		\end{displaymath}
		where the coefficients $c_0,c_1,\dots,c_{n-1}$ are integers and $c_0\neq0$. Any rational solution of this equation
		must be an integer that divides $c_0$.
	\end{cor}
	\begin{proof}
		By the Rational Zeros Theorem \ref{thm:2.2}, the denominator of $r$ must divide the coefficient of $x^n$, which is $1$. Thus $r$ is an integer dividing $c_0$.
	\end{proof}
	
	\newpage
	\section{The Set $\R$ of Real Numbers}
	The set $\Q$ of Rational numbers also have the following properties for addition and multiplication:
	\begin{itemize}
		\item[\textbf{A1.}] $a+(b+c)=(a+b)+c$ for all $a,b,c$.
		\item[\textbf{A2.}] $a+b=b+a$ for all $a,b$.
		\item[\textbf{A3.}] $a+0=a$ for all $a$.
		\item[\textbf{A4.}] For each $a$, there is an element $-a$ such that $a+(-a)=0$.
		\item[\textbf{M1.}] $a(bc)=(ab)c$ for all $a,b,c$.
		\item[\textbf{M2.}] $ab=ba$ for all $a,b$.
		\item[\textbf{M3.}] $a\cdot1=a$ for all $a$.
		\item[\textbf{M4.}] For each $a\neq0$, there is an element $a^{-1}$ such that $aa^{-1}=1$.
		\item[\textbf{DL}] $a(b+c)=ab+ac$ for all $a,b,c$.  
	\end{itemize}
	
	The set $\Q$ also has an order structure $\leq$ satisfying
	\begin{itemize}
		\item[\textbf{O1.}] Given $a$ and $b$, either $a\leq b$ or $b\leq a$.
		\item[\textbf{O2.}] If $a\leq b$ and $b\leq a$, then $a=b$.
		\item[\textbf{O3.}] If $a\leq b$ and $b\leq c$, then $a\leq c$.
		\item[\textbf{O4.}] If $a\leq b$, then $a+c\leq b+c$.
		\item[\textbf{O5.}] If $a\leq b$ and $0\leq c$, then $ac\leq bc$.
	\end{itemize}
	
	\begin{thm}\label{thm:3.1}
		The following are consequences of the field properties:
		\begin{enumerate}[(i)]
			\item \textcolor{red}{$a+c=b+c\implies a=b$};
			\item $a\cdot0=0$ for all $a$;
			\item $(-a)b=-ab$ for all $a,b$;
			\item $(-a)(-b)=ab$ for all $a,b$;
			\item $(ac=bc)\land(c\neq0) \implies a=b$;
			\item $ab=0\implies(a=0)\lor(b=0)$ for $a,b,c\in\R$.
		\end{enumerate}
		for $a,c,c\in\R$.
	\end{thm}
	
	\begin{thm}\label{thm:3.2}
		The following are consequences of the properties of an ordered field:
		\begin{enumerate}[(i)]
			\item $a\leq b\implies-b\leq-a$;
			\item $(a\leq b)\land(c\leq0)\implies bc\leq ac$;
			\item $(0\leq a)\land(0\leq b)\implies 0\leq ab$;
			\item $0\leq a^2$ for all $a$;
			\item $0<1$;
			\item $0<a\implies0< a^{-1}$;
			\item $0<a<b\implies0<b^{-1}<a^{-1}$;
		\end{enumerate}
		for $a,c,c\in\R$.
	\end{thm}
	Note that $a<b$ can be represented as $(a\leq b)\land(a<b)$.
	
	\begin{dfn}\label{def:3.3}
		We define
		\begin{displaymath}
			\text{$|a|=a$ if $a\geq0$  and   $|a|=-a$ if $a\leq0$}
		\end{displaymath}
	\end{dfn}
	
	An useful fact: $|a|\leq b\iff -b\leq a\leq b$.
	
	\begin{dfn}\label{def:3.4}
		For numbers $a$ and $b$ we define dist$(a,b)=|a-b|$; dist$(a,b)$ represents the \emph{distance between $a$ and $b$}.
	\end{dfn}
	
	\begin{thm}\label{thm:3.5}
		\begin{enumerate}[(i)]
			\item[]
			\item $|a|\geq0$ for all $a\in\R$.
			\item $|ab|=|a|\cdot|b|$ for all $a,b\in\R$.
			\item $|a+b|\leq|a|+|b|$ for all $a,b\in\R$. 
		\end{enumerate}
	\end{thm}
	
	\begin{cor}\label{cor:3.6}
		dist$(a,c)\leq$ dist$(a,b)+$ dist$(b,c)$ for all $a,b,c\in\R$. This is equivalent to $|a-c|\leq|b-c|+|b-c|$.
	\end{cor}
	
	\begin{thm}[Triangle Inequality]\label{thm:3.7}
		$|a+b|\leq|a|+|b|$ for all $a,b$.
	\end{thm}
	\begin{cor}[Reverse Triangular Inequality]\label{cor: reverse triangular}
		\textcolor{red}{$\big||a|-|b|\big|\leq|a-b|$ for all $a,b\in\R$.}	
	\end{cor}
	
	\begin{tcolorbox}
		\textcolor{red}{Here is one of the most important techniques in real analysis}.
		\begin{enumerate}[(a)]
			\item If $a\leq b+\epsilon$ for any $\epsilon>0$, then $a\leq b$.
			\item If $a\geq b-\epsilon$ for any $\epsilon>0$, then $a\geq b$.
			\item If $|a-b|<\epsilon$ for any $\epsilon>0$, then $|a-b|=0$.
		\end{enumerate}
	\end{tcolorbox}
	
	\newpage
	\section{The Completeness Axiom}
	The completeness axiom for $\R$ ensure us $\R$ has no "gaps".
	\begin{dfn}\label{def:4.1}
		Let $S$ be a nonempty subset of $\R$.
		\begin{enumerate}[(a)]
			\item If $S$ contains a largest element $s_0$ [that is, $s_0\in S$ and $\forall s\in S,\ s\leq s_0$], then we call $s_0$ the \emph{maximum} of $S$ and write $s_0=\max S$.
			\item If $S$ contains a smallest element $s_0$ [that is, $s_0\in S$ and $\forall s\in S,\ s\geq s_0$], then we call $s_0$ the \emph{minimum} of $S$ and write $s_0=\min S$. 
		\end{enumerate}
	\end{dfn}
	Open intervals like $(a,b)=\{x\in\R: a<x\leq b\}$ have no minimum or maximum since the endpoints $a$ and $b$ is not in the interval.
	
	\begin{dfn}\label{def:4.2}
		Let $S$ be a nonempty subset of $\R$.
		\begin{enumerate}[(a)]
			\item If a real number $M$ satisfies $s\leq M$ for all $s\in S$, then $M$ is called an \emph{upper bound} of $S$ and the set $S$ is said to be \emph{bounded above}.
			\item If a real number $m$ satisfies $m\leq s$ for all $s\in S$, then $m$ is called an \emph{lower bound} of $S$ and the set $S$ is said to be \emph{bounded below}.
			\item The set $S$ is said to be \emph{bounded} if it is bounded above and bounded below. Thus $S$ is bounded if there exist real numbers $m$ and $M$ such that $S\subseteq[m,M]$.
		\end{enumerate}
	\end{dfn}
	The maximum of a set is always an upper bound for the set. Likewise, the minimum of a set is always a lower bound for the set.
	
	\begin{dfn}
		Least Upper Bound Property (LUBP)\newline
		An ordered set $S$ has the LUBP if every nonempty subset $\mathcal{A}\subset S$ that has an upper bound has a least upper bound in $S$.
	\end{dfn}
	Note that the set $\Q$ of rational number does not satisfy the LUBP but $\R$ does. e.g. $\mathcal(A)=\{q\in\Q: q^2<2\}$.
	
	\begin{dfn}\label{def:4.3}
		Let $S$ be a nonempty subset of $\R$.
		\begin{enumerate}[(a)]
			\item If $S$ is bounded above and $S$ has a least upper bound, then we will call it the \emph{supremum} of $S$ and denote it by $\sup S$.
			\item If $S$ is bounded below and $S$ has a greatest lower bound, then we will call it the \emph{infimum} of $S$ and denote it by $\inf S$.
		\end{enumerate}
	\end{dfn}
	If $S$ is bounded above, then $M=\sup S$ if and only if (i) $s\leq M$ for all $s\in S$, and (ii) whenever $M_1<M$, there exists $s_1\in S$ such that $s_1>M_1$. Or for each $\epsilon>0$, there exists $s\in S$ such that $s>\sup S-\epsilon$.
	
	Note that for a positive set $S=\{s: s>0\}$, its infimum is not always positive. Example: $\{\frac{1}{n}: n\in\N\}$. Each element is positive but the infimum is $0$.
	
	Here are some basic facts:
	\begin{itemize}
		\item If a set $S$ has finitely many elements, then $\max S$ exists.
		\item If $\max S$ exists, then $\sup S=\max S$.
		\item For any set $S\neq\emptyset$, $\inf S\leq \sup S$
	\end{itemize}
	
	\begin{thm}[Completeness Axiom]\label{thm:4.4}
		Every nonempty subset $S$ of $\R$ that is bounded above has a least upper bound. In other words, $\sup S$ exists and is a real number.
	\end{thm}
	Note that the completeness axiom does not hold for $\Q$. 
	\begin{cor}
		Every nonempty subset $S$ of $\R$ that is bounded below has a greatest lower bound. In other words, $\inf S$ exists and is a real number.
	\end{cor}
	
	\begin{thm}[Archimedean Property]\label{thm:4.6}
		If $a>0$ and $b>0$, then for some positive integer $n$, we have $na>b$.
	\end{thm}
	\begin{cor}
		(Set $a=1$). For any $b>0$, there exists $n\in\N$ such that $n>b$
	\end{cor}
	\begin{cor}
		(Set $b=1$). For any $a>0$, there exists $n\in\N$ such that $na>1\implies \frac{1}{n}<a$.
	\end{cor}
	
	\begin{lem}\label{lem:4.7}
		If $x,y\in\R$ such that $y-x>1$, then there exists $m\in\Z$ such that $x<m<y$.
	\end{lem}
	
	
	\begin{thm}[Denseness of $\Q$]\label{thm:4.7}
		If $a,b\in\R$ and $a<b$, then there is a rational $r\in\Q$ such that $a<r<b$.
	\end{thm}
	
	\newpage
	\section{The Symbols $+\infty$ and $-\infty$}
	The symbols $+\infty$ and $-\infty$ are extremely useful even though they are \textbf{not} real numbers. So for each real number a, $-\infty<a<\infty$. If a set $S$ is not bounded above, we define $\sup S=+\infty$. Likewise, if $S$ is not bounded below, then we define $\inf S=-\infty$.
	
	We can extend real numbers to $\R\cup\{-\infty,\infty\}$. Notice that this is not a \textbf{field}, so it does not satisfy all field properties.
	
	For emphasis, we recapitulate:
	
	Let $S$ be any nonempty subset of $\R$. The \emph{symbols} $\sup S$ and $\inf S$ always make sense. If $S$ is not bounded above, then $\sup S$ is a \emph{real} number; otherwise $\sup S=+\infty$. If $S$ is bounded below, then $\inf S$ is a \emph{real} number; otherwise $\inf S=-\infty$. Moreover, we have $\inf S\leq\sup S$.
	
	\chapter{Sequences}
	\newpage
	\section{Limits of Sequences}		
	\begin{dfn}\label{def:limit}
		A sequence $(s_n)$ of real numbers is said to \textbf{converge} to the real number \emph{$s$} provided that
		\begin{displaymath}
			\forall \epsilon > 0,\ \exists N,\ n > N \Rightarrow |s_n-s| < \epsilon.
		\end{displaymath}
		If $(s_n)$ converges to $s$, we write $\lim_{n\rightarrow \infty}s_n=s$ or $s_n\rightarrow s$. $s$ is the \emph{limit} of the sequence $(s_n)$.
		A sequence that does not converge (i.e. it has no \emph{limit}) is said to \emph{diverge}.\\
		Notice that in the definition, instead of simple $\epsilon$, we can also use some other complicated forms with some extra constants like $M\epsilon,\ \frac{\epsilon}{c},\ a^2\epsilon$ and so on.
	\end{dfn}
	
	Intuitively, the definition means that no matter how small you pick $\epsilon>0$, \textbf{eventually} the sequence will stay within $\epsilon$ of $s$ at some point (the threshold $N$) and forever after.
	
	\begin{thm}
		The limit of a sequence $(s_n)$ is unique. i.e. $(\lim s_n=s) \land (lim s_n=t) \Rightarrow s=t$.
	\end{thm}
	
	\begin{thm}
		\textcolor{red}{
			\begin{itemize}
				\item If $s_n\geq a$ for all but finitely many $n$, then $\lim s_n\geq a$.
				\item If $s_n\leq b$ for all but finitely many $n$, then $\lim s_n\leq b$.
		\end{itemize}}
	\end{thm}
	
	\begin{thm}[Squeeze Lemma]\label{lem: squeeze}
		\textcolor{red}{If $a_n\leq s_n\leq b_n$ for all $n$ and $\lim a_n=\lim b_n=s$, then $\lim s_n=s$.}
	\end{thm}
	
	\newpage
	
	\section{A Discussion about Proofs}
	This section gives several examples of proofs with some discussion using the definition of the limit of a sequence.
	\begin{eg}
		Prove $\lim \frac{1}{n^2}=0$.
	\end{eg}
	\emph{Discussion}. According to the definition of the limit, we need to consider an $\epsilon >0$ such that $|\frac{1}{n^2}-0|<\epsilon$ for $n>\text{some}N$.
	$|\frac{1}{n^2}-0|<\epsilon$ implies that $\frac{1}{\epsilon}<n^2 \text{or} \frac{1}{\sqrt{\epsilon}}<n$.
	Thus we can suppose $N=\frac{1}{\sqrt{\epsilon}}$ and check if we reverse our reasoning into proof, it still makes sense. 
	
	\begin{eg}
		Prove $\lim \frac{3n+1}{7n-4}=\frac{3}{7}$
	\end{eg}
	\emph{Discussion}. Just like the last example, we can start from the definition \ref{def:limit} to get a suitable $N$.
	\begin{proof}
		Let $\epsilon>0$ and $N=\frac{19}{49\epsilon}+\frac{4}{7}$, then
		\begin{align*}
			n>N &\Rightarrow 7n>\frac{19}{7\epsilon}+4\\
			&\Rightarrow \frac{19}{7(7n-4)}<\epsilon\\
			&\Rightarrow \frac{3n+1}{7n-4} - \frac{3}{7}<\epsilon\\
			&\Rightarrow \abs{\frac{3n+1}{7n-4} - \frac{3}{7}}<\epsilon\qquad \text{since}\ n>0
		\end{align*}
		This proofs $\lim \frac{3n+1}{7n-4}=\frac{3}{7}$ according to the definition of the limit \ref{def:limit}.
	\end{proof}
	
	\begin{eg}
		Prove $\lim\frac{4n^3+3n}{n^3-6}=4$
	\end{eg}
	\emph{Discussion}. Since $\frac{4n^3+3n}{n^3-6}-4 = \frac{3n+24}{n^3-6}$, when $n>1$, we can find an upper bound for
	$\frac{3n+24}{n^3-6}$ so that the bound $<\epsilon \Rightarrow \abs{\frac{3n+24}{n^3-6}}<\epsilon$. Finding an upper bound for a fraction is equivalent to finding a upper bound for its numerator and a lower bound for its denominator.
	We know $3n+24\leq27n$ for $n>1$. Also we note $n^3-6\geq\frac{n^3}{2} \Rightarrow n>2$. Thus we can have $\frac{3n+24}{n^3-6}<\frac{27n}{n^3/2}<\epsilon \Rightarrow n>\sqrt{\frac{54}{\epsilon}},\ \text{provided}\ n>2$.
	\begin{proof}
		Let $\epsilon>0$ and $N=\max\{2,\sqrt{\frac{54}{\epsilon}}\}$, then
		\begin{align*}
			n>N &\Rightarrow (n>\sqrt{\frac{54}{\epsilon}})\land(n>2)\\
			&\Rightarrow (\frac{27n}{n^3/2}<\epsilon)\land(\frac{n^3}{2}\leq n^3-6)\land(27n\geq3n+24)\\
			&\Rightarrow \frac{3n+24}{n^3-6} < \frac{27n}{n^3/2} < \epsilon\\
			&\Rightarrow \abs{\frac{4n^3+3n}{n^3-6}-4} < \epsilon
		\end{align*}
		This proofs $\lim\frac{4n^3+3n}{n^3-6}=4$ according to the definition of the limit \ref{def:limit}.
	\end{proof}
	
	\begin{eg}
		Show that $a_n=(-1)^n$ does not converge.
	\end{eg}
	\emph{Discussion}. Assume $\lim(-1)^n=a$, and we can see that no matter what $a$ is, either $1$ or $-1$ is at least
	$1$ from $a$, so it means $\abs{(-1)^n-a}<1$ will not hold for all large $n$.
	\begin{proof}
		Suppose $\lim(-1)^n=a$ and $\epsilon = 1$. By \ref{def:limit}, $\abs{(-1)^n-a}<1 \Rightarrow (|1-a|<1)\land(|-1-a|<1)$.
		Now by \ref{def:tri-ineq}, $2=|1-a+a-(-1)|\leq|1-a|+|a-(-1)|<1+1=2$, which is a contradiction. 
	\end{proof}
	
	\begin{eg}
		Let $(s_n)$ be a sequence of nonnegative real numbers and suppose $s=\lim s_n$. Note $s\geq0$. Prove $\lim\sqrt{s_n}=\sqrt{s}$
	\end{eg}
	\begin{proof}
		There are two cases.
		\begin{enumerate}
			\item $s>0$: Let $\epsilon>0$. $\lim s_n=s \Rightarrow (\exists N,\ n>N \Rightarrow |s_n-s|<\sqrt{s}\epsilon)$.
			$n>N$ also implies
			\begin{displaymath}
				|\sqrt{s_n}-\sqrt{s}|=\frac{(\sqrt{s_n}-\sqrt{s})(\sqrt{s_n}+\sqrt{s})}{\sqrt{s_n}+\sqrt{s}}=\frac{|s_n-s|}{\sqrt{s_n}+\sqrt{s}}\leq\frac{|s_n-s|}{\sqrt{s}}<\frac{\sqrt{s}\epsilon}{\sqrt{s}}=\epsilon
			\end{displaymath}
			\item $s=0$: EXERCISE 8.3 
		\end{enumerate}
	\end{proof}
	
	\begin{eg}
		Let $(s_n)$ be a convergent sequence of real numbers such that $s_n\neq0$ for all $n\in \mathbb{N}$ and $\lim s_n=s
		\neq0$. Prove $\inf\{|s_n|: n\in\mathbb{N}\}>0$
	\end{eg}
	\begin{proof}
		Let $\epsilon=\frac{|s|}{2}$. Since $\lim s_n=s$,
		\begin{displaymath}
			n>N \Rightarrow |s_n-s|<\frac{|s|}{2} \Rightarrow |s_n|\geq\frac{|s|}{2}
		\end{displaymath}
		The last implication is because otherwise 
		\begin{displaymath}
			|s|=|s-s_n+s_n|\leq|s-s_n|+|s_n|<\frac{|s|}{2}+\frac{|s|}{2}=|s|
		\end{displaymath}
		which is a contradiction. Now if we set $m=\min\{\frac{|s|}{2},|s_1|,|s_2|,\dots,|s_N|\}$, then clearly we have
		$m>0$ since and $|s_n|\geq m$ for all $n\in \mathbb{N}$. Thus $\inf\{|s_n|: n\in\mathbb{N}\}\geq m>0$\textbf{WHY???} 
	\end{proof}
	\newpage
	
	\section{Limit Theorems for Sequences}
	\begin{dfn}\label{def:bound}
		A sequence $(s_n)$ is said to be \emph{bounded} if $\exists M,\ \forall n,\ \text{such that}\ |s_n|\leq M$
	\end{dfn}
	
	\begin{thm}\label{def:convergence is bounded}
		Convergent sequences are bounded.
	\end{thm}
	\begin{rem}
		In other words, unbounded sequences are not convergent.
	\end{rem}
	
	\begin{thm}
		If the sequence $(s_n)$ converges to $s$ and $k\in\mathbb{R}$, then $(ks_n)$ converges to $ks$. i.e. $\lim(ks_n)=k\cdot\lim s_n$.
	\end{thm}
	
	\begin{thm}\label{def:addition}
		If $(s_n)$ and $(t_n)$ converge to $s$ and $t$, then $(s_n+t_n)$ converges to $s+t$. That is,
		\begin{displaymath}
			\lim(s_n+t_n)=\lim s_n+\lim t_n.
		\end{displaymath} 
	\end{thm}
	
	\begin{thm}\label{def:multiplication}
		If $(s_n)$ and $(t_n)$ converge to $s$ and $t$, then $(s_nt_n)$ converges to $st$. That is,
		\begin{displaymath}
			\lim(s_nt_n)=(\lim s_n)(\lim t_n)
		\end{displaymath} 
	\end{thm}
	
	\begin{lem}
		If $(s_n)\rightarrow s\neq0$ and $s_n\neq0$ and  for all $n$, then $\inf\{|s_n|: n\in\N\}>0$.
	\end{lem}
	
	\begin{lem}\label{def:reciprocal}
		If $(s_n)$ converges to $s$, $s_n\neq 0$ for all $n$, and $s\neq 0$, then $(1/s_n)$ converges to $1/s$.
	\end{lem}
	
	
	\begin{thm}
		Suppose $(s_n)$ and $(t_n)$ converge to $s$ and $t$. If $s\neq 0$ and $s_n\neq 0$ for all $n$, then $(t_n/s_n)$ converges to t/s.
	\end{thm}
	
	\begin{thm}
		\begin{enumerate}[(a)]
			\item[]
			\item $\lim_{n\rightarrow\infty}(\frac{1}{n^p})=0$ for $p>0$.
			\item $\lim_{n\rightarrow\infty}a^n=0$ if $|a|<1$.
			\item $\lim(n^{1/n})=1$.
			\item $\lim_{n\rightarrow\infty}a^{1/n}=1$ for $a>0$.
		\end{enumerate}
	\end{thm}
	\newpage
	
	\begin{dfn}
		For a $(s_n)$, we write $\lim s_n=+\infty$ provided for each $M>0$ there is a number $N$ wuch that $n>N\Rightarrow s_n>M$. Similarly, we write $\lim s_n=-\infty$ provided for each $M<0$ there is a number $N$ wuch that $n>N\Rightarrow s_n<M$.
	\end{dfn}
	This implies that if $\lim s_n>-\infty$, $\exists T,\ \forall n, s_n>T$. $\lim s_n<\infty$, $\exists T,\ \forall n, s_n<T$. 
	Be careful that we say $\lim s_n=+\infty$ as $(s_n)$ \textbf{diverges} to $\infty$, \textbf{not converge} to $\infty$.
	
	\begin{thm}
		Let $\lim s_n=+\infty$ and $\lim t_n>0$. Then $\lim s_nt_n=+\infty$.
	\end{thm}
	
	\begin{thm}
		For a $(s_n)$ of \emph{positive} real numbers, we have $\lim s_n=+\infty$ if and only if $\lim(\frac{1}{s_n})=0$.
	\end{thm}
	
	\begin{thm}
		Assume all $s_n\neq0$ and that the limit $L=\lim\left|\frac{s_{n+1}}{s_n}\right|$ exists.
		\begin{enumerate}[(a)]
			\item If $L<1$, then $\lim s_n=0$.
			\item If $L>1$, then $\lim |s_n|=+\infty$.
		\end{enumerate} 
	\end{thm}
	
	\newpage
	\section{Monotone Sequences and Cauchy Sequence}
	\begin{dfn}
		$(s_n)$ is called an \emph{increasing sequence (or nondecreasing)} if $\forall n,\ s_n\leq s_{n+1}$ and $s_n\leq s_m\ \text{whenever}\ n<m$.
		Similarly, $(s_n)$ is called an \emph{decreasing sequence (or nonincreasing)} if $\forall n,\ s_n\geq s_{n+1}$. An increasing or decreasing sequence is called \emph{monotone} or \emph{monotonic} sequence.
	\end{dfn}
	
	\begin{thm}\label{def:bounded monotone seq}
		All bounded monotone sequences converge.
	\end{thm}
	\begin{rem}
		From the proof procedure above, we can see that bounded monotone sequences \textbf{converge to its infimum or supremum}.
	\end{rem}
	
	\begin{thm}\label{def:unbounded monotone seq}
		\begin{enumerate}[(i)]
			\item[]
			\item If $(s_n)$ is an unbounded increasing sequence, then $\lim s_n=+\infty$.
			\item If $(s_n)$ is an unbounded decreasing sequence, then $\lim s_n=-\infty$.
		\end{enumerate}
	\end{thm}
	\begin{cor}
		If $(s_n)$ is monotone, then $\lim s_n$ is always meaningful. i.e. $\lim s_n=s,\ +\infty,\ \text{or}\ -\infty$.
	\end{cor}
	
	Suppose $(s_n)$ is bounded. Define $u_n=\inf\{s_m: m\geq n\}$ and $v_n=\sup{s_m: m\geq n}$. Then observe that $(u_n)$ is nondecreasing and $(v_n)$ is nonincreasing since as $n$ increases, the set has fewer elements. i.e. we have fewer choices for infimum and supremum. In general, if $A\subseteq B$, then $\inf A\geq \inf B$ and $\sup A\leq \sup B$.
	
	\begin{dfn}
		Let $(s_n)$ be a sequence in $\mathbb{R}$, define
		\begin{itemize}
			\item $\lim\sup s_n=\lim\limits_{N\rightarrow\infty}\sup\{s_n: n>N\}$
			\item $\lim\inf s_n=\lim\limits_{N\rightarrow\infty}\inf\{s_n: n>N\}$
		\end{itemize}
	\end{dfn}
	If $(s_n)$ is not bounded above. $\sup\{s_n: n>N\}=+\infty$ for all $N$ and we decree $\lim\sup s_n=+\infty$.
	Likewise, if $(s_n)$ is not bounded below. $\inf\{s_n: n>N\}=-\infty$ for all $N$ and we decree $\lim\inf s_n=-\infty$.
	
	Notice that $\lim\sup s_n$ need not equal to $\sup\{s_n: n>N\}$, but $\lim\sup s_n\leq\sup\{s_n: n>N\}$.
	\begin{rem}
		Since $v_n$ and $u_n$ are monotone, $\lim v_n=\lim\sup s_n$ and $\lim u_n=\lim\inf s_n$ always exist.
	\end{rem}
	\newpage
	
	\begin{thm}\label{def:condition for limit}
		Let $(s_n)$ be a sequence in $\mathbb{R}$.
		\begin{enumerate}[(i)]
			\item If $\lim s_n$ is defined, then $\lim\inf s_n=\lim s_n=\lim\sup s_n$.
			\item If $\lim\inf s_n=\lim\sup s_n$, then $\lim s_n$ is defined and $\lim s_n=\lim\inf s_n=\lim\sup s_n$.
		\end{enumerate}
	\end{thm}
	
	\begin{dfn}\label{def:cauchy-seq}
		A $(s_n)$ is called a \emph{Cauchy sequence} if 
		\begin{displaymath}
			\forall\epsilon>0,\ \exists N\ \text{such that}\ m,n>N\Rightarrow|s_n-s_m|<\epsilon
		\end{displaymath}
	\end{dfn}
	\begin{lem}
		Convergent sequences are Cauchy sequences.
	\end{lem}
	
	\begin{lem}
		Cauchy sequences are bounded.
	\end{lem}
	
	\begin{thm}\label{def:cauchy iff convergent}
		A sequence in $\R$ is a convergent sequence if and only if it is a Cauchy sequence.
	\end{thm}
	\newpage
	
	\section{Subsequences}
		\begin{dfn}
			Suppose $(s_n)_{n\in\mathbb{N}}$ is a sequence. A \emph{subsequence} of this sequence is $(t_k)_{k\in\mathbb{N}}$ where for each $k$ there is a positive integer $n_k$ such that
			\begin{equation*}
				n_1<n_2<\cdots<n_k<n_{k+1}<\cdots
			\end{equation*}
			and
			\begin{equation*}
				t_k=s_{n_k}.
			\end{equation*}
			Thus $(t_k)$ is just a selection of some [possibly all] of the $s_n$'s taken in order.
		\end{dfn}
		For the subset $\{n_1,n_2,\dots\}$ there is a natural function $\sigma$ given by $\sigma(k)=n_k$ for $k\in\mathbb{N}$. The function $\sigma$ "selects" an infinite subset of $\mathbb{N}$ in order. Then
		the subsequence of $s$ corresponding to $\sigma$ is simply the composite function $t=s\circ\sigma$. That is
		\begin{displaymath}
			t_k=t(k)=s\circ\sigma(k)=s(\sigma(k))=s(n_k)=s_{n_k}\quad\text{for}\quad k\in N.
		\end{displaymath}
		Notice that $\sigma$ needs to be an \emph{increasing} function.
		 
		Recall that the set $\Q$ of rational numbers is \emph{countable}: there is a bijection from $\N$ to $\Q$. Therefore we have a sequence $(q_n)=(q_1,q_2,q_3,\dots)$ such that $\{q_n: n\in\N\}=\Q$. Then we have the following proposition:
		\begin{thm}
			Let $(q_n)$ be an enumeration of $\Q$. Then for any $a\in\R$, there exists a subsequence $(q_{n_k})$ of $(q_n)$ such that $q_{n_k}\rightarrow a$.
		\end{thm}
	
		\setcounter{equation}{0}
		\begin{thm}\label{def:limit-subseq}
			Let $(s_n)$ be a sequence in $\R$.
			\begin{enumerate}[(i)]
				\item If $t$ is in $\mathbb{R}$ then there is a subsequence of $(s_n)$ converging to $t$ if and only if
				the set $\{n\in\mathbb{N}: |s_n-t|<\epsilon\}$ is \emph{infinite} for all $\epsilon>0$.
				\item If $(s_n)$ is unbounded above, it has a subsequence with limit $+\infty$.
				\item If $(s_n)$ is unbounded below, it has a subsequence with limit $-\infty$.
			\end{enumerate}
			In each case, the subsequence can be taken to be \emph{monotonic}.
		\end{thm}
		
		\begin{thm}\label{def:subsequence converges to the same limit}
			If $(s_n)$ in $\R$ converges, then every subsequence converges to the same limit. If there are two subsequences of $(s_n)$ with different limits, $(s_n)$ does not converge.
		\end{thm}
		
		\begin{thm}
			Every sequence $(s_n)$ in $\R$ has a monotonic subsequence.
		\end{thm}
		\setcounter{equation}{0}
		
		\begin{thm}[Bolzano-Weierstrass Theorem]\label{def:B-W}
			Every bounded sequence in $\R$ has a convergent subsequence.
		\end{thm}
		
		\begin{dfn}
			Let $(s_n)$ be a sequence in $\mathbb{R}$. A \emph{subsequential limit} is any real number or symbol $+\infty$ or $-\infty$ that is the limit of some subsequence of $(s_n)$.
		\end{dfn}
		
		\setcounter{equation}{0}
		\begin{thm}\label{def:subsequence with limit limsup or liminf}
			Let $(s_n)$ be any sequence. There exists a monotonic subsequence whose limit is $\lim\sup s_n$, and there exists a monotonic subsequence whose limit is $\lim\inf s_n$.
		\end{thm}
		
		\begin{thm}\label{def:subsequential limit condition}
			Let $(s_n)$ be any sequence in $\mathbb{R}$, and let $S$ denote the set of subsequential limits of $(s_n)$.
			\begin{enumerate}[(i)]
				\item S is nonempty.
				\item $\sup S=\lim\sup s_n$ and $\inf S=\lim\inf s_n$.
				\item $\lim s_n$ exists if and only if $S$ has exactly one element, namely $\lim s_n$.
				\item $\lim\sup s_n\in S$ and $\lim\inf s_n\in S$.
			\end{enumerate}
		\end{thm}
		
		\begin{thm}
			Let $S$ denote the set of subsequential limits of a sequence $(s_n)$. Suppose $(t_n)$is a sequence in $S\cap\mathbb{R}$ and that $t=\lim t_n$. Then $t$ belongs to $S$.
		\end{thm}
	
		\newpage
		
	\section{lim sup's and lim inf's}
		\setcounter{equation}{0}
		\begin{thm}\label{thm:12.1}
		If $(s_n)$ converges to a positive real number $s$ and $(t_n)$ is any sequence, then
		\begin{displaymath}
			\lim\sup s_nt_n=s\cdot\lim\sup t_n.
		\end{displaymath}
		Here we allow the conventions $s\cdot(+\infty)=+\infty$ and $s\cdot(-\infty)=-\infty$ for $s>0$.
		\end{thm}
		
		\setcounter{equation}{0}
		\begin{thm}\label{thm:12.2}
		Let $(s_n)$ be any sequence of nonzero real numbers. Then we have
		\begin{displaymath}
		\lim\inf\abs{\frac{s_{n+1}}{s_n}}\leq\lim\inf|s_n|^{1/n}\leq\lim\sup|s_n|^{1/n}\leq\lim\sup\abs{\frac{s_{n+1}}{s_n}}
		\end{displaymath}
		\end{thm}
		
		\begin{cor}\label{def:12.3}
		If $\lim\abs{\frac{s_{n+1}}{s_n}}$ exists [and equals L], then $\lim|s_n|^{1/n}$ exists [and equals L].
		\end{cor}
		
		\newpage
	\section{Some Topological Concepts in Metric Spaces}
	\begin{dfn}
		Let $X$ be a set, and suppose $d$ is a function $d: X\times X\rightarrow[0,\infty]$ defined for all pairs $(x,y)$ of elements from $X$ satisfying
		\begin{enumerate}
			\item $d(x,x)=0$ for all $x\in S$ and $d(x,y)>0$ for distinct $x,y\in X$. (Positive Definiteness)
			\item $d(x,y)=d(y,x)$ for all $x,y\in X$. (Symmetry)
			\item $d(x,z)\leq d(x,y)+d(y,z)$ for all $x,y,z\in X$. (Triangle Inequality)
		\end{enumerate}
		Such a function $d$ is called a \emph{distance function} or a \emph{metric} on $X$. A \emph{metric space} $X$ is a set $X$ together with a metric on it.
	\end{dfn}
	\begin{rem}
		The positive definiteness can be also expressed as $\forall x,y\in X\ d(x,y)\geq 0$ and $d(x,y)=0\iff x=y$. The distance function cannot be $+\infty$.
	\end{rem}
	\begin{eg}
		Discrete metric space is defined as
		\begin{displaymath}
			\text{For any set $X$ with metric or distance function as}\begin{cases}
				1\quad\text{$x\neq y$}\\ 0\quad\text{$x=y$}.
			\end{cases}
		\end{displaymath}
		Notice that all sets in discrete metric space are both open and closed.
	\end{eg}
	
	\begin{dfn}[Convergence]
		A sequence $(x_n)$ in a metric space $(X,d)$ converges to $x$ in $X$ if $\lim_{n\rightarrow\infty}d(s_n,s)=0$.
	\end{dfn}
	\begin{rem}
		In other words, a sequence $(x_n)$ converges to $x$ if for any $\epsilon>0$, there exists $N\in\N$ such that $n\geq N\implies d(x_n,x)<\epsilon$.
	\end{rem}
	
	\begin{dfn}[Cauchy]
		A sequence $(x_n)$ in $X$ is a \emph{Cauchy} if for any $\epsilon>0$ there exists an $N\in\N$ such that
		\begin{displaymath}
			m,n\geq\implies d(x_m,x_n)<\epsilon.
		\end{displaymath}
	\end{dfn}
	
	\begin{dfn}[Complete]
		A metric space $(X,d)$ is \emph{complete} if every Cauchy sequence in $X$ converges.
	\end{dfn}
	\begin{rem}
		Every convergent sequence $(x_n)$ in $X$ is Cauchy.
	\end{rem}
	
	\begin{dfn}[Open Ball]
		Let $(X,d)$ be a metrc space. For $x\in X$ and $r>0$, the open ball of radius $r$ centered at $x$ is the set
		\begin{displaymath}
			B_r(x)=\{y\in X: d(y,x)<r\}
		\end{displaymath}
	\end{dfn}
	
	\begin{dfn}[Interior Point]
		Let $(X,d)$ be a metric space. Let $E$ be a subset of $X$. An element $x\in E$ is \emph{interior} to $E$ if for some $r>0$ we have
		\begin{displaymath}
			B_r(x)\subseteq E
		\end{displaymath}
		We write $E^\circ$ for the set of points in $E$ that are interior to $E$.
	\end{dfn}
	\begin{rem}
		\begin{itemize}
			\item The relationship between $E$ and $X$ may affect whether a point in $E$ is interior to $E$. For example, for $E=[0,1]\subset[-1,2]=X$, $0$ is not interior to $[0,1]$. However if $E=[0,1]\subset[0,1]=X$, then $0$ is interior to $0$ since there is not point in $X$ beyond the left of $0$.
			\item $E^\circ$ is open.
			\item $E=E^\circ$ if and only if $E$ is open.
			\item If $F$ is an open set such that $F\subseteq E$, then $F\subseteq E^\circ$.
		\end{itemize}		
	\end{rem}
	
	\begin{dfn}[Open Set]
		A set $E\subseteq X$ is \emph{open} if every point $x\in E$ is an interior point of $E$. i.e., if $E=E^\circ$.	
	\end{dfn}
	\begin{rem}
		\begin{itemize}
			\item[]
			
			\item A set being open does \textbf{not} mean it is \textbf{not} closed. e.g. $[0,1)$ is neither open nor closed.
		\end{itemize}
	\end{rem}
	\begin{eg}
		\begin{itemize}
			\item[]
			\item $(a,b),(a,\infty),(-\infty,a)$ are open sets.
			\item In $\R$, $\Q$ is \emph{not} open since $B_r(q)$ may contain irrational numbers in $\R$ so $B_r(q)\nsubseteq\Q$.
			\item In any metric space $(X,d)$, $X$ and $\Q$ are open trivially.
		\end{itemize}
	\end{eg}
	
	\begin{thm}[Open ball is open]
		Let $(X,d)$ be a metric space. Given $x\in X$ and $r>0$, $B_r(x)$ is an open set in $X$.
	\end{thm}
	\begin{proof}
		Consider arbitrary $y\in B_r(x)$ and let $s=r-d(x,y)$. It is easy to show that $B_x(y)\subseteq B_r(x)$. Thus $y$ is an interior point of $B_r(x)$. Since $y$ is arbitrary, by the definition $B_r(x)$ is open.
	\end{proof}
	
	\begin{thm}[Union and intersection of open sets]\label{thm: Union and intersection of open sets}
		Let $(X,d)$ be a metric space.
		\begin{enumerate}[(i)]
			\item If $\{\mathcal{U}_\alpha\}_{\alpha\in \mathcal{A}}$ is any collection of open sets in $X$, then $\bigcup_{\alpha\in \mathcal{A}}\mathcal{U}_\alpha$ is open. i.e. the union of \emph{any} collection of open sets is open.
			\item If $\{\mathcal{U}_1,\dots,\mathcal{U}_n\}$ is a finite collection of open sets in $X$, then $\bigcap_{i=1}^{n}\mathcal{U}_i$ is open.
		\end{enumerate}
	\end{thm}
	\begin{proof}
		\begin{enumerate}[(i)]
			\item[]
			\item Consider $x\in\bigcup_{\alpha\in \mathcal{A}}\mathcal{U}_\alpha$, then $\exists \beta\in\mathcal{A}$ such that $x\in\mathcal{U}_\beta$. Since $\mathcal{U}_\beta$ is open, $\exists r>0$ such that $B_r(x)\subseteq\mathcal{U}_\beta\subseteq\bigcup_{\alpha\in \mathcal{A}}\mathcal{U}_\alpha$. Thus $x$ is interior to $\bigcup_{\alpha\in \mathcal{A}}\mathcal{U}_\alpha$, completing the proof.
			\item Consider $x\in\bigcap_{i=1}^{n}\mathcal{U}_i$. Since $x\in\mathcal{U}_i$ for $i=1,\dots,n$, $\exists r_i>0$ such that $B_{r_i}(x)\subseteq\mathcal{U}_i$. Take $r=\min\{r_1,\dots,r_n\}$, then clearly $B_r(x)\subseteq\bigcap_{i=1}^{n}\mathcal{U}_i$.
		\end{enumerate}
	\end{proof}
	\begin{rem}
		The examples for infinite collection in (ii) is $\bigcap_{n=1}^{\infty}(1-\frac{1}{n}, 1+\frac{1}{n})=\{1\}$. Since $1$ is not an interior point of $\{1\}$, $\{1\}$ is not open.
	\end{rem}
	
	\begin{dfn}[Complement]
		For a set $E\subseteq X$, the \emph{complement} of $E$ is the set $E^C=X\backslash E=\{x\in X: x\notin E\}$.
	\end{dfn}
	
	\begin{dfn}[Limit Point]
		For a set $E\subseteq X$, a point $x\in X$ is a \emph{limit point} of $E$ if for any $r>0$, we have that $(B_r(x)\backslash\{x\})\cap E\neq\emptyset$.\smallskip
		
		$E^\prime$ denotes the set of all limit points of $E$.
	\end{dfn}
	\begin{rem}
		\begin{itemize}
			\item[]
			\item In other words, for any radius $r>0$, no matter how small is $r$, there is some element of $E$ which sits in $B_r(x)$ other than $x$ itself. 
			\item If $E\subseteq F$, then $E^\prime\subseteq F^\prime$.
			\item $(E\cup F)^\prime=E^\prime\cup F^\prime$.
		\end{itemize}
	\end{rem}
	\begin{eg}
		\begin{itemize}
			\item[]
			\item In $\R$, the set of limit points of $(0,1)$ is $[0,1]$.
			\item In $\R$, the only limit point of $\{\frac{1}{n}: n\in\N\}$ is $0$.
			\item In $\R$, the set of limit point of $\Q$ is $\R$. 
		\end{itemize}
	\end{eg}
	
	\begin{thm}
		A point $x$ is a limit point of a set $E\subseteq X$ if and only if $x=\lim x_n$ for some sequence $x_n$ of points in $E\backslash\{x\}$. 
	\end{thm}
	\begin{proof}
		See homework 3.7.
	\end{proof}
	
	\begin{dfn}[Isolated Point]
		For a set $E\subseteq X$, $x\in E$ is called an \emph{isolated point} if $x$ is not a limit point of $E$
	\end{dfn}
	\begin{rem}
		In other words, $x$ is an isolated point or not a limit point of $E$ if there exists a radius $r$ such that $B_r(x)$ does not contain any element of $E$ except $x$ itself.
	\end{rem}
	\begin{eg}
		\begin{itemize}
			\item[]
			\item In $\R$, every integer is an isolated point of $\mathbb{Z}$.
			\item In $\R$, the set $Q$ has no isolated point.
			\item In $\R$, every element of $\{\frac{1}{n}: n\in\N\}$ is an isolated point. 
		\end{itemize}
	\end{eg}
	
	\begin{dfn}[Closed Set]
		A set is \emph{closed} if $E^\prime\subseteq E$.
	\end{dfn}
	\begin{dfn}[Closed Set]
		Let $(X,d)$ be a metric space. A subset $E$ of $X$ is \emph{closed} if its complement $E^\com$ is an open set.
	\end{dfn}
	\begin{rem}
		\begin{itemize}
			\item[]
			\item The above two definitions are equivalent.
			\item In other words, $E$ contains all of its limit points, or every limit point of $E$ is in $E$.
			\item In any metric space $(X,d)$, $X$ and $\emptyset$ are closed.
			\item A set being closed does \textbf{not} mean it is \textbf{not} open. e.g. $[0,1)$ is neither open nor closed.
		\end{itemize}
	\end{rem}
	\begin{eg}
		\begin{itemize}
			\item In $\R$, $[0,1]$ is closed. $[a,\infty),(-\infty,a]$ are closed.
			\item In $\R$, the set $\{\frac{1}{n}: n\in\N\}$ is not closed, but $\{\frac{1}{n}: n\in\N\}\cup\{0\}$ is closed.
			\item In any metric space, $X$ and $\emptyset$ are closed.
			\item All finite sets do not have limit point, so they are trivially closed.
		\end{itemize}
	\end{eg}
	
	\begin{thm}
		A set $E\subseteq \R$ is closed if and only if every Cauchy sequence contained in $E$ has a limit that is also an element of $E$.
	\end{thm}
	
	\begin{thm}[The set of limit points is closed]
		Let $(X,d)$ be a metric space. Let $E\subseteq X$, then $E^\prime$, (the set of limit points of $E$), is closed.
	\end{thm}
	\begin{proof}
		We need to show for any limit point $x$ of $E^\prime$, $x$ is in $E^\prime$. Since $x$ is a limit point of $E^\prime$, $\forall r>0$, $(B_r(x)\backslash\{x\})\cap E^\prime\neq\emptyset$. i.e. there exists $y\in E^\prime$ such that $y\neq x$ and $y\in B_r(x)$. Take $s=\min\{r-d(x,y), d(x,y)\}$. Since $y\in E^\prime$, $(B_s(y)\backslash\{y\})\cap E\neq\emptyset$. i.e. $\exists z\in(B_s(y)\backslash\{y\})\cap E\neq\emptyset$.\smallskip
		
		Now since $s<r-d(x,y)$, $d(x,z)\leq d(x,y)+d(y,z)<d(x,y)+(r-d(x,y))=r\implies z\in B_r(x)$. Also since $s<d(x,y)$, $z\neq x$. Thus $z\in(B_r(x)\backslash\{x\})\cap E\implies (B_r(x)\backslash\{x\})\cap E\neq\emptyset$, which implies $x$ is a limit point of $E$. i.e. $x\in E^\prime$, completing the proof. 
	\end{proof}
	
	\begin{thm}[Union and intersection of closed sets]
		\begin{enumerate}[(i)]
			\item[]
			\item If $\{\mathcal{E}_\alpha\}_{\alpha\in\mathcal{A}}$ is any collection of closed set, then $\bigcap_{\alpha\in\mathcal{A}}\mathcal{E}_\alpha$ is closed.
			\item If $\{\mathcal{E}_1,\dots,\mathcal{E}_n\}$ is a finite collection of closed sets in $X$, then $\bigcup_{i=1}^{n}\mathcal{E}_i$ is closed.
		\end{enumerate}
	\end{thm}
	\begin{proof}
		\begin{enumerate}[(i)]
			\item[]
			\item Observe that $\left(\bigcap_{\alpha\in\mathcal{A}}\mathcal{E}_\alpha\right)^\com=\bigcup_{\alpha\in\mathcal{A}}\mathcal{E}_\alpha^\com$. Since $\mathcal{E}_\alpha$ is closed, $\mathcal{E}_\alpha^\com$ is open. By \ref{thm: Union and intersection of open sets}, the union of open sets $\bigcup_{\alpha\in\mathcal{A}}\mathcal{E}_\alpha^\com$ is open, completing the proof.
			\item Observe that $\left(\bigcup_{i=1}^{n}\mathcal{E}_i\right)^\com=\bigcap_{i=1}^{n}\mathcal{E}_i^\com$. Since $\mathcal{E}_i$ is closed, $\mathcal{E}_i^\com$ is open. By \ref{thm: Union and intersection of open sets}, the intersection of finite open sets $\bigcap_{i=1}^{n}\mathcal{E}_i^\com$ is open, completing the proof.
		\end{enumerate}
	\end{proof}
	\begin{rem}
		$\bigcup_{x\in(0,1)}\{x\}=(0,1)$ is an example to the union of infinite closed sets is open in (ii).
	\end{rem}
	\newpage
	
	The proof above uses one of DeMorgan's Laws for sets.
	\setcounter{equation}{0}
	\begin{tcolorbox}[title=\textbf{DeMorgan's Laws for sets}]
		Suppose a metric space $(X,d)$ and let $\forall\alpha\in\mathcal{A}\ U_\alpha\in X$. Then $\bigcap_{\alpha\in\mathcal{A}}\mathcal{U}_{\alpha}^{\com}=\left(\bigcup_{\alpha\in\mathcal{A}}\mathcal{U}_\alpha\right)^\com$.
		\tcblower
		\begin{proof}
			We want to show both directions.
			\begin{itemize}
				\item[$\subseteq$:] Consider $u\in\bigcap_{\alpha\in\mathcal{A}}\mathcal{U}_{\alpha}^{\com}$, then we have
				\begin{align}
					\forall\alpha\in\mathcal{A}\ u\in\mathcal{U}_{\alpha}^{\com}&\implies\forall\alpha\in\mathcal{A}\ u\notin\mathcal{U}_{\alpha}\\
					&\implies u\notin \bigcup_{\alpha\in\mathcal{A}}\mathcal{U}_\alpha\\
					&\implies u\in\left(\bigcup_{\alpha\in\mathcal{A}}\mathcal{U}_\alpha\right)^\com.
				\end{align}
				(1) $\implies$ (2) because
				\begin{displaymath}
					\left(\neg\left(u\in\mathcal{U}_1\right)\right)\land\left(\neg\left(u\in\mathcal{U}_2\right)\right)\land\cdots=\neg\left((u\in\mathcal{U}_1\lor(u\in\mathcal{U}_2)\lor\cdots\right)=\neg\left(u\in\bigcup\mathcal{U}_i\right)
				\end{displaymath} 
				Thus $\bigcap_{\alpha\in\mathcal{A}}\mathcal{U}_{\alpha}^{\com}\subseteq\left(\bigcup_{\alpha\in\mathcal{A}}\mathcal{U}_\alpha\right)^\com$.
				\item[$\supseteq$:] Consider $u\in\left(\bigcup_{\alpha\in\mathcal{A}}\mathcal{U}_\alpha\right)^\com$, then we have
				\begin{align*}
					u\notin\bigcup_{\alpha\in\mathcal{A}}\mathcal{U}_\alpha &\implies\forall\alpha\in\mathcal{A}\ u\notin\mathcal{U}_\alpha\\
					&\implies\forall\alpha\in\mathcal{A}\ u\in\mathcal{U}_{\alpha}^{\com}\\
					&\implies u\in\bigcap_{\alpha\in\mathcal{A}}\mathcal{U}_{\alpha}^{\com}
				\end{align*}
				Thus $\bigcap_{\alpha\in\mathcal{A}}\mathcal{U}_{\alpha}^{\com}\supseteq\left(\bigcup_{\alpha\in\mathcal{A}}\mathcal{U}_\alpha\right)^\com$, and hence $\bigcap_{\alpha\in\mathcal{A}}\mathcal{U}_{\alpha}^{\com}=\left(\bigcup_{\alpha\in\mathcal{A}}\mathcal{U}_\alpha\right)^\com$.
			\end{itemize}
		\end{proof}
	\end{tcolorbox}\medskip
	
	\begin{dfn}[Bounded Set]
		A set $E\subseteq X$ is bounded if for some $x\in X$ and $M>0$ such that $d(x,y)\leq M$ for all $y\in E$.
	\end{dfn}
	\begin{rem}
		\begin{itemize}
			\item[]
			\item In $\R^k$, $X\subseteq\R^k$ is bounded if there exists $M>0$ such that $\forall\mathbf{x}\in X\ d(\mathbf{x},\mathbf{0})=\sqrt{x_1^2+\cdots+x_k^2}\leq M$.
			\item Finite union of bounded sets is bounded.
			\item Intersection of bounded sets is bounded.
			\item Contained in some open ball.
		\end{itemize}
	\end{rem}
	
	\begin{thm}
		In $R$, any closed and bounded sets always have maximum and minimum.
	\end{thm}
	
	\begin{dfn}[Closure]
		The \emph{closure} of $E$ in $X$ is $\bar{E}=E\cup E^\prime$.
	\end{dfn}
	\begin{rem}
		\begin{itemize}
			\item[]
			\item $\bar{E}$ is the intersection of all closed sets containing $E$.
			\item $\bar{E}$ is closed.
			\item $E=\bar{E}$ if and only if $E$ is closed.
			\item If $F$ is a closed set such that $E\subseteq F$, then $\bar{E}\subseteq F$.
			\item The union of closures of finite sets is equal to the closure of unions of the sets. i.e. $\overline{A\cup B}=\overline{A}\cup\overline{B}$.
		\end{itemize}
	\end{rem}
	
	\begin{thm}
		For any $E\subseteq X$, its closure $\bar{E}=E\cup E^\prime$ is closed and is the smallest closed set containing $A$.
	\end{thm}
	
	\begin{dfn}[Dense Set]
		A set $E\subseteq X$ is \emph{dense} in $X$ if $\bar{E}=X$.
	\end{dfn}
	\begin{eg}
		\begin{itemize}
			\item[]
			\item $\Q$ is dense in $\R$.
			\item In any metric space $(X,d)$, $X$ is dense in $X$. 
		\end{itemize}
	\end{eg}
	
	\begin{dfn}[Dense Set]
		A set $E\subseteq X$ is dense in $X$ if and only if for any $x\in X$ and $r>0$.
		\begin{displaymath}
			B_r(x)\cap E\neq\emptyset.
		\end{displaymath} 
	\end{dfn}
	
	\setcounter{equation}{0}
	\begin{lem}\label{def:sequence in R^k}
		\begin{itemize}
			\item[]
			\item A sequence $(\mathbf{x}^{(n)})$ in $\mathbb{R}^k$ converges to $\mathbf{x}=(x_1,\dots,x_k)$ if and only if for each $j=1,2\dots,k$, the sequence $(x_j^{(n)})$ converges in $\mathbb{R}$.
			\item A sequence $(\mathbf{x}^{(n)})$ in $\mathbb{R}^k$ is a Cauchy sequence if and only if each sequence $(x_j^{(n)})$ is a Cauchy sequence in $\mathbb{R}$.
		\end{itemize}	
	\end{lem}
	\begin{proof}
		First observe for $\textbf{x,y}\in\mathbb{R}^k$ and $j=1,\dots,k$
		\begin{align}
			|x_j-y_j|=\sqrt{(x_j-y_j)^2}\leq\sqrt{(x_1-y_1)^2+\cdots+(x_k-y_k)^2}&=d(\textbf{x,y})\notag\\
			&\leq \sqrt{k}\max\{|x_j-y_j|: j=1,\dots,k\}
		\end{align}
		First assertion: 
		\begin{itemize}
			\item[$\implies$:] Given that $(\textbf{x}^{(n)})$ converges to $\mathbf{x}$. For each $epsilon>0$ there exists $N\in\N$ such that $n\geq N\implies d(\textbf{x}^{(n)},\mathbf{x})<\epsilon$. Then by (1) for $j=1,\dots,k$
			\begin{displaymath}
				n\geq N\implies |x_j^{(n)}-x_j|\leq d(\textbf{x}^{(n)},\mathbf{x})<\epsilon,
			\end{displaymath}
			so $x_j^{(n)}\rightarrow x_j$.
			\item[$\impliedby$:] For $j=1,\dots,k$, $\forall \epsilon>0$, there exists $N_j\in\N$ such that 
			\begin{displaymath}
				n\geq N_j\implies |x_j^{(n)}-x_j|<\frac{\epsilon}{\sqrt{k}}.
			\end{displaymath}
			Take $N=\max\{N_1,\dots,N_k\}$, then by (1) we have
			\begin{displaymath}
				n\geq N\implies d(\textbf{x}^{(n)},\mathbf{x})\leq\sqrt{k}\max\{|x_j-y_j|: j=1,\dots,k\}<\sqrt{k}\cdot\frac{\epsilon}{\sqrt{k}}=\epsilon.
			\end{displaymath}
			Thus $(\textbf{x}^{(n)})\rightarrow\mathbf{x}$
		\end{itemize}
		Second assertion:	
		\begin{itemize}
			\item[$\Rightarrow$]: Suppose $(\textbf{x}^{(n)})$ is a Cauchy sequence, from the definition we know
			\begin{displaymath}
				m,n>N\Rightarrow d(\textbf{x}^{(m)},\textbf{x}^{(n)})<\epsilon
			\end{displaymath}
			From (1) we see
			\begin{displaymath}
				m,n>N\Rightarrow|x_j^{(m)}-x_j^{(n)}|<\epsilon
			\end{displaymath}
			so $(x_j^{(n)})$ is a Cauchy sequence.
			\item[$\Leftarrow$]: Suppose $(x_j^{(n)})$ is a Cauchy sequence, then for $j=1,\dots,k$
			\begin{displaymath}
				m,n>N_j\Rightarrow|x_j^{(m)}-x_j^{(n)}|<\frac{\epsilon}{\sqrt{k}}.
			\end{displaymath}
			If $N=\max\{N_1,N_2,\dots,N_k\}$, then by (1)
			\begin{displaymath}
				m,n>N\Rightarrow d(\textbf{x}^{(m)},\textbf{y}^{(n)})<\epsilon
			\end{displaymath}
			i.e. $(\textbf{x}^{(n)})$ is a Cauchy sequence.
		\end{itemize}
	\end{proof}
	
	\begin{thm}
		Euclidean k-space $\mathbb{R}^k$ is complete.
	\end{thm}
	\begin{proof}
		Consider a Cauchy sequence $(\textbf{x}^{(n)})$ in $\mathbb{R}^k$. By \ref{def:sequence in R^k}, each $(x_j^{(n)})$ is a Cauchy sequence. By \ref{def:cauchy iff convergent} each $(x_j^{(n)})$ converges. Thus by \ref{def:sequence in R^k} $(\textbf{x}^{(n)})$ converges.
	\end{proof}
	
	\begin{thm}[Bolzano-Weierstrass in $\R^k$]
		Every bounded sequence in $\mathbb{R}^k$ has a convergent subsequence.
	\end{thm}
	\begin{proof}
		Since $(\textbf{x}^{(n)})$ is bounded, then each $(x_j^{(n)})$ is bounded in $\mathbb{R}$. By \ref{def:B-W}, we
		could replace $(\textbf{x}^{(n)})$ by one of its subsequence, say $(\bar{\mathbf{x}}^{(n)})$, whose $(x_1^{(n)})$
		converges. By \ref{def:B-W} again, we may replace $(\textbf{x}^{(n)})$ by a subsequence of $(\textbf{x}^{(n)})$ such
		that both $(x_1^{(n)})$ and $(x_2^{(n)})$ converge. $(x_1^{(n)})$ still converges because \ref{def:subsequence converges to the same limit}. Repeating this argument by $k$ times, we obtain a new sequence $(\textbf{x}^{(n)})$ where each $(x_j^{(n)})$ converges, $j=1,\dots,k$, which is a subsequence of the original sequence, and it converges by \ref{def:sequence in R^k}. 
	\end{proof}
	\begin{rem}
		In any general metric space $(X,d)$, it is not true that any bounded sequence has a convergent subsequence. E.g. $(\Q,d)$ and infinite discrete metric space
	\end{rem}
	
	\begin{thm}
		Let $E$ be a subset of a metric space $(S,d)$.
		\begin{enumerate}
			\item $E$ is closed $\iff$ $E=E^-$.
			\item $E$ is closed $\iff$ $E$ contains the limit of every convergent sequence of points in $E$.
			\item An element is in $E^-$ $\iff$ it is the limit of some sequence of points in $E$.
			\item A point is in the boundary of $E$ $\iff$ it belongs to the closure of both $E$ and its complement.
		\end{enumerate}
	\end{thm}
	
	\noindent\rule{\textwidth}{1pt}
	\subsection*{Compactness}
	
	\begin{dfn}[Open Cover]
		Let $(X,d)$ be a metric space and $E\subseteq X$. An open cover of $E$ is a collection of open sets $\{\mathcal{U}_\alpha\}_{\alpha\in\mathcal{A}}$ such that $E\subseteq\bigcup_{\alpha\in\mathcal{A}}\mathcal{U}_\alpha$. An open cover is finite if it contains finitely many sets.
	\end{dfn}
	
	\begin{dfn}[Subcover]
		A subcover of an open cover $\{\mathcal{U}_\alpha\}_{\alpha\in\mathcal{A}}$ of $E$ is an \emph{open cover} $\{\mathcal{U}_\alpha\}_{\alpha\in\mathcal{B}}$ such that $\mathcal{B}\subseteq\mathcal{A}$.
	\end{dfn}
	
	\begin{dfn}[Compact Set]
		A set $E\subseteq X$ is compact if every open cover of $E$ has a \emph{finite} subcover.
	\end{dfn}
	\begin{eg}
		\begin{itemize}
			\item[]
			\item Every finite set is compact.
			\item Infinite discrete metric space is not compact.
			\item $\R$ is not compact: $\{(-n,n)\}_{n\in\N}$ is an open cover of $\R$ but does not have a finite subcover.
			\item $(0,1)$ is not compact: $\{(0,r)\}_{r\in(0,1)} $ is an open cover of $(0,1)$ but does not have a finite subcover.
			\item Closed interval in $R$ is compact.
		\end{itemize}
	\end{eg}
	
	\begin{thm}
		Compact sets are closed in any metric space.
	\end{thm}
	\begin{proof}
		Let $E\subseteq X$ be compact. To show $E$ is closed, we can show $E^\com$ is open. Consider $x\in E^\com$. For each $y\in E$, let $r_y:=\frac{1}{2}d(x,y)$. Clearly $\{B_{r_y}(y)\}_{y\in E}$ is an open cover of $E$ because each point in $E$ is a center of an open ball. By the assumption, $E$ is compact, so there is a finite subcover $\{B_{r_{y_1}}(y_1),\dots,B_{r_{y_n}}(y_n)\}$ such that $E\subseteq\bigcup_{i=1}^{n}B_{r_{y_i}}(y_i)$. 
		
		Now take $r=\min\{r_{y_1},\dots,r_{y_n}\}$, and hence $B_r(x)\cap(\bigcup_{i=1}^{n}B_{r_{y_i}}(y_i))=\emptyset$. Since $E\subseteq\bigcup_{i=1}^{n}B_{r_{y_i}}(y_i)$, $B_r(x)\cap E=\emptyset\implies B_r(x)\subseteq E^\com$. Thus $x$ is an interior point of $E^\com$, completing the proof.
	\end{proof}
	\begin{rem}
		Non-closed sets are not compact in any metric space. Notice open set does not mean non-closed.
	\end{rem}
	
	\begin{thm}
		Closed subsets of compact sets are compact.
	\end{thm}
	\begin{proof}
		See worksheet 7.
	\end{proof}
	\begin{cor}
		If $\{K_\alpha\}_{\alpha\in A}$ is a collection of compact sets, then $\bigcap_{\alpha\in\mathcal{A}}K_\alpha$ is compact.
	\end{cor}
	\begin{proof}
		Since compact sets are closed, $\bigcap_{\alpha\in\mathcal{A}}K_\alpha$ is the intersection of closed sets, which is also closed. Since $\bigcap_{\alpha\in\mathcal{A}}K_\alpha$ is a subset of compact sets $U_\alpha$, it is compact.
	\end{proof}
	\begin{rem}
		Finite union of compact sets in $X$ is compact.
	\end{rem}
	
	\begin{thm}
		Every sequence in a compact set has a convergent subsequence.
	\end{thm}
	\begin{proof}
		See worksheet 7.
	\end{proof}
	
	\begin{thm}[Compact Set]
		A set $E\subseteq X$ is compact if and only if every sequence in $E$ has a convergent subsequence converging to a point in $E$.
	\end{thm}
	
	\begin{thm}[Nested Compact Sets Property]
		Let $(F_n)$ be a sequence of closed, bounded, nonempty sets in $\R^k$ such that $F_1\supseteq F_2\supseteq\cdots$, then $F=\bigcap_{n=1}^{\infty}F_n\neq\emptyset$ and $F$ is closed and bounded.
	\end{thm}
	
	\begin{thm}
		Suppose $\{E_\alpha\}_{\alpha\in\mathcal{A}}$ is a collection of compact sets such that $\bigcap_{\alpha\in\mathcal{B}}E_\alpha\neq\emptyset$ for any finite $\mathcal{B}\subseteq \mathcal{A}$. Then $\bigcap_{\alpha\in\mathcal{A}}E_\alpha\neq\emptyset$.
	\end{thm}
	
	\begin{dfn}[K-cell]
		A K-cell is a subset of $\R^k$ of the form $[a_1,b_1]\times[a_2,b_2]\times\cdots\times[a_k,b_k]$.
	\end{dfn}
	
	\begin{thm}
		Every k-cell $F$ in $\mathbb{R}^k$ is compact.
	\end{thm}
	\begin{proof}
		TODO
	\end{proof}
	
	\begin{thm}
		A subset $E$ of $\mathbb{R}^k$ is compact if and only if it is closed and bounded.
	\end{thm}
	\begin{proof}
		TODO
	\end{proof}
	\begin{rem}
		The forward direction is true in any metric space.
	\end{rem}
	
	\begin{tcolorbox}[title=\textbf{Characterization of compact sets}]
		(1) and (2) are equivalent in any metric space. Forward direction of (3) is true in any metric space. All of three are equivalent in $\R^k$.
		\begin{enumerate}
			\item Every open cover of $E$ has a finite subcover.
			\item A set $E\subseteq X$ is compact if and only if every sequence in $E$ has a convergent subsequence converging to a point in $E$.
			\item A subset $E$ of $\mathbb{R}^k$ is compact if and only if it is closed and bounded.
		\end{enumerate}
	\end{tcolorbox}
	\lline
	
	\subsection*{Cantor Set}
	\begin{dfn}[Cantor Set]
		Let $\mathcal{C}_0$ be $[0,1]$. Then define $\mathcal{C}_1$ as the union of $2^1$ interval $[0,\frac{1}{3}]\cup[\frac{2}{3},1]$. Each time delete the middle $\frac{1}{3}$ of intervals. Thus $\mathcal{C}_2$ is the union of $2^2$ intervals which is $[0,\frac{1}{9}]\cup[\frac{2}{9},\frac{1}{3}]\cup[\frac{2}{3},\frac{7}{9}]\cup[\frac{8}{9},1]$.\smallskip
		
		In short, $C_n$ is the union of $2^n$ disjoint closed intervals of which length is $(\frac{1}{3})^n$. Then define Cantor Set
		\begin{displaymath}
			\mathcal{C}=\bigcap_{i=0}^{\infty}\mathcal{C}_i.
		\end{displaymath}
	\end{dfn}
	\begin{thm}
		Here are some facts/properties about the Cantor set $\mathcal{C}$:
		\begin{itemize}
			\item $\mathcal{C}$ is compact.
			\item $\mathcal{C}$ does not contain any intervals.
			\item $\mathcal{C}$ does not have any interior points.
			\item Every point in $\mathcal{C}$ is a limit point of $\mathcal{C}$.
			\item Every point in $\mathcal{C}$ is a limit point of $\mathcal{C}^\com$.
		\end{itemize}
	\end{thm}
	\newpage
	
	\section{Series}
	For an infinite series $\sum_{n=m}^{\infty}a_n$, we say it \emph{converge} provided the sequence $(s_n)$ of 
	partial sums
	\begin{displaymath}
		s_n=a_m+a_{m+1}+\cdots+a_n=\sum_{k=m}^{n}a_k
	\end{displaymath}
	also converges to a real number $S$. i.e.
	\begin{displaymath}
		\sum_{n=m}^{\infty}a_n=S\quad\text{means}\quad\lim s_n=S\quad\text{or}\quad\lim\limits_{n\rightarrow\infty}\left(\sum_{k=m}^{n}a_k\right)=S
	\end{displaymath}
	A series that does not converge is said to \emph{diverge}, so $\sum_{n=m}^{\infty}a_n$ \emph{diverge} to $+\infty$
	, $\sum_{n=m}^{\infty}a_n=+\infty$, provided $\lim s_n=+\infty$. Similar for diverging to $-\infty$.
	
	If the terms in $\sum a_n$ are all nonnegative, then the corresponding partial sums $(s_n)$ form an increasing sequence, so $\sum a_n$ either converges or diverges to $+\infty$ by \ref{def:bounded monotone seq} and \ref{def:unbounded monotone seq}. In particular, $\sum |a_n|$ is meaningful for any $(s_n)$ whatever. The series
	$\sum a_n$ is said to \emph{converge absolutely} or to be \emph{absolutely convergent} if $\sum |a_n|$ converges.
	
	We use $\sum a_n$ to represent $\sum_{n=m}^{\infty}a_n$
	
	\begin{eg}[Geometric Series]
		A series of the form $\sum_{n=0}^{\infty}ar^n$ for constants $a$ and $r$ is called a geometric series. For $r\neq1$,
		the partial sums $s_n$ are given by
		\begin{displaymath}
			\sum_{k=0}^{n}ar^k=a\frac{1-r^{n+1}}{1-r}.
		\end{displaymath}
		Furthermore, if $|r|<1$, then $\lim_{n\rightarrow \infty}r^{n+1}=0$ and
		\begin{displaymath}
			\sum_{n=0}^{\infty}ar^n=\frac{a}{1-r}
		\end{displaymath}
		If $a\neq0$ and $|r|\geq1$, then $(ar^n)$ does not converge to $0$, so $\sum ar^n$ diverges.
	\end{eg}
	
	\begin{eg}
		\begin{displaymath}
			\sum_{n=1}^{\infty}\frac{1}{n^p}\quad \text{converges if and only if}\quad p>1
		\end{displaymath}
		If $p\leq1$, $\sum1/n^p=+\infty$
	\end{eg}
	
	\setcounter{equation}{0}
	\begin{dfn}\label{def:cauchy criterion}
		We say a series $\sum a_n$ satisfies the \emph{Cauchy criterion} if its sequence $(s_n)$ of partial sums is a
		Cauchy sequence which is:
		\begin{equation}
			\forall\epsilon>0,\ \exists N,\ m,n>N\Rightarrow|s_n-s_m|<\epsilon
		\end{equation}
		which is equivalent to
		\begin{equation}
			\forall\epsilon>0,\ \exists N,\ n\geq m>N\Rightarrow|s_n-s_{m-1}|<\epsilon.
		\end{equation}
		Since $s_n-s_{m-1}=\sum_{k=m}^{n}a_k$, we can write (2) as
		\begin{equation}
			\forall\epsilon>0,\ \exists N,\ n\geq m>N\Rightarrow\left|\sum_{k=m}^{n}a_k\right|<\epsilon
		\end{equation}
	\end{dfn}
	
	\begin{thm}
		A series converges $\iff$ it satisfies the Cauchy criterion. 
	\end{thm}
	\begin{proof}
		By \ref{def:cauchy iff convergent}, we know its partial sum converges, so the series also converges.
	\end{proof}
	\begin{cor}\label{def:If a_n converges then lim(a_n)=0}
		If a series $\sum a_n$ converges, then $\lim a_n=0$
	\end{cor}
	\begin{proof}
		By setting $n=m$ in the condition of \ref{def:cauchy criterion}, we get
		\begin{displaymath}
			\left(\forall\epsilon>0,\ \exists N,\ n>N\Rightarrow\left|a_n\right|<\epsilon\right)\Rightarrow\lim a_n=0
		\end{displaymath}
	\end{proof}
	A useful contrapositive of this corollary is "If $\lim a_n\neq0$, then $\sum a_n$ does not converge."
	
	\setcounter{equation}{0}
	\begin{thm}[Comparison Test]\label{def:Comparison Test}
		Let $\sum a_n$ be a series where $a_n\geq0$ for all $n$.
		\begin{enumerate}[(i)]
			\item If $\sum a_n$ converges and $|b_n|\leq a_n$ for all $n$, then $\sum b_n$ converges.
			\item If $\sum a_n=+\infty$ and $b_n\geq a_n$ for all $n$, then $\sum b_n=+\infty$
		\end{enumerate}
	\end{thm}
	\begin{proof}
		\begin{enumerate}[(i)]
			\item[]
			\item For $n\geq m$ we have
			\begin{equation}
				\left|\sum_{k=m}^{n}b_k\right|\leq\sum_{k=m}^{n}|b_k|\leq\sum_{k=m}^{n}a_k
			\end{equation} 
			Since $\sum a_n$ converges, it satisfies \ref{def:cauchy criterion}(1). Then from (1)  we can see $\sum b_n$
			also satisfies the Cauchy criterion in \ref{def:cauchy criterion}(3), and hence $\sum b_n$ converges.
			\item Since $b_n\geq a_n$ for all $n$, obviously we have $\sum_{k=m}^{n}b_k\geq\sum_{k=m}^{n}a_k$. Since $\lim\sum_{k=m}^{n}b_k=+\infty$, $\lim\sum_{k=m}^{n}a_k=+\infty$.			
		\end{enumerate}
	\end{proof}
	\begin{cor}
		Absolutely convergent series are convergent.
	\end{cor}
	\begin{proof}
		Suppose $\sum b_n$ is absolutely convergent. This means $\sum a_n$ converges where $a_n=|b_n|$ for all $n$. Then
		$|b_n|\leq a_n$ and $\sum b_n$ converges trivially from \ref{def:Comparison Test}.
	\end{proof}
	
	\begin{thm}[Root Test]\label{def:Root Test}
		Let	$\sum a_n$ be a series and let $\alpha=\lim\sup|a_n|^{1/n}$. The series $\sum a_n$
		\begin{enumerate}[(i)]
			\item converges absolutely if $\alpha<1$
			\item diverges if $\alpha>1$
			\item Otherwise the test does not provide any useful information.
		\end{enumerate}
	\end{thm}
	\begin{proof}
		\begin{enumerate}[(i)]
			\item Suppose $\alpha<1$, and select $\epsilon>0$ so that $\alpha+\epsilon<1$. Then
			\begin{displaymath}
				\alpha-\epsilon<\sup\{|a_n|^{1/n}: n>N\}<\alpha+\epsilon
			\end{displaymath}
			so
			\begin{displaymath}
				|a_n|<(a+\epsilon)^n\quad\text{for}\quad n>N.
			\end{displaymath}
			Since $0<\alpha+\epsilon<1$, $\sum_{n=N+1}^{\infty}(\alpha+\epsilon)^n$ converges and \ref{def:Comparison Test}(i) tells $\sum_{n=N+1}^{\infty}a_n$ converges. Then clearly $\sum a_n$ converges.
			\item If $\alpha>1$, then there is a subsequence of $|a_n|^{1/n}$ has limit $\alpha>1$ by \ref{def:subsequence with limit limsup or liminf}. This means $|a_n|>1$ for infinitely many choices of $n$. In particular, $(a_n)$
			cannot possibly converge to $0$, so $\sum a_n$ cannot converge by the contrapositive of \ref{def:If a_n converges then lim(a_n)=0}.
			\item Example: $\sum\frac{1}{n}$ diverges but $\sum\frac{1}{n^2}$ converges.
		\end{enumerate}
	\end{proof}
	
	\begin{thm}[Ratio Test]\label{def:Ratio Test}
		A series $\sum a_n$ of nonzero terms
		\begin{enumerate}[(i)]
			\item converges absolutely if $\lim\sup\left|\frac{a_{n+1}}{a_n}\right|<1$,
			\item diverges if $\lim\inf\left|\frac{a_{n+1}}{a_n}\right|>1$.
			\item Otherwise $\lim\inf\left|\frac{a_{n+1}}{a_n}\right|\leq1\leq\lim\sup\left|\frac{a_{n+1}}{a_n}\right|$ and
			the test gives no information.
		\end{enumerate}
	\end{thm}
	\begin{proof}
		let $\alpha=\lim\sup|a_n|^{1/n}$. By \ref{def:12.2} we have
		\begin{displaymath}
			\lim\inf\left|\frac{a_{n+1}}{a_n}\right|\leq\alpha\leq\lim\sup\left|\frac{a_{n+1}}{a_n}\right|.
		\end{displaymath}
		\begin{enumerate}[(i)]
			\item If $\lim\sup\left|\frac{a_{n+1}}{a_n}\right|<1$, then $\alpha<1$ and the series converges by \ref{def:Root Test}.
			\item If $\lim\inf\left|\frac{a_{n+1}}{a_n}\right|>1$, then $\alpha>1$ and the series diverges by \ref{def:Root Test}.
			\item If $\alpha=1$, then same reasoning as the proof in \ref{def:Root Test}(iii).
		\end{enumerate}
	\end{proof}
	\bigskip
	If the terms $a^n$ are nonzero and if $\lim\left|\frac{a_{n+1}}{a_n}\right|=1$, then $\alpha=\lim\sup|a_n|^{1/n}=1$
	by \ref{def:12.3}, so neither the Ratio Test nor the Root Test gives information about the convergence of $\sum a_n$.
	\newpage

	\chapter{Useful Tricks}
	\begin{enumerate}
		\item Here is one of the most important techniques in real analysis.
		\begin{enumerate}[(a)]
			\item If $a\leq b+\epsilon$ for any $\epsilon>0$, then $a\leq b$.
			\item If $a\geq b-\epsilon$ for any $\epsilon>0$, then $a\geq b$.
			\item If $|a-b|<\epsilon$ for any $\epsilon>0$, then $|a-b|=0$.
		\end{enumerate}
		\item Let $S$ be a bounded nonempty subset of $\R$ and suppose $\sup S\notin S$. Then there is a (strictly) increasing sequence $(s_n)$ of points in $S$ such that $\lim s_n=\sup S$.
		\item A point $x$ is a limit point of a set $E\subseteq X$ if and only if $x=\lim x_n$ for some sequence $x_n$ of points in $E\backslash\{x\}$. 
		\item Let $(s_n)$ be a convergent sequence.
		\begin{itemize}
			\item If $s_n\geq a$ for all but finitely many $n$, then $\lim s_n\geq a$.
			\item If $s_n\leq b$ for all but finitely many $n$, then $\lim s_n\leq b$.
		\end{itemize}
		\item (Squeeze Theorem) If $a_n\leq s_n\leq b_n$ for all $n$ and $\lim a_n=\lim b_n=s$, then $\lim s_n=s$.
		\item Assume all $s_n\neq0$ and that the limit $L=\lim\left|\frac{s_{n+1}}{s_n}\right|$ exists.
		\begin{enumerate}[(a)]
			\item If $L<1$, then $\lim s_n=0$.
			\item If $L>1$, then $\lim |s_n|=+\infty$.
		\end{enumerate} 
		\item The set $\mathbb{Q}$ of rational number can be listed as a sequence $(r_n)$. Given any real number $a$ there exists a subsequence $(r_{n_k})$ of $(r_n)$ converging to $a$.
		\item Given two \textbf{convergent} sequences $(s_n)$ and $(t_n)$. If there exists $N\in\N$ such that $s_n\leq t_n$ for all $n\geq N$, then $\lim s_n\leq\lim t_n$.
		\item In general, if $A\subseteq B$, then $\inf A\geq \inf B$ and $\sup A\leq \sup B$.
	\end{enumerate}
	
\end{document}