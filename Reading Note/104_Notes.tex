\documentclass[12pt, lettersize]{book}

% format setting
\usepackage[margin=1in]{geometry}

\usepackage{amsmath}
\usepackage{amsthm}
\usepackage{amssymb}
\usepackage{physics}
\usepackage{enumerate}
\usepackage{hyperref}
\usepackage{chngcntr}
\usepackage{xcolor}
\usepackage{tcolorbox}


% Theorem declaration
\theoremstyle{plain}
\newtheorem{thm}{Theorem}[section]
\newtheorem{nte}[thm]{Notation}
\newtheorem{lem}[thm]{Lemma}
\newtheorem{cor}{Corollary}[thm]

\theoremstyle{definition}
\newtheorem{dfn}[thm]{Definition}
\newtheorem*{eg}{Example}

\theoremstyle{remark}
\newtheorem*{rem}{Remark}

\tcolorboxenvironment{thm}{
	colframe=cyan, colback=cyan!5, before skip=10pt,after skip=10pt}

\tcolorboxenvironment{dfn}{
	colframe=orange, colback=orange!5, before skip=10pt,after skip=10pt}

\tcolorboxenvironment{lem}{
	colframe=blue, colback=blue!5, before skip=10pt,after skip=10pt}

\tcolorboxenvironment{cor}{
	colframe=blue, colback=blue!5, before skip=10pt,after skip=10pt}

\tcolorboxenvironment{eg}{
	colframe=red, colback=red!5, before skip=10pt,after skip=10pt}

\renewcommand\qedsymbol{\hfill $\blacksquare$}
\newcommand{\R}{\mathbb{R}}
\newcommand{\N}{\mathbb{N}}
\newcommand{\Q}{\mathbb{Q}}
\newcommand{\Z}{\mathbb{Z}}
\newcommand{\dom}{\text{dom}\,}
\newcommand{\com}{\mathsf{C}}
\newcommand{\lline}{\noindent\rule{\textwidth}{1pt}}

\let\oldemptyset\emptyset
\let\emptyset\varnothing

\counterwithout{equation}{chapter}

\title{MATH 104 Notes}
\author{Wenhao Pan}
\date{\today}

\begin{document}
	\maketitle
	\tableofcontents
	
	\chapter{Introduction}
	\newpage
	\section{The Set $\N$ of Natural Numbers}
		We denote the set $\{1,2,3,\dots\}$ of all \emph{positive integers} by $\N$. Each positive integer $n$ has a successor, namely $n+1$. The following is 5 properties of $\N$:
		\begin{itemize}
			\item[\textbf{N1.}] $1$ belongs to $\N$.
			\item[\textbf{N2.}] If $n\in\N$, then its successor $n+1\in\N$.
			\item[\textbf{N3.}] $1$ is not the successor of any element in $\N$.
			\item[\textbf{N4.}] If $n$ and $m$ in $\N$ have the same successor, then $n=m$.
			\item[\textbf{N5.}] A subset of $\N$ which contains $1$, and which contains $n+1$ whenever it contains $n$, must equal $\N$.
		\end{itemize}
		Axiom \textbf{N5} is the basis of mathematical induction, which asserts all the statements $P_1,P_2,P_3,\dots$ are true provided
		\begin{itemize}
			\item[(\textbf{$I_1$})] $P_1$ is true,
			\item[(\textbf{$I_2$})] $P_{n+1}$ is true whenever $P_n$ is true.
		\end{itemize}
		\newpage
	\section{The Set $\Q$ of Rational Numbers}
		\begin{dfn}
		A number is called an \emph{algebraic number} if it satisfies a polynomial equation
		\begin{displaymath}
			c_nx^n+c_{n-1}x^{n-1}+\cdots+c_1x+c_0=0
		\end{displaymath}
		where the coefficients $c_0,c_1,\dots,c_n$ are integers, $c_n\neq0$ and $n\geq1$.
		\end{dfn}
		Rational numbers are always algebraic numbers. If $r=\frac{m}{n}$ is a rational number [$m,n\in\Z$ and $n\neq0$], then it satisfies the equation $nx-m=0$.
		
		\setcounter{equation}{0}
		\begin{thm}[Rational Zeros Theorem]\label{thm:2.2}
		Suppose $c_0,c_1,\dots,c_n$ are integers and $r$ is a rational number satisfying the polynomial equation
		\begin{equation}
			c_nx^n+c_{n-1}x^{n-1}+\cdots+c_1x+c_0=0
		\end{equation}
		where $n\geq1$, $c_n\neq0$ and $c_0\neq0$. Let $r=\frac{c}{d}$ where $c,d$ are integers having no common factors and $d\neq 0$. Then $c\,|\, c_0$ and $d\,|\,c_n$.
		\end{thm}
		In other words, the only rational candidates for solutions of (1) have the form $\frac{c}{d}$ where $c$ divides $c_0$ and $d$ divides $c_n$.
		\begin{proof}
		We are given
		\begin{displaymath}
			c_n\left(\frac{c}{d}\right)^n+c_{n-1}\left(\frac{c}{d}\right)^{n-1}+\cdots+c_1\left(\frac{c}{d}\right)+c_0=0
		\end{displaymath}
	 	Multiply both sides by $d^n$ and obtain
	 	\begin{displaymath}
	 		c_nc^n+c_{n-1}c^{n-1}d+c_{n-2}c^{n-2}d^2+\cdots+c_2c^2d^{n-2}+c_1cd^{n-1}+c_0d^n=0
	 	\end{displaymath}
 		Solve for $c_0d^n$ and obtain
 		\begin{displaymath}
 			c_0d^n=-c[c_nc^{n-1}+c_{n-1}c^{n-2}d+\cdots+c_2cd^{n-2}+c_1d^{n-1}]
 		\end{displaymath}
 		Since $c$ and $d^n$ have no common factors, $c$ divides $c_0$. Do the same thing and solve for $c_nc^n$ and we will see $d$ divides $c_n$.
		\end{proof}
		
		\begin{cor}\label{cor:2.3}
		Consider the polynomial equation
		\begin{displaymath}
			x^n+c_{n-1}x^{n-1}+\cdots+c_1x+c_0=0
		\end{displaymath}
		where the coefficients $c_0,c_1,\dots,c_{n-1}$ are integers and $c_0\neq0$. Any rational solution of this equation
		must be an integer that divides $c_0$.
		\end{cor}
		\begin{proof}
		By the Rational Zeros Theorem \ref{thm:2.2}, the denominator of $r$ must divide the coefficient of $x^n$, which is $1$. Thus $r$ is an integer dividing $c_0$.
		\end{proof}
		
		\newpage
	\section{The Set $\R$ of Real Numbers}
		The set $\Q$ of Rational numbers also have the following properties for addition and multiplication:
		\begin{itemize}
			\item[\textbf{A1.}] $a+(b+c)=(a+b)+c$ for all $a,b,c$.
			\item[\textbf{A2.}] $a+b=b+a$ for all $a,b$.
			\item[\textbf{A3.}] $a+0=a$ for all $a$.
			\item[\textbf{A4.}] For each $a$, there is an element $-a$ such that $a+(-a)=0$.
			\item[\textbf{M1.}] $a(bc)=(ab)c$ for all $a,b,c$.
			\item[\textbf{M2.}] $ab=ba$ for all $a,b$.
			\item[\textbf{M3.}] $a\cdot1=a$ for all $a$.
			\item[\textbf{M4.}] For each $a\neq0$, there is an element $a^{-1}$ such that $aa^{-1}=1$.
			\item[\textbf{DL}] $a(b+c)=ab+ac$ for all $a,b,c$.  
		\end{itemize}
		
		The set $\Q$ also has an order structure $\leq$ satisfying
		\begin{itemize}
			\item[\textbf{O1.}] Given $a$ and $b$, either $a\leq b$ or $b\leq a$.
			\item[\textbf{O2.}] If $a\leq b$ and $b\leq a$, then $a=b$.
			\item[\textbf{O3.}] If $a\leq b$ and $b\leq c$, then $a\leq c$.
			\item[\textbf{O4.}] If $a\leq b$, then $a+c\leq b+c$.
			\item[\textbf{O5.}] If $a\leq b$ and $0\leq c$, then $ac\leq bc$.
		\end{itemize}
		
		\begin{thm}\label{thm:3.1}
		The following are consequences of the field properties:
		\begin{enumerate}[(i)]
			\item \textcolor{red}{$a+c=b+c\implies a=b$};
			\item $a\cdot0=0$ for all $a$;
			\item $(-a)b=-ab$ for all $a,b$;
			\item $(-a)(-b)=ab$ for all $a,b$;
			\item $(ac=bc)\land(c\neq0) \implies a=b$;
			\item $ab=0\implies(a=0)\lor(b=0)$ for $a,b,c\in\R$.
		\end{enumerate}
		for $a,c,c\in\R$.
		\end{thm}
		
		\begin{thm}\label{thm:3.2}
		The following are consequences of the properties of an ordered field:
		\begin{enumerate}[(i)]
			\item $a\leq b\implies-b\leq-a$;
			\item $(a\leq b)\land(c\leq0)\implies bc\leq ac$;
			\item $(0\leq a)\land(0\leq b)\implies 0\leq ab$;
			\item $0\leq a^2$ for all $a$;
			\item $0<1$;
			\item $0<a\implies0< a^{-1}$;
			\item $0<a<b\implies0<b^{-1}<a^{-1}$;
		\end{enumerate}
		for $a,c,c\in\R$.
		\end{thm}
		Note that $a<b$ can be represented as $(a\leq b)\land(a<b)$.
		
		\begin{dfn}\label{def:3.3}
		We define
		\begin{displaymath}
			\text{$|a|=a$ if $a\geq0$  and   $|a|=-a$ if $a\leq0$}
		\end{displaymath}
		\end{dfn}
	
		An useful fact: $|a|\leq b\iff -b\leq a\leq b$.
		
		\begin{dfn}\label{def:3.4}
		For numbers $a$ and $b$ we define dist$(a,b)=|a-b|$; dist$(a,b)$ represents the \emph{distance between $a$ and $b$}.
		\end{dfn}
		
		\begin{thm}\label{thm:3.5}
		\begin{enumerate}[(i)]
			\item[]
			\item $|a|\geq0$ for all $a\in\R$.
			\item $|ab|=|a|\cdot|b|$ for all $a,b\in\R$.
			\item $|a+b|\leq|a|+|b|$ for all $a,b\in\R$. 
		\end{enumerate}
		\end{thm}
		
		\begin{cor}\label{cor:3.6}
		dist$(a,c)\leq$ dist$(a,b)+$ dist$(b,c)$ for all $a,b,c\in\R$. This is equivalent to $|a-c|\leq|b-c|+|b-c|$.
		\end{cor}
		
		\begin{thm}[Triangle Inequality]\label{thm:3.7}
		$|a+b|\leq|a|+|b|$ for all $a,b$.
		\end{thm}
		\begin{cor}[Reverse Triangular Inequality]\label{cor: reverse triangular}
		\textcolor{red}{$\big||a|-|b|\big|\leq|a-b|$ for all $a,b\in\R$.}	
		\end{cor}
	
		\begin{tcolorbox}
			\textcolor{red}{Here is one of the most important techniques in real analysis}.
			\begin{enumerate}[(a)]
				\item If $a\leq b+\epsilon$ for any $\epsilon>0$, then $a\leq b$.
				\item If $a\geq b-\epsilon$ for any $\epsilon>0$, then $a\geq b$.
				\item If $|a-b|<\epsilon$ for any $\epsilon>0$, then $|a-b|=0$.
			\end{enumerate}
			\tcblower
			\begin{rem}
				(i) is equivalent to "If $a\leq b_1$ for any $b_1>b$, then $a\leq b$". (ii) is similar.
			\end{rem}
			\tcblower
			\begin{proof}
				The proof for two cases is similar, so I will only show (a) here. Suppose that $a>b$. Let $\epsilon=(a-b)/2>0$. Then $a>b+\epsilon$, so the statement that $a\leq b+\epsilon$ for any $\epsilon>0$ is not true.
			\end{proof}
		\end{tcolorbox}
	
		\newpage
	\section{The Completeness Axiom}
		The completeness axiom for $\R$ ensure us $\R$ has no "gaps".
		\begin{dfn}\label{def:4.1}
		Let $S$ be a nonempty subset of $\R$.
		\begin{enumerate}[(a)]
			\item If $S$ contains a largest element $s_0$ [that is, $s_0\in S$ and $\forall s\in S,\ s\leq s_0$], then we call $s_0$ the \emph{maximum} of $S$ and write $s_0=\max S$.
			\item If $S$ contains a smallest element $s_0$ [that is, $s_0\in S$ and $\forall s\in S,\ s\geq s_0$], then we call $s_0$ the \emph{minimum} of $S$ and write $s_0=\min S$. 
		\end{enumerate}
		\end{dfn}
		Open intervals like $(a,b)=\{x\in\R: a<x\leq b\}$ have no minimum or maximum since the endpoints $a$ and $b$ is not in the interval.
		
		\begin{dfn}\label{def:4.2}
		Let $S$ be a nonempty subset of $\R$.
		\begin{enumerate}[(a)]
			\item If a real number $M$ satisfies $s\leq M$ for all $s\in S$, then $M$ is called an \emph{upper bound} of $S$ and the set $S$ is said to be \emph{bounded above}.
			\item If a real number $m$ satisfies $m\leq s$ for all $s\in S$, then $m$ is called an \emph{lower bound} of $S$ and the set $S$ is said to be \emph{bounded below}.
			\item The set $S$ is said to be \emph{bounded} if it is bounded above and bounded below. Thus $S$ is bounded if there exist real numbers $m$ and $M$ such that $S\subseteq[m,M]$.
		\end{enumerate}
		\end{dfn}
		The maximum of a set is always an upper bound for the set. Likewise, the minimum of a set is always a lower bound for the set.
	
		\begin{dfn}
		Least Upper Bound Property (LUBP)\newline
		An ordered set $S$ has the LUBP if every nonempty subset $\mathcal{A}\subset S$ that has an upper bound has a least upper bound in $S$.
		\end{dfn}
		Note that the set $\Q$ of rational number does not satisfy the LUBP but $\R$ does. e.g. $\mathcal(A)=\{q\in\Q: q^2<2\}$.
			
		\begin{dfn}\label{def:4.3}
		Let $S$ be a nonempty subset of $\R$.
		\begin{enumerate}[(a)]
			\item If $S$ is bounded above and $S$ has a least upper bound, then we will call it the \emph{supremum} of $S$ and denote it by $\sup S$.
			\item If $S$ is bounded below and $S$ has a greatest lower bound, then we will call it the \emph{infimum} of $S$ and denote it by $\inf S$.
		\end{enumerate}
		\end{dfn}
		If $S$ is bounded above, then $M=\sup S$ if and only if (i) $s\leq M$ for all $s\in S$, and (ii) whenever $M_1<M$, there exists $s_1\in S$ such that $s_1>M_1$. Or for each $\epsilon>0$, there exists $s\in S$ such that $s>\sup S-\epsilon$.
		
		Note that for a positive set $S=\{s: s>0\}$, its infimum is not always positive. Example: $\{\frac{1}{n}: n\in\N\}$. Each element is positive but the infimum is $0$.
		
		Here are some basic facts:
		\begin{itemize}
			\item If a set $S$ has finitely many elements, then $\max S$ exists.
			\item If $\max S$ exists, then $\sup S=\max S$.
			\item For any set $S\neq\emptyset$, $\inf S\leq \sup S$
		\end{itemize}
		
		\begin{thm}[Completeness Axiom]\label{thm:4.4}
		Every nonempty subset $S$ of $\R$ that is bounded above has a least upper bound. In other words, $\sup S$ exists and is a real number.
		\end{thm}
		Note that the completeness axiom does not hold for $\Q$. 
		\begin{cor}
		Every nonempty subset $S$ of $\R$ that is bounded below has a greatest lower bound. In other words, $\inf S$ exists and is a real number.
		\end{cor}
		
		\begin{thm}[Archimedean Property]\label{thm:4.6}
		If $a>0$ and $b>0$, then for some positive integer $n$, we have $na>b$.
		\end{thm}
	 	\begin{cor}
	 	(Set $a=1$). For any $b>0$, there exists $n\in\N$ such that $n>b$
	 	\end{cor}
 		\begin{cor}
 		(Set $b=1$). For any $a>0$, there exists $n\in\N$ such that $na>1\implies \frac{1}{n}<a$.
 		\end{cor}
		
		\begin{lem}\label{lem:4.7}
		If $x,y\in\R$ such that $y-x>1$, then there exists $m\in\Z$ such that $x<m<y$.
		\end{lem}
		\begin{proof}
		\begin{itemize}
			\item[]
			\item[Case 1:] $x\geq0$. Let $S=\{n\in\Z_+\: n\leq x\}$. By the corollary of Archimedean property \ref{thm:4.6} (set $a=1$), $S$ has finitely many elements, so $k=\max S$ exists. Then we have
			\begin{displaymath}
				x<k+1\leq x+1<y
			\end{displaymath}
			where $k+1$ is an integer.
			\item[Case 2:] $x<0$. Then $-x>0$. By the corollary of Archimedean property \ref{thm:4.6} (set $a=1$), there exists $N\in\N$ such that $N>-x$. Consider $x+N>0$ and $(y+N)-(x+N)>1$. By Case 1, there exists $m\in\Z$ such that $x+N<m<y+N$. Then $x<m-N<y$ where $m-N$ is an integer.
		\end{itemize}
		\end{proof}
		
		\begin{thm}[Denseness of $\Q$]\label{thm:4.7}
		If $a,b\in\R$ and $a<b$, then there is a rational $r\in\Q$ such that $a<r<b$.
		\end{thm}
		\begin{proof}
		By Archimean property \ref{thm:4.6}, there exists $n\in\N$ such that $n(b-a)>1$. i.e. $nb-na>1$. By \ref{lem:4.7}, there exists an integer $m\in\Z$ between $na$ and $nb$. Thus $na<m<nb\implies a<\frac{m}{n}<b$.
		\end{proof}
		\newpage
	\section{The Symbols $+\infty$ and $-\infty$}
		The symbols $+\infty$ and $-\infty$ are extremely useful even though they are \textbf{not} real numbers. So for each real number a, $-\infty<a<\infty$. If a set $S$ is not bounded above, we define $\sup S=+\infty$. Likewise, if $S$ is not bounded below, then we define $\inf S=-\infty$.
		
		We can extend real numbers to $\R\cup\{-\infty,\infty\}$. Notice that this is not a \textbf{field}, so it does not satisfy all field properties.
		
		For emphasis, we recapitulate:
		
		Let $S$ be any nonempty subset of $\R$. The \emph{symbols} $\sup S$ and $\inf S$ always make sense. If $S$ is not bounded above, then $\sup S$ is a \emph{real} number; otherwise $\sup S=+\infty$. If $S$ is bounded below, then $\inf S$ is a \emph{real} number; otherwise $\inf S=-\infty$. Moreover, we have $\inf S\leq\sup S$.
		
	\chapter{Sequences}
	\newpage
	\section{Limits of Sequences}
		A \emph{sequence} is a function whose domain is $\{n \in \mathbf{Z}: n \geq m,\ m\ \text{is usually}\ 1\ \text{or}\ 0\}$.
		We usually denote a sequence by $s$ and its value at $n$ by $s_n$. $(s_n)^\infty_{n=m}=(s_m,s_{m+1},\dots)$. $(s_n)_{n\in N}$ represents the sequence with $m=1$.
		\begin{eg}
		\begin{itemize}
			\item[]
			\item $(s_n)_{n\in N}$ where $s_n=\frac{1}{n^2}$ is the sequence $(1,\frac{1}{4},\frac{1}{9},\dots)$
			\item $(a_n)^\infty_{n=0}$ where $a_n=(-1)^n$ is the sequence $(1,-1,1,-1,1,\dots)$ 
		\end{itemize}
		\end{eg}
		The "limits" of a sequence is a real number that the values $s_n$ are close to for large values of $n$.
		
		\begin{dfn}\label{def:limit}
		A sequence $(s_n)$ of real numbers is said to \textbf{converge} to the real number \emph{$s$} provided that
		\begin{displaymath}
			\forall \epsilon > 0,\ \exists N,\ n > N \Rightarrow |s_n-s| < \epsilon.
		\end{displaymath}
		If $(s_n)$ converges to $s$, we write $\lim_{n\rightarrow \infty}s_n=s$ or $s_n\rightarrow s$. $s$ is the \emph{limit} of the sequence $(s_n)$.
		A sequence that does not converge (i.e. it has no \emph{limit}) is said to \emph{diverge}.\\
		Notice that in the definition, instead of simple $\epsilon$, we can also use some other complicated forms with some extra constants like $M\epsilon,\ \frac{\epsilon}{c},\ a^2\epsilon$ and so on.
		\end{dfn}
	
		Intuitively, the definition means that no matter how small you pick $\epsilon>0$, \textbf{eventually} the sequence will stay within $\epsilon$ of $s$ at some point (the threshold $N$) and forever after.
	
		\begin{thm}
		The limit of a sequence $(s_n)$ is unique. i.e. $(\lim s_n=s) \land (lim s_n=t) \Rightarrow s=t$.
		\end{thm}
		\begin{proof}
		By the definition of limit, we have
		\begin{align*}
			n > N_1 &\Rightarrow |s_n-s| < \frac{\epsilon}{2}\\
			n > N_2 &\Rightarrow |s_n-t| < \frac{\epsilon}{2}
		\end{align*}
		For $n>\max\{N_1,N_2\}$, by Triangular Inequality \ref{def:tri-ineq},
		\begin{displaymath}
			|s-t|=|(s-s_n)+(s_n-t)|\leq |s-s_n|+|s_n-t|\leq \frac{\epsilon}{2}+\frac{\epsilon}{2}=\epsilon
		\end{displaymath}
		This shows $\forall \epsilon>0,\ |s-t|<\epsilon\Rightarrow |s-t|=0\Rightarrow s=t$
		\end{proof}
		\newpage
		
		\begin{thm}
		Let $(s_n)$ be a convergent sequence.\textcolor{red}{
		\begin{itemize}
			\item If $s_n\geq a$ for all but finitely many $n$, then $\lim s_n\geq a$.
			\item If $s_n\leq b$ for all but finitely many $n$, then $\lim s_n\leq b$.
		\end{itemize}}
		\end{thm}
		\begin{rem}
			Notice that $a_n$ and $b_n$ need to converge so that the theorem can work.
		\end{rem}
		
		\begin{thm}[Squeeze Lemma]\label{lem: squeeze}
			\textcolor{red}{If $a_n\leq s_n\leq b_n$ for all $n$ and $\lim a_n=\lim b_n=s$, then $\lim s_n=s$.}
		\end{thm}
		\begin{rem}
			Notice that $a_n$ and $b_n$ need to converge so that the theorem can work.
		\end{rem}
	
	\section{A Discussion about Proofs}
		This section gives several examples of proofs with some discussion using the definition of the limit of a sequence.
		\begin{eg}
		Prove $\lim \frac{1}{n^2}=0$.
		\end{eg}
		\emph{Discussion}. According to the definition of the limit, we need to consider an $\epsilon >0$ such that $|\frac{1}{n^2}-0|<\epsilon$ for $n>\text{some}N$.
		$|\frac{1}{n^2}-0|<\epsilon$ implies that $\frac{1}{\epsilon}<n^2 \text{or} \frac{1}{\sqrt{\epsilon}}<n$.
		Thus we can suppose $N=\frac{1}{\sqrt{\epsilon}}$ and check if we reverse our reasoning into proof, it still makes sense. 
		\begin{proof}
		Let $\epsilon>0$ and $N=\frac{1}{\sqrt{\epsilon}}$, then
		\begin{align*}
			n>N &\Rightarrow \epsilon > \frac{1}{n^2}\\
				&\Rightarrow \frac{1}{n^2}-0 < \epsilon -0\\
				&\Rightarrow \abs{\frac{1}{n^2}-0}<\epsilon
		\end{align*}
		This proofs $\lim \frac{1}{n^2}=0$ according to the definition of the limit \ref{def:limit}.
		\end{proof}
		
		\begin{eg}
		Prove $\lim \frac{3n+1}{7n-4}=\frac{3}{7}$
		\end{eg}
		\emph{Discussion}. Just like the last example, we can start from the definition \ref{def:limit} to get a suitable $N$.
		\begin{proof}
			Let $\epsilon>0$ and $N=\frac{19}{49\epsilon}+\frac{4}{7}$, then
		\begin{align*}
			n>N &\Rightarrow 7n>\frac{19}{7\epsilon}+4\\
				&\Rightarrow \frac{19}{7(7n-4)}<\epsilon\\
				&\Rightarrow \frac{3n+1}{7n-4} - \frac{3}{7}<\epsilon\\
				&\Rightarrow \abs{\frac{3n+1}{7n-4} - \frac{3}{7}}<\epsilon\qquad \text{since}\ n>0
		\end{align*}
		This proofs $\lim \frac{3n+1}{7n-4}=\frac{3}{7}$ according to the definition of the limit \ref{def:limit}.
		\end{proof}
		
		\begin{eg}
			Prove $\lim\frac{4n^3+3n}{n^3-6}=4$
		\end{eg}
		\emph{Discussion}. Since $\frac{4n^3+3n}{n^3-6}-4 = \frac{3n+24}{n^3-6}$, when $n>1$, we can find an upper bound for
		$\frac{3n+24}{n^3-6}$ so that the bound $<\epsilon \Rightarrow \abs{\frac{3n+24}{n^3-6}}<\epsilon$. Finding an upper bound for a fraction is equivalent to finding a upper bound for its numerator and a lower bound for its denominator.
		We know $3n+24\leq27n$ for $n>1$. Also we note $n^3-6\geq\frac{n^3}{2} \Rightarrow n>2$. Thus we can have $\frac{3n+24}{n^3-6}<\frac{27n}{n^3/2}<\epsilon \Rightarrow n>\sqrt{\frac{54}{\epsilon}},\ \text{provided}\ n>2$.
		\begin{proof}
			Let $\epsilon>0$ and $N=\max\{2,\sqrt{\frac{54}{\epsilon}}\}$, then
		\begin{align*}
			n>N &\Rightarrow (n>\sqrt{\frac{54}{\epsilon}})\land(n>2)\\
			&\Rightarrow (\frac{27n}{n^3/2}<\epsilon)\land(\frac{n^3}{2}\leq n^3-6)\land(27n\geq3n+24)\\
			&\Rightarrow \frac{3n+24}{n^3-6} < \frac{27n}{n^3/2} < \epsilon\\
			&\Rightarrow \abs{\frac{4n^3+3n}{n^3-6}-4} < \epsilon
		\end{align*}
		This proofs $\lim\frac{4n^3+3n}{n^3-6}=4$ according to the definition of the limit \ref{def:limit}.
		\end{proof}
		
		\begin{eg}
		Show that $a_n=(-1)^n$ does not converge.
		\end{eg}
		\emph{Discussion}. Assume $\lim(-1)^n=a$, and we can see that no matter what $a$ is, either $1$ or $-1$ is at least
		$1$ from $a$, so it means $\abs{(-1)^n-a}<1$ will not hold for all large $n$.
		\begin{proof}
		Suppose $\lim(-1)^n=a$ and $\epsilon = 1$. By \ref{def:limit}, $\abs{(-1)^n-a}<1 \Rightarrow (|1-a|<1)\land(|-1-a|<1)$.
		Now by \ref{def:tri-ineq}, $2=|1-a+a-(-1)|\leq|1-a|+|a-(-1)|<1+1=2$, which is a contradiction. 
		\end{proof}
		
		\begin{eg}
		Let $(s_n)$ be a sequence of nonnegative real numbers and suppose $s=\lim s_n$. Note $s\geq0$. Prove $\lim\sqrt{s_n}=\sqrt{s}$
		\end{eg}
		\begin{proof}
		There are two cases.
		\begin{enumerate}
			\item $s>0$: Let $\epsilon>0$. $\lim s_n=s \Rightarrow (\exists N,\ n>N \Rightarrow |s_n-s|<\sqrt{s}\epsilon)$.
			$n>N$ also implies
			\begin{displaymath}
				|\sqrt{s_n}-\sqrt{s}|=\frac{(\sqrt{s_n}-\sqrt{s})(\sqrt{s_n}+\sqrt{s})}{\sqrt{s_n}+\sqrt{s}}=\frac{|s_n-s|}{\sqrt{s_n}+\sqrt{s}}\leq\frac{|s_n-s|}{\sqrt{s}}<\frac{\sqrt{s}\epsilon}{\sqrt{s}}=\epsilon
			\end{displaymath}
			\item $s=0$: EXERCISE 8.3 
		\end{enumerate}
		\end{proof}
	
		\begin{eg}
		Let $(s_n)$ be a convergent sequence of real numbers such that $s_n\neq0$ for all $n\in \mathbb{N}$ and $\lim s_n=s
		\neq0$. Prove $\inf\{|s_n|: n\in\mathbb{N}\}>0$
		\end{eg}
		\begin{proof}
		Let $\epsilon=\frac{|s|}{2}$. Since $\lim s_n=s$,
		\begin{displaymath}
			n>N \Rightarrow |s_n-s|<\frac{|s|}{2} \Rightarrow |s_n|\geq\frac{|s|}{2}
		\end{displaymath}
		The last implication is because otherwise 
		\begin{displaymath}
			|s|=|s-s_n+s_n|\leq|s-s_n|+|s_n|<\frac{|s|}{2}+\frac{|s|}{2}=|s|
		\end{displaymath}
		which is a contradiction. Now if we set $m=\min\{\frac{|s|}{2},|s_1|,|s_2|,\dots,|s_N|\}$, then clearly we have
		$m>0$ since and $|s_n|\geq m$ for all $n\in \mathbb{N}$. Thus $\inf\{|s_n|: n\in\mathbb{N}\}\geq m>0$\textbf{WHY???} 
		\end{proof}
		\newpage
	\section{Limit Theorems for Sequences}
		\begin{dfn}\label{def:bound}
		A sequence $(s_n)$ is said to be \emph{bounded} if $\exists M,\ \forall n,\ \text{such that}\ |s_n|\leq M$
		\end{dfn}
		
		\begin{thm}\label{def:convergence is bounded}
		Convergent sequences are bounded.
		\end{thm}
		\begin{proof}
		Let $(s_n)$ be a convergent sequence and $\lim s_n=s$, then select $\epsilon=1$ and we have
		\begin{displaymath}
			n>N \Rightarrow |s_n-s|<1
		\end{displaymath}
		From the reverse triangular inequality \ref{cor: reverse triangular}, $|s_n|-|s|\leq|s_n-s|<1 \Rightarrow |s_n|<|s|+1$ when $n>N$. Thus define $M=\max\{|s|+1,|s_1|,|s_2|,\dots,|s_N|\}$, then $|s_n|<M\ \text{for all $n$}$. 	
		\end{proof}
		\begin{rem}
			In other words, unbounded sequences are not convergent.
		\end{rem}
		
		\begin{thm}
		If the sequence $(s_n)$ converges to $s$ and $k\in\mathbb{R}$, then $(ks_n)$ converges to $ks$. i.e. $\lim(ks_n)=k\cdot\lim s_n$.
		\end{thm}
		\begin{proof}
		Assume $k\neq0$ and let $\frac{\epsilon}{|k|}$, then there exists $N$ such that
		\begin{displaymath}
			n>N \Rightarrow |s_n-s|<\frac{\epsilon}{|k|} \Rightarrow |ks_n-ks|<\epsilon
		\end{displaymath}
		\end{proof}
		
		\begin{thm}\label{def:addition}
		If $(s_n)$ and $(t_n)$ converge to $s$ and $t$, then $(s_n+t_n)$ converges to $s+t$. That is,
		\begin{displaymath}
			\lim(s_n+t_n)=\lim s_n+\lim t_n.
		\end{displaymath} 
		\end{thm}
		\begin{proof}
		From \ref{def:limit}, we know
		\begin{align*}
			n>N_1&\Rightarrow|s_n-s|<\frac{\epsilon}{2}\\
			n>N_2&\Rightarrow|t_n-t|<\frac{\epsilon}{2}
		\end{align*}
		Thus, let $N=\max\{N_1,N_2\}$,
		\begin{displaymath}
			n>N \Rightarrow |s_n+t_n-(s+t)|\leq|s_n-s|+|t_n-t|<\epsilon
		\end{displaymath}
		\end{proof}
	
		\begin{thm}\label{def:multiplication}
		If $(s_n)$ and $(t_n)$ converge to $s$ and $t$, then $(s_nt_n)$ converges to $st$. That is,
		\begin{displaymath}
			\lim(s_nt_n)=(\lim s_n)(\lim t_n)
		\end{displaymath} 
		\end{thm}
		\begin{proof}
		Let $\epsilon>0$. By \ref{def:convergence is bounded}, $|s_n|\leq M$ for some $M>0$. From \ref{def:limit}, we have
		\begin{align*}
			n>N_1&\Rightarrow|s_n-s|<\frac{\epsilon}{2(|t|+1)}\\
			n>N_2&\Rightarrow|t_n-t|<\frac{\epsilon}{2M}
		\end{align*}
		Thus, let $N=\max\{N_1,N_2\}$,
		\begin{align*}
			n>N \Rightarrow |s_nt_n-st|&=|s_nt_n-s_nt+s_nt-st|\\
									   &\leq|s_nt_n-s_nt|+|s_nt-st|\\
									   &=|s_n|\cdot|t_n-t|+|t|\cdot|s_n-s|\\
									   &\leq M\cdot\frac{\epsilon}{2M}+|t|\cdot\frac{\epsilon}{2(|t|+1)}\\
									   &=\epsilon
		\end{align*}
		\end{proof}
		
		\begin{lem}
		If $(s_n)\rightarrow s\neq0$ and $s_n\neq0$ and  for all $n$, then $\inf\{|s_n|: n\in\N\}>0$.
		\end{lem}
		\begin{proof}
		Since $(s_n)\rightarrow s$, select $\epsilon=\frac{|s|}{2}$ and we have $n\geq N\implies|s_n-s|<\frac{|s|}{2}$, which implies $|s_n|>\frac{|s|}{2}$. Thus select $m=\min\{s_1,\dots,s_N,\frac{|s|}{2}\}$, and then $|s_n|\geq m$ for all $n$. Since $m>0$ and $m$ is a lower bound of $(|s_n|)$, $\inf\{|s_n|: n\in\N\}\geq m>0$.
		\end{proof}
		
		\begin{lem}\label{def:reciprocal}
		If $(s_n)$ converges to $s$, $s_n\neq 0$ for all $n$, and $s\neq 0$, then $(1/s_n)$ converges to $1/s$.
		\end{lem}
		\begin{proof}
		Let $\epsilon>0$. Since there exists $m>0$ such that $|s_n|\geq m$ for all $n$. By \ref{def:limit}, we have
		\begin{align*}
			n>N &\Rightarrow |s-s_n|<\epsilon\cdot m|s|\\
				&\Rightarrow \abs{\frac{1}{s_n}-\frac{1}{s}}=\frac{|s-s_n|}{|s_ns|}\leq\frac{|s-s_n|}{m|s|}<\epsilon.
		\end{align*} 
		\end{proof}
	
		\begin{thm}
		Suppose $(s_n)$ and $(t_n)$ converge to $s$ and $t$. If $s\neq 0$ and $s_n\neq 0$ for all $n$, then $(t_n/s_n)$ converges to t/s.
		\end{thm}
		\begin{proof}
		By \ref{def:reciprocal}, $(1/s_n)$ converges to $1/s$, so
		\begin{displaymath}
			\lim\frac{t_n}{s_n}=\lim\frac{1}{s_n}\cdot\lim t_n=\frac{1}{s}\cdot t=\frac{t}{s}
		\end{displaymath}
		by \ref{def:multiplication}.
		\end{proof}
		
		\begin{thm}
		\begin{enumerate}[(a)]
			\item[]
			\item $\lim_{n\rightarrow\infty}(\frac{1}{n^p})=0$ for $p>0$.
			\item $\lim_{n\rightarrow\infty}a^n=0$ if $|a|<1$.
			\item $\lim(n^{1/n})=1$.
			\item $\lim_{n\rightarrow\infty}a^{1/n}=1$ for $a>0$.
		\end{enumerate}
		\end{thm}
		\begin{proof}
		\begin{enumerate}[(a)]
			\item[]
			\item Let $N=(\frac{1}{\epsilon})^{1/p}$ and the rest is easy.
			\item If $a=0$ then it's obvious. Otherwise, since $|a|<1$ we can write $|a|=\frac{1}{1+b}$ where $b>0$.
			Since $(1+b)^n\geq 1+nb>nb$,
			\begin{displaymath}
				|a^n-0|=|a^n|=\frac{1}{(1+b)^n}<\frac{1}{nb}
			\end{displaymath}.
		 	Then let $N=\frac{1}{\epsilon b}$ and finish the proof.
			\item Let $s_n=(n^{1/n})-1$ and note $s_n\geq0$ for all $n$. By \ref{def:addition}, we only need to show
			$\lim s_n=0$. $1+s_n=(n^{1/n})\Rightarrow n=(1+s_n)^n$. For $n\geq2$, the binomial expansion tells
			\begin{displaymath}
				n=(1+s_n)^n\geq1+ns_n+\frac{1}{2}n(n-1)s_n^2>\frac{1}{2}n(n-1)s_n^2.
			\end{displaymath}
			This implies $s_n<\sqrt{\frac{2}{n-1}}$ for $n\geq2$. Now we can suppose $N=\frac{\epsilon}{\epsilon-2}$ to finish the proof.
			\item If $a\geq1$, then for $n\geq a$ we have $1\leq a^{1/n}\leq n^{1/n}$.
			Since $\lim n^{1/n}=1$, by Squeeze Theorem we have $\lim a^{1/n}=1$. Now if $0<a<1$, then $\frac{1}{a}>1$, so
			$\lim (\frac{1}{a})^{1/n}=1$ from above. By \ref{def:reciprocal}, $\lim a^{1/n}=1$.
		\end{enumerate}
		\end{proof}
		
		\begin{dfn}
		For a $(s_n)$, we write $\lim s_n=+\infty$ provided for each $M>0$ there is a number $N$ wuch that $n>N\Rightarrow s_n>M$. Similarly, we write $\lim s_n=-\infty$ provided for each $M<0$ there is a number $N$ wuch that $n>N\Rightarrow s_n<M$.
		\end{dfn}
		This implies that if $\lim s_n>-\infty$, $\exists T,\ \forall n, s_n>T$. $\lim s_n<\infty$, $\exists T,\ \forall n, s_n<T$. 
		Be careful that we say $\lim s_n=+\infty$ as $(s_n)$ \textbf{diverges} to $\infty$, \textbf{not converge} to $\infty$.
		\begin{eg} 
		Prove that $\lim(\sqrt{n}+7)=+\infty$.
		\end{eg}
		\begin{proof}
		Let $M>0$ and let $N=(M-7)^2$. Then $n>N\Rightarrow \sqrt{n}+7>M$.
		\end{proof}
		
		\begin{eg}
		Prove $\lim\frac{n^2+3}{n+1}=+\infty$
		\end{eg}
		\emph{Discussion}. We want to find a simpler lower bound for $\frac{n^2+3}{n+1}=+\infty$.
		\begin{proof}
		Let N=2M. Then
		\begin{displaymath}
			\frac{n^2+3}{n+1}>\frac{n^2}{2n}=\frac{1}{2}n>M.
		\end{displaymath}
		\end{proof}
		
		\begin{thm}
		Let $\lim s_n=+\infty$ and $\lim t_n>0$. Then $\lim s_nt_n=+\infty$.
		\end{thm}
		\begin{proof}
		Let $M>0$ and select an $m$ so that $0<m<\lim t_n$. It is clear that there exists $N_1$ so that
		\begin{displaymath}
			n>N_1 \Rightarrow t_n>m
		\end{displaymath}
		Since $\lim s_n=+\infty$, there exists $N_2$ so that 
		\begin{displaymath}
			n>N_2 \Rightarrow s_n>\frac{M}{m}
		\end{displaymath}
		Thus $n>\max\{N_1,N_2\} \Rightarrow s_nt_n>\frac{M}{m}\cdot m=M$.
		\end{proof}
		
		\begin{thm}
		For a $(s_n)$ of \emph{positive} real numbers, we have $\lim s_n=+\infty$ if and only if $\lim(\frac{1}{s_n})=0$.
		\end{thm}
		\begin{proof}
		We need to show it in both directions.
		\begin{itemize}
			\item[$\Rightarrow$]: Let $\epsilon>0$ and $M=\frac{1}{\epsilon}$. Since $\lim s_n=+\infty$, $n>N\Rightarrow s_n>M=\frac{1}{\epsilon}$.
			Therefore, $n>N\Rightarrow\abs{\frac{1}{s_n}-0}<\epsilon$. 
			\item[$\Leftarrow$]: Let $M>0$ and $\epsilon=\frac{1}{M}$, then $n>N\Rightarrow\abs{\frac{1}{s_n}-0}<\epsilon=\frac{1}{M}$. Since $s_n>0$, we have
			\begin{displaymath}
				n>N\Rightarrow 0<\frac{1}{s_n}<\frac{1}{M}\Rightarrow s_n>M.
			\end{displaymath} 
		\end{itemize}
		\end{proof}
		
		\begin{thm}
		Assume all $s_n\neq0$ and that the limit $L=\lim\left|\frac{s_{n+1}}{s_n}\right|$ exists.
		\begin{enumerate}[(a)]
			\item If $L<1$, then $\lim s_n=0$.
			\item If $L>1$, then $\lim |s_n|=+\infty$.
		\end{enumerate} 
		\end{thm}
		\begin{proof}
		See exercise 9.12 and HW2 Q8.
		\end{proof}
		
		\begin{thm}
		Given two \textbf{convergent} sequences $(s_n)$ and $t_n$. If there exists $N\in\N$ such that $s_n\leq t_n$ for all $n\geq N_0$, then $\lim s_n\leq\lim t_n$.
		\end{thm}
		\begin{proof}
			See textbook exercise 9.9(c)
		\end{proof}
		\newpage
	\section{Monotone Sequences and Cauchy Sequence}
		\begin{dfn}
		$(s_n)$ is called an \emph{increasing sequence (or nondecreasing)} if $\forall n,\ s_n\leq s_{n+1}$ and $s_n\leq s_m\ \text{whenever}\ n<m$.
		Similarly, $(s_n)$ is called an \emph{decreasing sequence (or nonincreasing)} if $\forall n,\ s_n\geq s_{n+1}$. An increasing or decreasing sequence is called \emph{monotone} or \emph{monotonic} sequence.
		\end{dfn}
	
		\begin{thm}\label{def:bounded monotone seq}
		All bounded monotone sequences converge.
		\end{thm}
		\begin{proof}
		Let $(s_n)$ be a bounded increasing sequence, $S=\{s_n: n\in \mathbb{N}\}$. We can say $u=\sup S$ since $(s_n)$ is bounded by \ref{thm:4.4}. 
		Since $u-\epsilon<u$, there exists $N$ such that $s_N>u-\epsilon \Rightarrow \forall n>N,\ s_n>u-\epsilon$. Since $u$ is the supremum, $u-\epsilon<s_n\leq u \Rightarrow |s_n-u|<\epsilon$.
		The proof for decreasing sequence is in exercise 10.2.
		\end{proof}
		\begin{rem}
		From the proof procedure above, we can see that bounded monotone sequences \textbf{converge to its infimum or supremum}.
		\end{rem}
		
		\subsection*{Discussion of Decimals}
		Notice that real numbers are simply decimal expansions. For a decimal expansion like $K.d_1d_2d_3d_4\cdots$, we can define a sequence by
		\begin{displaymath}
			s_n=K+\frac{d_1}{10}+\frac{d_2}{10^2}+\cdots+\frac{d_n}{10^n}
		\end{displaymath} 
		where $K$ is an nonnegative integer and each $d_j\in\{0,1,2,3,4,5,6,7,8,9\}$. When trying to $\lim s_n$, the formula
		of geometric series could help:
		\begin{displaymath}
			\lim\limits_{n\rightarrow\infty}a(1+r+r^2+\cdots+r^n)=\frac{a}{1-r}\quad \text{for}\quad |r|<1;
		\end{displaymath}
		
		There are two important reversible facts:
		\begin{enumerate}
			\item Different decimal expansions can represent the same real number.
			\item Every nonnegative real number has at least one decimal expansion
		\end{enumerate}\bigskip

		\begin{thm}\label{def:unbounded monotone seq}
		\begin{enumerate}[(i)]
			\item[]
			\item If $(s_n)$ is an unbounded increasing sequence, then $\lim s_n=+\infty$.
			\item If $(s_n)$ is an unbounded decreasing sequence, then $\lim s_n=-\infty$.
		\end{enumerate}
		\end{thm}
		\begin{proof}
		\begin{enumerate}[(i)]
			\item[]
			\item Let $M>0$. Since $\{s_n: n\in\mathbb{N}\}$ is unbounded and bounded by $s_1$, it must be unbounded above.
			Thus there must be some $N\in\mathbb{N}$ so that $s_N>M$. Since $s_n$ is increasing, we have $n>N\Rightarrow s_n\geq s_N>M$, so $\lim s_n=+\infty$. 
			\item Exercise 10.5
		\end{enumerate}
		\end{proof}
		\begin{thm}
		\begin{itemize}
			\item[]
			\item If $(s_n)$ is a bounded and nonincreasing sequence, then $\lim s_n=\inf\{s_n: n\in\N\}$.
			\item If $(s_n)$ is a bounded and nondecreasing sequence, then $\lim s_n=\sup\{s_n: n\in\N\}$.
		\end{itemize}
		\end{thm}
		\begin{cor}
		If $(s_n)$ is monotone, then $\lim s_n$ is always meaningful. i.e. $\lim s_n=s,\ +\infty,\ \text{or}\ -\infty$.
		\end{cor}\bigskip
	
		Suppose $(s_n)$ is bounded. Define $u_n=\inf\{s_m: m\geq n\}$ and $v_n=\sup{s_m: m\geq n}$. Then observe that $(u_n)$ is nondecreasing and $(v_n)$ is nonincreasing since as $n$ increases, the set has fewer elements. i.e. we have fewer choices for infimum and supremum. In general, if $A\subseteq B$, then $\inf A\geq \inf B$ and $\sup A\leq \sup B$.
		
		\begin{thm}
		Let $S$ be a bounded nonempty subset of $\R$ and suppose $\sup S\notin S$. Then there is a (strictly) increasing sequence $(s_n)$ of points in $S$ such that $\lim s_n=\sup S$.
		\end{thm}
		\begin{proof}
			See Homework 3.10
		\end{proof}
		
		\begin{dfn}
		Let $(s_n)$ be a sequence in $\mathbb{R}$, define
		\begin{itemize}
			\item $\lim\sup s_n=\lim\limits_{N\rightarrow\infty}\sup\{s_n: n>N\}$
			\item $\lim\inf s_n=\lim\limits_{N\rightarrow\infty}\inf\{s_n: n>N\}$
		\end{itemize}
		\end{dfn}
		If $(s_n)$ is not bounded above. $\sup\{s_n: n>N\}=+\infty$ for all $N$ and we decree $\lim\sup s_n=+\infty$.
		Likewise, if $(s_n)$ is not bounded below. $\inf\{s_n: n>N\}=-\infty$ for all $N$ and we decree $\lim\inf s_n=-\infty$.
		
		Notice that $\lim\sup s_n$ need not equal to $\sup\{s_n: n>N\}$, but $\lim\sup s_n\leq\sup\{s_n: n>N\}$
		\begin{rem}
			Since $v_n$ and $u_n$ are monotone, $\lim v_n=\lim\sup s_n$ and $\lim u_n=\lim\inf s_n$ always exist.
		\end{rem}
		
		\begin{thm}\label{def:condition for limit}
		Let $(s_n)$ be a sequence in $\mathbb{R}$.
		\begin{enumerate}[(i)]
			\item If $\lim s_n$ is defined, then $\lim\inf s_n=\lim s_n=\lim\sup s_n$.
			\item If $\lim\inf s_n=\lim\sup s_n$, then $\lim s_n$ is defined and $\lim s_n=\lim\inf s_n=\lim\sup s_n$.
		\end{enumerate}
		\end{thm}
		\begin{proof}
		We use the notation $u_N=\inf\{s_n: n>N\},\ v_N=\sup \{s_n: n>N\},\ u=\lim u_N=\lim\inf s_n\ \text{and}\ v=\lim v_N=\lim\sup s_n$.
		\begin{enumerate}[(i)]
			\item \begin{itemize}
					  \item $\lim s_n=+\infty$: Let $M>0$ then there is a $N$ so that $n>N\Rightarrow s_n>M$. Then
					  $u_N=\inf\{s_n: n>N\}\geq M$. This means $m>N\Rightarrow u_m\geq M\Rightarrow\lim u_N=\lim\inf s_n=+\infty$.
					  \item $\lim s_n=-\infty$: It is similar to the previous proof.
					  \item $\lim s_n=s$: Let $\epsilon>0$ then $n>N\Rightarrow |s_n-s|<\epsilon$, so
					  \begin{displaymath}
					  	v_N=\sup\{s_n: n>N\}\leq s+\epsilon.
					  \end{displaymath}
				  	  Also $m>N\Rightarrow v_m\leq s+\epsilon$ since $v_n$ is nonincreasing, so 
				  	  \begin{displaymath}
				  	  	\lim\sup s_n=\lim v_m\leq s+\epsilon\Rightarrow\lim\sup s_n\leq s=\lim s_n.
				  	  \end{displaymath}
			  	  	  A similar argument shows $\lim s_n\leq\lim\inf s_n$. Since $\lim\inf s_n\leq\lim\sup s_n$ we get
			  	  	  \begin{displaymath}
			  	  	  	\lim\inf s_n=\lim s_n=\lim\sup s_n
			  	  	  \end{displaymath}
				  \end{itemize}
			\item \begin{itemize}
					  \item If $\lim\inf s_n=\lim\sup s_n=s$, then we have
					  \begin{displaymath}
					  	\abs{s-\sup{s_n:n>N_0}}<\epsilon
					  \end{displaymath}
				  	  which implies $\sup\{s_n: n>N_0\}<s+\epsilon\Rightarrow\forall n>N_0,\ s_n<s+\epsilon$.
				  	  Similarly, we have 
				  	  \begin{displaymath}
				  	  	\abs{s-\inf{s_n:n>N_1}}<\epsilon
				  	  \end{displaymath}
			  	  	  which implies $\inf\{s_n: n>N_1\}>s-\epsilon\Rightarrow\forall n>N_1,\ s_n>s-\epsilon$.
			  	  	  Therefore,
			  	  	  \begin{displaymath}
			  	  	  	\forall n>\max\{N_0,N_1\},\ s-\epsilon<s_n<s+\epsilon\Rightarrow |s_n-s|<\epsilon	
			  	  	  \end{displaymath}
		  	  	  	  \item If $\lim\inf s_n=\lim\sup s_n=+\infty$, then
		  	  	  	  \begin{displaymath}
		  	  	  	  	\lim\inf s_n=+\infty\Rightarrow \forall M>0,\ \inf\{s_n:n>N_0\}>M\Rightarrow n>N_0,\ s_n>M.
		  	  	  	  \end{displaymath}
	  	  	  	  	  \item If $\lim\inf s_n=\lim\sup s_n=-\infty$, then
	  	  	  	  	  \begin{displaymath}
	  	  	  	  	  	\lim\sup s_n=-\infty\Rightarrow \forall M<0,\ \sup\{s_n:n>N_0\}<M\Rightarrow n>N_0,\ s_n<M.
	  	  	  	  	  \end{displaymath}
				  \end{itemize}
		\end{enumerate}
		\end{proof}
		
		\begin{dfn}\label{def:cauchy-seq}
		A $(s_n)$ is called a \emph{Cauchy sequence} if 
		\begin{displaymath}
			\forall\epsilon>0,\ \exists N\ \text{such that}\ m,n>N\Rightarrow|s_n-s_m|<\epsilon
		\end{displaymath}
		\end{dfn}
		\begin{lem}
		Convergent sequences are Cauchy sequences.
		\end{lem}
		\begin{proof}
		Suppose $\lim s_n=s$. Let $\epsilon>0$ then
		\begin{align*}
			n,m>N &\Rightarrow |s_n-s|<\frac{\epsilon}{2}\ \text{and}\ |s_m-s|<\frac{\epsilon}{2}\\
				  &\Rightarrow |s_n-s_m|=|s_n-s+s-s_m|\leq|s_n-s|+|s-s_m|<\epsilon.
		\end{align*}
		\end{proof}
		
		\begin{lem}
		Cauchy sequences are bounded.
		\end{lem}
		\begin{proof}
		By \ref{def:cauchy-seq} and set $\epsilon=1$ we have
		\begin{displaymath}
			m,n>N\Rightarrow|s_n-s_m|<1
		\end{displaymath}
		In particular $n>N\Rightarrow |s_n|-|s_{N+1}|\leq|s_n-s_{N+1}|<1$. Let $M=\max\{|s_{N+1}|+1,|s_1|,|s_2|,\dots,|s_N|\}$ then $|s_n|\leq M$.
		\end{proof}
	
		\begin{thm}\label{def:cauchy iff convergent}
		A sequence is a convergent sequence if and only if it is a Cauchy sequence.
		\end{thm}
		\begin{proof}
		We've already proved the right direction so we only need to proved the left direction by showing $\lim\inf s_n=\lim\sup s_n$ from \ref{def:condition for limit}.
		Let $\epsilon>0$ and since $(s_n)$ is a Cauchy sequence, there exists $N$ such that
		\begin{displaymath}
			m,n<N \Rightarrow |s_n-s_m|<\epsilon\Rightarrow \forall m>N,\ v_N=\sup\{s_n: n>N\}\leq s_m+\epsilon
		\end{displaymath}
		Now $v_N-\epsilon$ becomes a lower bound for $\{s_m: m>N\}$ so $v_N-\epsilon\leq\inf\{s_m: m>N\}=u_N$.
		Thus
		\begin{displaymath}
			\lim\sup s_n\leq v_N\leq u_N+\epsilon\leq\lim\inf s_n+\epsilon
		\end{displaymath}
		Since this is true for all $\epsilon>0$, $\lim\sup s_n$ cannot be greater than $\lim\inf s_n$ (imagine $\epsilon$ is extremely small, then $\lim\sup s_n>\lim\inf s_n+\epsilon$). Thus we have $\lim\inf s_n\geq\lim\sup s_n$. $\lim\inf s_n\leq\lim\sup s_n$ is obviously true, so we have the equality.
		\end{proof}
		\newpage
		
	\section{Subsequences}
		\begin{dfn}
		Suppose $(s_n)_{n\in\mathbb{N}}$ is a sequence. A \emph{subsequence} of this sequence is $(t_k)_{k\in\mathbb{N}}$ where for each $k$ there is a positive integer $n_k$ such that
		\begin{equation*}
			n_1<n_2<\cdots<n_k<n_{k+1}<\cdots
		\end{equation*}
		and
		\begin{equation*}
			t_k=s_{n_k}.
		\end{equation*}
		Thus $(t_k)$ is just a selection of some [possibly all] of the $s_n$'s taken in order.
		\end{dfn}
		For the subset $\{n_1,n_2,\dots\}$ there is a natural function $\sigma$ given by $\sigma(k)=n_k$ for $k\in\mathbb{N}$. The function $\sigma$ "selects" an infinite subset of $\mathbb{N}$ in order. Then
		the subsequence of $s$ corresponding to $\sigma$ is simply the composite function $t=s\circ\sigma$. That is
		\begin{displaymath}
			t_k=t(k)=s\circ\sigma(k)=s(\sigma(k))=s(n_k)=s_{n_k}\quad\text{for}\quad k\in N.
		\end{displaymath}
		Notice that $\sigma$ needs to be an \emph{increasing} function. 
		
		Recall that the set $\Q$ of rational numbers is \emph{countable}: there is a bijection from $\N$ to $\Q$. Therefore we have a sequence $(q_n)=(q_1,q_2,q_3,\dots)$ such that $\{q_n: n\in\N\}=\Q$. Then we have the following proposition:
		\begin{thm}
		Let $(q_n)$ be an enumeration of $\Q$. Then for any $a\in\R$, there exists a subsequence $(q_{n_k})$ of $(q_n)$ such that $q_{n_k}\rightarrow a$.
		\end{thm}
		\begin{proof}
		First there exists $r_1\in\Q$ such that $a-1<r_1<a+1$ by the denseness of $\Q$. Since $(q_n)$ is an enumeration of $\Q$, there exists $n_1\in\N$ such that $q_{n_1}=r_1$.
		
		Given that we've already constructed $n_1,\dots,n_k$ such that $a-\frac{1}{j}<q_{n_j}<a+\frac{1}{j}$ for $j=1,\dots,k$. Since there are infinitely many rational numbers between $a-\frac{1}{k+1}$ and $a+\frac{1}{k+1}$, and only finite many of them have been selected as $q_{n_1},\dots,q_{n_k}$, we are able to find $n_{k+1}>n_k$ such that $a-\frac{1}{k+1}<q_{n_{k+1}}<a+\frac{1}{k+1}$. 
		
		Now we have $(q_{n_k})$ such that $a-\frac{1}{k}<q_{n_k}<a+\frac{1}{k}$ for each $k\in\N$. Thus by Squeeze Lemma, $\lim_k q_{n_k}=a$.
		\end{proof}
	
		\setcounter{equation}{0}
		\begin{thm}\label{def:limit-subseq}
		Let $(s_n)$ be a sequence.
		\begin{enumerate}[(i)]
			\item If $t$ is in $\mathbb{R}$ then there is a subsequence of $(s_n)$ converging to $t$ if and only if
			the set $\{n\in\mathbb{N}: |s_n-t|<\epsilon\}$ is \emph{infinite} for all $\epsilon>0$.
			\item If $(s_n)$ is unbounded above, it has a subsequence with limit $+\infty$.
			\item If $(s_n)$ is unbounded below, it has a subsequence with limit $-\infty$.
		\end{enumerate}
		In each case, the subsequence can be taken to be \emph{monotonic}.
		\end{thm}
		\begin{proof}
		The forward implications are easy to check. Let's check these backward implications:
		\begin{enumerate}[(i)]
			\item First suppose $\{n\in\mathbb{N}: s_n=t\}$ is infinite. Then we can simply create a subsequence $(s_{n_k})_{k\in\mathbb{N}}$ such that $s_{n_k}=t$ for all $k$.
			
			Otherwise, suppose $\{n\in\mathbb{N}: s_n=t\}$ is finite. Then
			\begin{displaymath}
				\{n\in\mathbb{N}: 0<|s_n-t|<\epsilon\}\quad\text{is infinite for all}\quad\epsilon>0.
			\end{displaymath}
			Since these sets are equal to
			\begin{displaymath}
				\{n\in\mathbb{N}: t-\epsilon<s_n<t\}\cup\{n\in\mathbb{N}: t<s_n<t+\epsilon\}
			\end{displaymath}
			and these sets get smaller as $\epsilon\rightarrow0$ we have
			\begin{equation}
				\{n\in\mathbb{N}: t-\epsilon<s_n<t\}\quad\text{is infinite for all}\quad\epsilon>0
			\end{equation}
			or
			\begin{equation}
				\{n\in\mathbb{N}: t<s_n<t+\epsilon\}\quad\text{is infinite for all}\quad\epsilon>0
			\end{equation}
			otherwise for sufficiently small $\epsilon>0$ the sets in both (1) and (2) would be finite.
			
			Assume (1) holds, now we want to construct a $(s_{n_k})_{k\in\mathbb{N}}$ satisfying
			\begin{equation}
				t-1<s_{n_1}<t\quad\text{and}\quad\max\{s_{n_{k-1}},t-\frac{1}{k}\}\leq s_{n_k}<t\quad\text{for}\quad k\geq2
			\end{equation}
			Assume $n_1,\dots,n_{k-1}$ have been selected satisfying (3) and show how to select $n_k$. This is called "inductive definition" or "definition by induction". A subsequence satisfying (3) is a monotone increasing sequence and by Squeeze Formula $\lim_ks_{n_k}=t$. Here is the construction:
			By (1) we can select $n_1$ such that $t-1<s_{n_1}<t$. Suppose we've selected $n_1,\dots,n_{k-1}$ so that $n_1<n_2,\cdots<n_{k-1}$ and
			\begin{equation}
				\max\{s_{n_{j-1}},t-\frac{1}{j}\}\leq s_{n_j}<t\quad\text{for}\quad j=2,\dots,k-1
			\end{equation}
			By using (1) with $\epsilon=max\{s_{n_{k-1}},t-\frac{1}{k}\}$, we can select $n_k>n_{k-1}$ satisfying (4) for $j=k$, so (3) also holds for $k$. 
			\item Given $n_1=1$ and $n_1<\cdots<n_{k-1}$, select $n_k>n_{k-1}$ so that $s_{n_k}>\max\{s_{n_{k-1}},k\}$. This is possible since $(s_n)$ is unbounded above. Then the subsequence will be monotonically unbounded above thereby have limit $+\infty$.
		\end{enumerate}
		\end{proof}
		
		\begin{thm}\label{def:subsequence converges to the same limit}
		If $(s_n)$ converges, then every subsequence converges to the same limit.
		\end{thm}
		\begin{proof}
		Let $(s_{n_k})$ denote a subsequence of $(s_n)$. Note that $n_k\geq k$ for all k. Let $s=\lim s_n$ and $\epsilon>0$. There exists $N$ so that $n>N\Rightarrow|s_n-s|<\epsilon$. Since $n_k\geq k>N$, $|s_{n_k}-s|<\epsilon$. Thus
		\begin{displaymath}
			\lim\limits_{k\rightarrow\infty}s_{n_k}=s.
		\end{displaymath}
		\end{proof}
		In the other way, if there are two subsequences of $(s_n)$ with different limits, $(s_n)$ does not converge.
		
		\begin{thm}
		Every sequence $(s_n)$ has a monotonic subsequence.
		\end{thm}
		\setcounter{equation}{0}
		\begin{proof}
		Define $n$-th term is \emph{dominant} if it is greater than every term which follows it
		\begin{equation}
			s_m<s_n\quad\text{for all}\quad m>n
		\end{equation}
		\begin{itemize}
			\item Case 1: Suppose there are infinitely many dominant terms, then we can easily construct a monotone decreasing subsequence.
			\item Case 2: Suppose there are only finitely many dominant terms. Select $n_1$ so that $s_{n_1}$ is beyond
			all the dominant terms of the sequence. Then
			\begin{equation}
				\text{given}\,N\geq n_1\,\text{there exists}\,m>N\,\text{such that}\quad s_m\geq s_N. 
			\end{equation}
		Suppose $n_1,\dots,n_{k-1}$ have been selected so that
		\begin{equation}
			n_1<n_2<\cdots<n_{k-1}
		\end{equation}
		and
		\begin{equation}
			s_{n_1}\leq\cdots\leq s_{n_{l-1}}
		\end{equation}
		Apply (2) with $N=n_{k-1}$ we can select $n_k>n_{k-1}$ such that $s_{n_k}\geq s_{n_{k-1}}$. Then the procedure continues by induction and we obtain an increasing subsequence.
		\end{itemize}
		\end{proof}
		
		\begin{thm}[Bolzano-Weierstrass Theorem]\label{def:B-W}
			Every bounded sequence has a convergent subsequence.
		\end{thm}
		\begin{proof}
			If $(s_n)$ is a bounded sequence, it has a bounded monotonic subsequence which converges. 
		\end{proof}
		
		\begin{dfn}
		Let $(s_n)$ be a sequence in $\mathbb{R}$. A \emph{subsequential limit} is any real number or symbol $+\infty$ or $-\infty$ that is the limit of some subsequence of $(s_n)$.
		\end{dfn}
		
		\setcounter{equation}{0}
		\begin{thm}\label{thm:subsequence with limit limsup or liminf}
		Let $(s_n)$ be any sequence. There exists a monotonic subsequence whose limit is $\lim\sup s_n$, and there exists a monotonic subsequence whose limit is $\lim\inf s_n$.
		\end{thm}
		\begin{proof}
			If $(s_n)$ is not bounded above, then by \ref{def:limit-subseq}(ii) there is a monotonic subsequence with limit $+\infty=\lim\sup s_n$. The proof for not bounded below is similar.
			
			Now if $(s_n)$ is bounded above, then let $t=\lim\sup s_n$, and consider $\epsilon>0$. There exists $N_0$ so that
			\begin{displaymath}
				\sup\{s_n: n>N\}<t+\epsilon\quad\text{for}\quad N\geq N_0.
			\end{displaymath}
			In particular, $s_n<t+\epsilon$ for all $n>N_0$. We now claim
			\begin{equation}
				\{n\in\mathbb{N}: t-\epsilon<s_n<t+\epsilon\}\quad\text{is infinite.}
			\end{equation}
			Otherwise, there exists $N_1>N_0$ so that $s_n\leq t-\epsilon$ for $n>N_1$(WHY???). Then $\sup\{s_n: n>N\}\leq t-\epsilon$ for $N\geq N_1$, so that $\lim\sup s_n<t$, a contradiction. Since (1) holds true for all $\epsilon>0$, \ref{def:limit-subseq}(i) shows that there is a monotonic subsequence verges to $t=\lim\sup s_n$.
		\end{proof}
		
		\begin{thm}\label{def:subsequential limit condition}
		Let $(s_n)$ be any sequence in $\mathbb{R}$, and let $S$ denote the set of subsequential limits of $(s_n)$.
		\begin{enumerate}[(i)]
			\item S is nonempty.
			\item $\sup S=\lim\sup s_n$ and $\inf S=\lim\inf s_n$.
			\item $\lim s_n$ exists if and only if $S$ has exactly one element, namely $\lim s_n$.
			\item $\lim\sup s_n\in S$ and $\lim\inf s_n\in S$.
		\end{enumerate}
		\end{thm}
		\begin{proof}
		\begin{enumerate}[(i)]
			\item[] 
			\item By the last theorem.
			\item Consider any limit $t$ os a subsequence $(s_{n_k})$ of $(s_n)$. By \ref{def:condition for limit} $t=\lim\inf_ks_{n_k}=\lim\sup_ks_{n_k}$. Since $n_k>k$ for all $k$, we have $\{s_{n_k}: k>N\}\subseteq\{s_n: n>N\}$ for each $N\in\mathbb{N}$. Therefore
			\begin{displaymath}
				\lim\limits_{n}\inf s_n\leq\lim\limits_{k}\inf s_{n_k}=t=\lim\limits_{k}\sup s_{n_k}\leq	\lim\limits_{n}\sup s_n
			\end{displaymath}
			The inequality below holds true for all $t$ in $S$, so
			\begin{displaymath}
				\lim\inf s_n\leq\inf S\leq\sup S\leq\lim\sup s_n
			\end{displaymath}
			By the last theorem we know both $\lim\inf s_n$ and $\lim\sup s_n$ is in $S$, so (ii) holds.
			\item This is simply a reformulation of \ref{def:condition for limit}.
			\item This is from \ref{thm:subsequence with limit limsup or liminf}
		\end{enumerate}
		\end{proof}
		
		\begin{thm}
		Let $S$ denote the set of subsequential limits of a sequence $(s_n)$. Suppose $(t_n)$is a sequence in $S\cap\mathbb{R}$ and that $t=\lim t_n$. Then $t$ belongs to $S$.
		\end{thm}
		\begin{proof}
		Suppose $t$ is finite, then some $t_N$ is in $(t-\epsilon,t+\epsilon)$. Let $\delta=\min\{t+\epsilon-t_N,t_N-t+\epsilon\}$, so that
		\begin{displaymath}
			(t_N-\delta, t_N+\delta)\subseteq(t-\epsilon,t+\epsilon)
		\end{displaymath}
		Since $t_N$ is a subsequential limit, the set $\{n\in\mathbb{N}: s_n\in(t_N-\delta, t_N+\delta)\}$ is infinite,
		so the set $\{n\in\mathbb{N}: s_n\in(t-\epsilon, t+\epsilon)\}$ is also infinite. Thus by \ref{def:condition for limit} $t$ itself is a subsequential limit of $(s_n)$. 
		
		If $t=+\infty$, then clearly the sequence $(s_n)$ is unbounded above, so a subsequence of $(s_n)$ has limit $+\infty$ by \ref{def:condition for limit}. Thus $+\infty$ is also in $S$. A similar argument applies if $t=-\infty$.
		\end{proof}
		\newpage
		
	\section{lim sup's and lim inf's}
		\setcounter{equation}{0}
		\begin{thm}\label{thm:12.1}
		If $(s_n)$ converges to a positive real number $s$ and $(t_n)$ is any sequence, then
		\begin{displaymath}
			\lim\sup s_nt_n=s\cdot\lim\sup t_n.
		\end{displaymath}
		Here we allow the conventions $s\cdot(+\infty)=+\infty$ and $s\cdot(-\infty)=-\infty$ for $s>0$.
		\end{thm}
		\begin{proof}
		We first want to show
		\begin{equation}
			\lim\sup s_nt_n\geq s\cdot\lim\sup t_n.
		\end{equation}
		We have three cases. Let $\beta=\lim\sup t_n$.
		\begin{enumerate}
			\item Suppose $\beta$ is finite. By \ref{def:subsequence with limit limsup or liminf}, there exists a subsequence
			$(t_{n_k})$ of $(t_n)$ such that $\lim_{k\rightarrow\infty}t_{n_k}=\beta$. We also have $\lim_{k\rightarrow\infty}s_{n_k}=s$ by \ref{def:subsequence converges to the same limit}, so $\lim_{k\rightarrow\infty}s_{n_k}t_{n_k}=s\beta$ thus $(s_{n_k}t_{n_k})$ is a subsequence of $(s_nt_n)$ converging to $s\beta$, and therefore $s\beta\leq\lim\sup s_nt_n$ by \ref{def:subsequential limit condition}. Thus (1) holds.
			\item Suppose $\beta=+\infty$. Then there exists a subsequence $(t_{n_k})$ of $(t_n)$ converging to $+\infty$.
			Since $\lim_{k\rightarrow\infty}s_{n_k}=s>0$, $\lim_{k\rightarrow\infty}s_{n_k}t_{n_k}=+\infty$. Hence $\lim\sup s_nt_n=+\infty$. Thus (1) holds.
			\item Suppose $\beta=-\infty$. Then the right-hand side of (1) is equal to $-\infty$. Hence (1) is obviously true.	
		\end{enumerate}
		To show $\lim\sup s_nt_n\leq s\cdot\lim\sup t_n$, we may ignore the first few terms of $(s_n)$ and assume all $s_n\neq0$. Then we can write $\lim\frac{1}{s_n}=\frac{1}{s}$. Now we apply (1) with $s_n$ replaced by $\frac{1}{s_n}$ and $t_n$ replaced by $s_nt_n$:
		\begin{displaymath}
			\lim\sup t_n=\lim\sup(\frac{1}{s_n})(s_nt_n)\geq(\frac{1}{s})\lim\sup s_nt_n,
		\end{displaymath}
		which is 
		\begin{displaymath}
			\lim\sup s_nt_n\leq s\cdot\lim\sup t_n
		\end{displaymath}
		Therefore we have $\lim\sup s_nt_n=s\cdot\lim\sup t_n$.
		\end{proof}
		
		\setcounter{equation}{0}
		\begin{thm}\label{thm:12.2}
		Let $(s_n)$ be any sequence of nonzero real numbers. Then we have
		\begin{displaymath}
			\lim\inf\abs{\frac{s_{n+1}}{s_n}}\leq\lim\inf|s_n|^{1/n}\leq\lim\sup|s_n|^{1/n}\leq\lim\sup\abs{\frac{s_{n+1}}{s_n}}
		\end{displaymath}
		\end{thm}
		\begin{proof}
			The middle inequality is obvious. The first and third inequalities have similar proofs. We will prove the third
			inequality as below:
			
			Let $\alpha=\lim\sup|s_n|^{1/n}$ and $L=\lim\sup\abs{\frac{s_{n+1}}{s_n}}$. Assume $L<+\infty$. To prove $\alpha\leq L$ it suffices to show
			\begin{equation}
				a\leq L_1\quad\text{for any}\quad L_1>L
			\end{equation}
			because if $\exists L_1>L,\ \alpha>L_1$, then $\alpha>L_1>L\Rightarrow\alpha>L$
			Since
			\begin{displaymath}
				L=\lim\sup\abs{\frac{s_{n+1}}{s_n}}=\lim\limits_{N\rightarrow\infty}\sup \left\{ \abs{\frac{s_{n+1}}{s_n}}: n>N \right\} <L_1
			\end{displaymath}
			there exists a positive integer $N$ such that
			\begin{displaymath}
				\sup\left\{\abs{\frac{s_{n+1}}{s_n}}: n\geq N\right\}<N_1
			\end{displaymath}
			Thus
			\begin{equation}
				\abs{\frac{s_{n+1}}{s_n}}<L_1\quad\text{for}\quad n\geq N
			\end{equation}
			Now for $n>N$ we can write
			\begin{displaymath}
				|s_n|=\abs{\frac{s_n}{s_{n-1}}}\cdot\frac{s_{n-1}}{s_{n-2}}\cdots\abs{\frac{s_{N+1}}{s_N}}\cdot|s_N|.
			\end{displaymath}
			Apply (2) we see that
			\begin{align*}
				|s_n|&<L_1^{n-N}|s_N|\quad\text{for}\quad n>N\\
				|s_n|&<L_1^na\quad\text{for}\quad n>N.\qquad \text{for}\ a=L_1^{-N}|s_N|\\
				|s_n|^{1/n}&<L_1a^{1/n}\quad\text{for}\quad n>N
			\end{align*}
			Since $\lim_{n\rightarrow\infty}a^{1/n}=1$ we conclude $\alpha=\lim\sup|s_n|^{1/n}\leq L_1$
		\end{proof}
		
		\begin{cor}\label{def:12.3}
		If $\lim\abs{\frac{s_{n+1}}{s_n}}$ exists [and equals L], then $\lim|s_n|^{1/n}$ exists [and equals L].
		\end{cor}
		\begin{proof}
		If $\lim\abs{\frac{s_{n+1}}{s_n}}=L$, then all four values in the last theorem are equal to $L$. Hence
		$\lim|s_n|^{1/n}=L$ by \ref{def:condition for limit}.
		\end{proof}
		\newpage
	\section{Some Topological Concepts in Metric Spaces}
		\begin{dfn}
		Let $X$ be a set, and suppose $d$ is a function $d: X\times X\rightarrow[0,\infty]$ defined for all pairs $(x,y)$ of elements from $X$ satisfying
		\begin{enumerate}
			\item $d(x,x)=0$ for all $x\in S$ and $d(x,y)>0$ for distinct $x,y\in X$. (Positive Definiteness)
			\item $d(x,y)=d(y,x)$ for all $x,y\in X$. (Symmetry)
			\item $d(x,z)\leq d(x,y)+d(y,z)$ for all $x,y,z\in X$. (Triangle Inequality)
		\end{enumerate}
		Such a function $d$ is called a \emph{distance function} or a \emph{metric} on $X$. A \emph{metric space} $X$ is a set $X$ together with a metric on it.
		\end{dfn}
		\begin{rem}
		The positive definiteness can be also expressed as $\forall x,y\in X\ d(x,y)\geq 0$ and $d(x,y)=0\iff x=y$. The distance function cannot be $+\infty$.
		\end{rem}
		\begin{eg}
		Discrete metric space is defined as
		\begin{displaymath}
			\text{For any set $X$ with metric or distance function as}\begin{cases}
				1\quad\text{$x\neq y$}\\ 0\quad\text{$x=y$}.
			\end{cases}
		\end{displaymath}
		Notice that all sets in discrete metric space are both open and closed.
		\end{eg}
		
		\begin{dfn}[Convergence]
		A sequence $(x_n)$ in a metric space $(X,d)$ converges to $x$ in $X$ if $\lim_{n\rightarrow\infty}d(s_n,s)=0$.
		\end{dfn}
		\begin{rem}
		In other words, a sequence $(x_n)$ converges to $x$ if for any $\epsilon>0$, there exists $N\in\N$ such that $n\geq N\implies d(x_n,x)<\epsilon$.
		\end{rem}
		
		\begin{dfn}[Cauchy]
		A sequence $(x_n)$ in $X$ is a \emph{Cauchy} if for any $\epsilon>0$ there exists an $N\in\N$ such that
		\begin{displaymath}
			m,n\geq\implies d(x_m,x_n)<\epsilon.
		\end{displaymath}
		\end{dfn}
		
		\begin{dfn}[Complete]
		A metric space $(X,d)$ is \emph{complete} if every Cauchy sequence in $X$ converges.
		\end{dfn}
		\begin{rem}
		Every convergent sequence $(x_n)$ in $X$ is Cauchy.
		\end{rem}
		
		\begin{dfn}[Open Ball]
			Let $(X,d)$ be a metrc space. For $x\in X$ and $r>0$, the open ball of radius $r$ centered at $x$ is the set
			\begin{displaymath}
				B_r(x)=\{y\in X: d(y,x)<r\}
			\end{displaymath}
		\end{dfn}
	
		\begin{dfn}[Interior Point]
			Let $(X,d)$ be a metric space. Let $E$ be a subset of $X$. An element $x\in E$ is \emph{interior} to $E$ if for some $r>0$ we have
			\begin{displaymath}
				B_r(x)\subseteq E
			\end{displaymath}
			We write $E^\circ$ for the set of points in $E$ that are interior to $E$.
		\end{dfn}
		\begin{rem}
		\begin{itemize}
			\item The relationship between $E$ and $X$ may affect whether a point in $E$ is interior to $E$. For example, for $E=[0,1]\subset[-1,2]=X$, $0$ is not interior to $[0,1]$. However if $E=[0,1]\subset[0,1]=X$, then $0$ is interior to $0$ since there is not point in $X$ beyond the left of $0$.
			\item $E^\circ$ is open.
			\item $E=E^\circ$ if and only if $E$ is open.
			\item If $F$ is an open set such that $F\subseteq E$, then $F\subseteq E^\circ$.
		\end{itemize}		
		\end{rem}
		
		\begin{dfn}[Open Set]
			A set $E\subseteq X$ is \emph{open} if every point $x\in E$ is an interior point of $E$. i.e., if $E=E^\circ$.	
		\end{dfn}
		\begin{rem}
		\begin{itemize}
			\item[]
 
			\item A set being open does \textbf{not} mean it is \textbf{not} closed. e.g. $[0,1)$ is neither open nor closed.
		\end{itemize}
		\end{rem}
		\begin{eg}
			\begin{itemize}
				\item[]
				\item $(a,b),(a,\infty),(-\infty,a)$ are open sets.
	 			\item In $\R$, $\Q$ is \emph{not} open since $B_r(q)$ may contain irrational numbers in $\R$ so $B_r(q)\nsubseteq\Q$.
		        \item In any metric space $(X,d)$, $X$ and $\Q$ are open trivially.
			\end{itemize}
		\end{eg}
		
		\begin{thm}[Open ball is open]
		Let $(X,d)$ be a metric space. Given $x\in X$ and $r>0$, $B_r(x)$ is an open set in $X$.
		\end{thm}
		\begin{proof}
		Consider arbitrary $y\in B_r(x)$ and let $s=r-d(x,y)$. It is easy to show that $B_x(y)\subseteq B_r(x)$. Thus $y$ is an interior point of $B_r(x)$. Since $y$ is arbitrary, by the definition $B_r(x)$ is open.
		\end{proof}
		
		\begin{thm}[Union and intersection of open sets]\label{thm: Union and intersection of open sets}
		Let $(X,d)$ be a metric space.
		\begin{enumerate}[(i)]
			\item If $\{\mathcal{U}_\alpha\}_{\alpha\in \mathcal{A}}$ is any collection of open sets in $X$, then $\bigcup_{\alpha\in \mathcal{A}}\mathcal{U}_\alpha$ is open. i.e. the union of \emph{any} collection of open sets is open.
			\item If $\{\mathcal{U}_1,\dots,\mathcal{U}_n\}$ is a finite collection of open sets in $X$, then $\bigcap_{i=1}^{n}\mathcal{U}_i$ is open.
		\end{enumerate}
		\end{thm}
		\begin{proof}
		\begin{enumerate}[(i)]
			\item[]
			\item Consider $x\in\bigcup_{\alpha\in \mathcal{A}}\mathcal{U}_\alpha$, then $\exists \beta\in\mathcal{A}$ such that $x\in\mathcal{U}_\beta$. Since $\mathcal{U}_\beta$ is open, $\exists r>0$ such that $B_r(x)\subseteq\mathcal{U}_\beta\subseteq\bigcup_{\alpha\in \mathcal{A}}\mathcal{U}_\alpha$. Thus $x$ is interior to $\bigcup_{\alpha\in \mathcal{A}}\mathcal{U}_\alpha$, completing the proof.
			\item Consider $x\in\bigcap_{i=1}^{n}\mathcal{U}_i$. Since $x\in\mathcal{U}_i$ for $i=1,\dots,n$, $\exists r_i>0$ such that $B_{r_i}(x)\subseteq\mathcal{U}_i$. Take $r=\min\{r_1,\dots,r_n\}$, then clearly $B_r(x)\subseteq\bigcap_{i=1}^{n}\mathcal{U}_i$.
		\end{enumerate}
		\end{proof}
		\begin{rem}
		The examples for infinite collection in (ii) is $\bigcap_{n=1}^{\infty}(1-\frac{1}{n}, 1+\frac{1}{n})=\{1\}$. Since $1$ is not an interior point of $\{1\}$, $\{1\}$ is not open.
		\end{rem}
	
		\begin{dfn}[Complement]
			For a set $E\subseteq X$, the \emph{complement} of $E$ is the set $E^C=X\backslash E=\{x\in X: x\notin E\}$.
		\end{dfn}
	
		\begin{dfn}[Limit Point]
			For a set $E\subseteq X$, a point $x\in X$ is a \emph{limit point} of $E$ if for any $r>0$, we have that $(B_r(x)\backslash\{x\})\cap E\neq\emptyset$.\smallskip
			
			$E^\prime$ denotes the set of all limit points of $E$.
		\end{dfn}
		\begin{rem}
		\begin{itemize}
			\item[]
			\item In other words, for any radius $r>0$, no matter how small is $r$, there is some element of $E$ which sits in $B_r(x)$ other than $x$ itself. 
			\item If $E\subseteq F$, then $E^\prime\subseteq F^\prime$.
			\item $(E\cup F)^\prime=E^\prime\cup F^\prime$.
		\end{itemize}
		\end{rem}
		\begin{eg}
		\begin{itemize}
			\item[]
			\item In $\R$, the set of limit points of $(0,1)$ is $[0,1]$.
			\item In $\R$, the only limit point of $\{\frac{1}{n}: n\in\N\}$ is $0$.
			\item In $\R$, the set of limit point of $\Q$ is $\R$. 
		\end{itemize}
		\end{eg}
		
		\begin{thm}
		A point $x$ is a limit point of a set $E\subseteq X$ if and only if $x=\lim x_n$ for some sequence $x_n$ of points in $E\backslash\{x\}$. 
		\end{thm}
		\begin{proof}
		See homework 3.7.
		\end{proof}
	
		\begin{dfn}[Isolated Point]
			For a set $E\subseteq X$, $x\in E$ is called an \emph{isolated point} if $x$ is not a limit point of $E$
		\end{dfn}
		\begin{rem}
			In other words, $x$ is an isolated point or not a limit point of $E$ if there exists a radius $r$ such that $B_r(x)$ does not contain any element of $E$ except $x$ itself.
		\end{rem}
		\begin{eg}
		\begin{itemize}
			\item[]
			\item In $\R$, every integer is an isolated point of $\mathbb{Z}$.
			\item In $\R$, the set $Q$ has no isolated point.
			\item In $\R$, every element of $\{\frac{1}{n}: n\in\N\}$ is an isolated point. 
		\end{itemize}
		\end{eg}
	
		\begin{dfn}[Closed Set]
			A set is \emph{closed} if $E^\prime\subseteq E$.
		\end{dfn}
		\begin{dfn}[Closed Set]
			Let $(X,d)$ be a metric space. A subset $E$ of $X$ is \emph{closed} if its complement $E^\com$ is an open set.
		\end{dfn}
		\begin{rem}
		\begin{itemize}
			\item[]
			\item The above two definitions are equivalent.
			\item In other words, $E$ contains all of its limit points, or every limit point of $E$ is in $E$.
			\item In any metric space $(X,d)$, $X$ and $\emptyset$ are closed.
			\item A set being closed does \textbf{not} mean it is \textbf{not} open. e.g. $[0,1)$ is neither open nor closed.
		\end{itemize}
		\end{rem}
		\begin{eg}
			\begin{itemize}
				\item In $\R$, $[0,1]$ is closed. $[a,\infty),(-\infty,a]$ are closed.
				\item In $\R$, the set $\{\frac{1}{n}: n\in\N\}$ is not closed, but $\{\frac{1}{n}: n\in\N\}\cup\{0\}$ is closed.
				\item In any metric space, $X$ and $\emptyset$ are closed.
				\item All finite sets do not have limit point, so they are trivially closed.
			\end{itemize}
		\end{eg}
		
		\begin{thm}
		A set $E\subseteq \R$ is closed if and only if every Cauchy sequence contained in $E$ has a limit that is also an element of $E$.
		\end{thm}
		
		\begin{thm}[The set of limit points is closed]
		Let $(X,d)$ be a metric space. Let $E\subseteq X$, then $E^\prime$, (the set of limit points of $E$), is closed.
		\end{thm}
		\begin{proof}
		We need to show for any limit point $x$ of $E^\prime$, $x$ is in $E^\prime$. Since $x$ is a limit point of $E^\prime$, $\forall r>0$, $(B_r(x)\backslash\{x\})\cap E^\prime\neq\emptyset$. i.e. there exists $y\in E^\prime$ such that $y\neq x$ and $y\in B_r(x)$. Take $s=\min\{r-d(x,y), d(x,y)\}$. Since $y\in E^\prime$, $(B_s(y)\backslash\{y\})\cap E\neq\emptyset$. i.e. $\exists z\in(B_s(y)\backslash\{y\})\cap E\neq\emptyset$.\smallskip
		
		Now since $s<r-d(x,y)$, $d(x,z)\leq d(x,y)+d(y,z)<d(x,y)+(r-d(x,y))=r\implies z\in B_r(x)$. Also since $s<d(x,y)$, $z\neq x$. Thus $z\in(B_r(x)\backslash\{x\})\cap E\implies (B_r(x)\backslash\{x\})\cap E\neq\emptyset$, which implies $x$ is a limit point of $E$. i.e. $x\in E^\prime$, completing the proof. 
		\end{proof}
		
		\begin{thm}[Union and intersection of closed sets]
		\begin{enumerate}[(i)]
			\item[]
			\item If $\{\mathcal{E}_\alpha\}_{\alpha\in\mathcal{A}}$ is any collection of closed set, then $\bigcap_{\alpha\in\mathcal{A}}\mathcal{E}_\alpha$ is closed.
			\item If $\{\mathcal{E}_1,\dots,\mathcal{E}_n\}$ is a finite collection of closed sets in $X$, then $\bigcup_{i=1}^{n}\mathcal{E}_i$ is closed.
		\end{enumerate}
		\end{thm}
		\begin{proof}
		\begin{enumerate}[(i)]
			\item[]
			\item Observe that $\left(\bigcap_{\alpha\in\mathcal{A}}\mathcal{E}_\alpha\right)^\com=\bigcup_{\alpha\in\mathcal{A}}\mathcal{E}_\alpha^\com$. Since $\mathcal{E}_\alpha$ is closed, $\mathcal{E}_\alpha^\com$ is open. By \ref{thm: Union and intersection of open sets}, the union of open sets $\bigcup_{\alpha\in\mathcal{A}}\mathcal{E}_\alpha^\com$ is open, completing the proof.
			\item Observe that $\left(\bigcup_{i=1}^{n}\mathcal{E}_i\right)^\com=\bigcap_{i=1}^{n}\mathcal{E}_i^\com$. Since $\mathcal{E}_i$ is closed, $\mathcal{E}_i^\com$ is open. By \ref{thm: Union and intersection of open sets}, the intersection of finite open sets $\bigcap_{i=1}^{n}\mathcal{E}_i^\com$ is open, completing the proof.
		\end{enumerate}
		\end{proof}
		\begin{rem}
			$\bigcup_{x\in(0,1)}\{x\}=(0,1)$ is an example to the union of infinite closed sets is open in (ii).
		\end{rem}
		\newpage
			
		The proof above uses one of DeMorgan's Laws for sets.
		\setcounter{equation}{0}
		\begin{tcolorbox}[title=\textbf{DeMorgan's Laws for sets}]
		Suppose a metric space $(X,d)$ and let $\forall\alpha\in\mathcal{A}\ U_\alpha\in X$. Then $\bigcap_{\alpha\in\mathcal{A}}\mathcal{U}_{\alpha}^{\com}=\left(\bigcup_{\alpha\in\mathcal{A}}\mathcal{U}_\alpha\right)^\com$.
		\tcblower
		\begin{proof}
			We want to show both directions.
		\begin{itemize}
			\item[$\subseteq$:] Consider $u\in\bigcap_{\alpha\in\mathcal{A}}\mathcal{U}_{\alpha}^{\com}$, then we have
			\begin{align}
				\forall\alpha\in\mathcal{A}\ u\in\mathcal{U}_{\alpha}^{\com}&\implies\forall\alpha\in\mathcal{A}\ u\notin\mathcal{U}_{\alpha}\\
				&\implies u\notin \bigcup_{\alpha\in\mathcal{A}}\mathcal{U}_\alpha\\
				&\implies u\in\left(\bigcup_{\alpha\in\mathcal{A}}\mathcal{U}_\alpha\right)^\com.
			\end{align}
			(1) $\implies$ (2) because
			\begin{displaymath}
				\left(\neg\left(u\in\mathcal{U}_1\right)\right)\land\left(\neg\left(u\in\mathcal{U}_2\right)\right)\land\cdots=\neg\left((u\in\mathcal{U}_1\lor(u\in\mathcal{U}_2)\lor\cdots\right)=\neg\left(u\in\bigcup\mathcal{U}_i\right)
			\end{displaymath} 
			Thus $\bigcap_{\alpha\in\mathcal{A}}\mathcal{U}_{\alpha}^{\com}\subseteq\left(\bigcup_{\alpha\in\mathcal{A}}\mathcal{U}_\alpha\right)^\com$.
			\item[$\supseteq$:] Consider $u\in\left(\bigcup_{\alpha\in\mathcal{A}}\mathcal{U}_\alpha\right)^\com$, then we have
			\begin{align*}
				u\notin\bigcup_{\alpha\in\mathcal{A}}\mathcal{U}_\alpha &\implies\forall\alpha\in\mathcal{A}\ u\notin\mathcal{U}_\alpha\\
				&\implies\forall\alpha\in\mathcal{A}\ u\in\mathcal{U}_{\alpha}^{\com}\\
				&\implies u\in\bigcap_{\alpha\in\mathcal{A}}\mathcal{U}_{\alpha}^{\com}
			\end{align*}
			Thus $\bigcap_{\alpha\in\mathcal{A}}\mathcal{U}_{\alpha}^{\com}\supseteq\left(\bigcup_{\alpha\in\mathcal{A}}\mathcal{U}_\alpha\right)^\com$, and hence $\bigcap_{\alpha\in\mathcal{A}}\mathcal{U}_{\alpha}^{\com}=\left(\bigcup_{\alpha\in\mathcal{A}}\mathcal{U}_\alpha\right)^\com$.
		\end{itemize}
		\end{proof}
		\end{tcolorbox}\medskip
		
		\begin{dfn}[Bounded Set]
			A set $E\subseteq X$ is bounded if for some $x\in X$ and $M>0$ such that $d(x,y)\leq M$ for all $y\in E$.
		\end{dfn}
		\begin{rem}
		\begin{itemize}
			\item[]
			\item In $\R^k$, $X\subseteq\R^k$ is bounded if there exists $M>0$ such that $\forall\mathbf{x}\in X\ d(\mathbf{x},\mathbf{0})=\sqrt{x_1^2+\cdots+x_k^2}\leq M$.
			\item Finite union of bounded sets is bounded.
			\item Intersection of bounded sets is bounded.
			\item Contained in some open ball.
		\end{itemize}
		\end{rem}
		
		\begin{thm}
			In $R$, any closed and bounded sets always have maximum and minimum.
		\end{thm}
		
		\begin{dfn}[Closure]
			The \emph{closure} of $E$ in $X$ is $\bar{E}=E\cup E^\prime$.
		\end{dfn}
		\begin{rem}
		\begin{itemize}
			\item[]
			\item $\bar{E}$ is the intersection of all closed sets containing $E$.
			\item $\bar{E}$ is closed.
			\item $E=\bar{E}$ if and only if $E$ is closed.
			\item If $F$ is a closed set such that $E\subseteq F$, then $\bar{E}\subseteq F$.
			\item The union of closures of finite sets is equal to the closure of unions of the sets. i.e. $\overline{A\cup B}=\overline{A}\cup\overline{B}$.
		\end{itemize}
		\end{rem}
		
		\begin{thm}
		For any $E\subseteq X$, its closure $\bar{E}=E\cup E^\prime$ is closed and is the smallest closed set containing $A$.
		\end{thm}
		
		\begin{dfn}[Dense Set]
			A set $E\subseteq X$ is \emph{dense} in $X$ if $\bar{E}=X$.
		\end{dfn}
		\begin{eg}
		\begin{itemize}
			\item[]
			\item $\Q$ is dense in $\R$.
			\item In any metric space $(X,d)$, $X$ is dense in $X$. 
		\end{itemize}
		\end{eg}
		
		\begin{dfn}[Dense Set]
			A set $E\subseteq X$ is dense in $X$ if and only if for any $x\in X$ and $r>0$.
			\begin{displaymath}
				B_r(x)\cap E\neq\emptyset.
			\end{displaymath} 
		\end{dfn}
		
		\setcounter{equation}{0}
		\begin{lem}\label{def:sequence in R^k}
		\begin{itemize}
			\item[]
			\item A sequence $(\mathbf{x}^{(n)})$ in $\mathbb{R}^k$ converges to $\mathbf{x}=(x_1,\dots,x_k)$ if and only if for each $j=1,2\dots,k$, the sequence $(x_j^{(n)})$ converges in $\mathbb{R}$.
			\item A sequence $(\mathbf{x}^{(n)})$ in $\mathbb{R}^k$ is a Cauchy sequence if and only if each sequence $(x_j^{(n)})$ is a Cauchy sequence in $\mathbb{R}$.
		\end{itemize}	
		\end{lem}
		\begin{proof}
		First observe for $\textbf{x,y}\in\mathbb{R}^k$ and $j=1,\dots,k$
		\begin{align}
			|x_j-y_j|=\sqrt{(x_j-y_j)^2}\leq\sqrt{(x_1-y_1)^2+\cdots+(x_k-y_k)^2}&=d(\textbf{x,y})\notag\\
			&\leq \sqrt{k}\max\{|x_j-y_j|: j=1,\dots,k\}
		\end{align}
		First assertion: 
		\begin{itemize}
			\item[$\implies$:] Given that $(\textbf{x}^{(n)})$ converges to $\mathbf{x}$. For each $epsilon>0$ there exists $N\in\N$ such that $n\geq N\implies d(\textbf{x}^{(n)},\mathbf{x})<\epsilon$. Then by (1) for $j=1,\dots,k$
			\begin{displaymath}
				n\geq N\implies |x_j^{(n)}-x_j|\leq d(\textbf{x}^{(n)},\mathbf{x})<\epsilon,
			\end{displaymath}
			so $x_j^{(n)}\rightarrow x_j$.
			\item[$\impliedby$:] For $j=1,\dots,k$, $\forall \epsilon>0$, there exists $N_j\in\N$ such that 
			\begin{displaymath}
				n\geq N_j\implies |x_j^{(n)}-x_j|<\frac{\epsilon}{\sqrt{k}}.
			\end{displaymath}
			Take $N=\max\{N_1,\dots,N_k\}$, then by (1) we have
			\begin{displaymath}
				n\geq N\implies d(\textbf{x}^{(n)},\mathbf{x})\leq\sqrt{k}\max\{|x_j-y_j|: j=1,\dots,k\}<\sqrt{k}\cdot\frac{\epsilon}{\sqrt{k}}=\epsilon.
			\end{displaymath}
			Thus $(\textbf{x}^{(n)})\rightarrow\mathbf{x}$
		\end{itemize}
		Second assertion:	
		\begin{itemize}
			\item[$\Rightarrow$]: Suppose $(\textbf{x}^{(n)})$ is a Cauchy sequence, from the definition we know
			\begin{displaymath}
				m,n>N\Rightarrow d(\textbf{x}^{(m)},\textbf{x}^{(n)})<\epsilon
			\end{displaymath}
			From (1) we see
			\begin{displaymath}
				m,n>N\Rightarrow|x_j^{(m)}-x_j^{(n)}|<\epsilon
			\end{displaymath}
			so $(x_j^{(n)})$ is a Cauchy sequence.
			\item[$\Leftarrow$]: Suppose $(x_j^{(n)})$ is a Cauchy sequence, then for $j=1,\dots,k$
			\begin{displaymath}
				m,n>N_j\Rightarrow|x_j^{(m)}-x_j^{(n)}|<\frac{\epsilon}{\sqrt{k}}.
			\end{displaymath}
			If $N=\max\{N_1,N_2,\dots,N_k\}$, then by (1)
			\begin{displaymath}
				m,n>N\Rightarrow d(\textbf{x}^{(m)},\textbf{y}^{(n)})<\epsilon
			\end{displaymath}
			i.e. $(\textbf{x}^{(n)})$ is a Cauchy sequence.
		\end{itemize}
		\end{proof}
		
		\begin{thm}
		Euclidean k-space $\mathbb{R}^k$ is complete.
		\end{thm}
		\begin{proof}
		Consider a Cauchy sequence $(\textbf{x}^{(n)})$ in $\mathbb{R}^k$. By \ref{def:sequence in R^k}, each $(x_j^{(n)})$ is a Cauchy sequence. By \ref{def:cauchy iff convergent} each $(x_j^{(n)})$ converges. Thus by \ref{def:sequence in R^k} $(\textbf{x}^{(n)})$ converges.
		\end{proof}
	
		\begin{thm}[Bolzano-Weierstrass in $\R^k$]
		Every bounded sequence in $\mathbb{R}^k$ has a convergent subsequence.
		\end{thm}
		\begin{proof}
		Since $(\textbf{x}^{(n)})$ is bounded, then each $(x_j^{(n)})$ is bounded in $\mathbb{R}$. By \ref{def:B-W}, we
		could replace $(\textbf{x}^{(n)})$ by one of its subsequence, say $(\bar{\mathbf{x}}^{(n)})$, whose $(x_1^{(n)})$
		converges. By \ref{def:B-W} again, we may replace $(\textbf{x}^{(n)})$ by a subsequence of $(\textbf{x}^{(n)})$ such
		that both $(x_1^{(n)})$ and $(x_2^{(n)})$ converge. $(x_1^{(n)})$ still converges because \ref{def:subsequence converges to the same limit}. Repeating this argument by $k$ times, we obtain a new sequence $(\textbf{x}^{(n)})$ where each $(x_j^{(n)})$ converges, $j=1,\dots,k$, which is a subsequence of the original sequence, and it converges by \ref{def:sequence in R^k}. 
		\end{proof}
		\begin{rem}
		In any general metric space $(X,d)$, it is not true that any bounded sequence has a convergent subsequence. E.g. $(\Q,d)$ and infinite discrete metric space
		\end{rem}
	
		\begin{thm}
			Let $E$ be a subset of a metric space $(S,d)$.
			\begin{enumerate}
				\item $E$ is closed $\iff$ $E=E^-$.
				\item $E$ is closed $\iff$ $E$ contains the limit of every convergent sequence of points in $E$.
				\item An element is in $E^-$ $\iff$ it is the limit of some sequence of points in $E$.
				\item A point is in the boundary of $E$ $\iff$ it belongs to the closure of both $E$ and its complement.
			\end{enumerate}
		\end{thm}
	
		\noindent\rule{\textwidth}{1pt}
		\subsection*{Compactness}
		
		\begin{dfn}[Open Cover]
		Let $(X,d)$ be a metric space and $E\subseteq X$. An open cover of $E$ is a collection of open sets $\{\mathcal{U}_\alpha\}_{\alpha\in\mathcal{A}}$ such that $E\subseteq\bigcup_{\alpha\in\mathcal{A}}\mathcal{U}_\alpha$. An open cover is finite if it contains finitely many sets.
		\end{dfn}
		
		\begin{dfn}[Subcover]
		A subcover of an open cover $\{\mathcal{U}_\alpha\}_{\alpha\in\mathcal{A}}$ of $E$ is an \emph{open cover} $\{\mathcal{U}_\alpha\}_{\alpha\in\mathcal{B}}$ such that $\mathcal{B}\subseteq\mathcal{A}$.
		\end{dfn}
	
		\begin{dfn}[Compact Set]
		A set $E\subseteq X$ is compact if every open cover of $E$ has a \emph{finite} subcover.
		\end{dfn}
		\begin{eg}
		\begin{itemize}
			\item[]
			\item Every finite set is compact.
			\item Infinite discrete metric space is not compact.
			\item $\R$ is not compact: $\{(-n,n)\}_{n\in\N}$ is an open cover of $\R$ but does not have a finite subcover.
			\item $(0,1)$ is not compact: $\{(0,r)\}_{r\in(0,1)} $ is an open cover of $(0,1)$ but does not have a finite subcover.
			\item Closed interval in $R$ is compact.
		\end{itemize}
		\end{eg}
		
		\begin{thm}
		Compact sets are closed in any metric space.
		\end{thm}
		\begin{proof}
		Let $E\subseteq X$ be compact. To show $E$ is closed, we can show $E^\com$ is open. Consider $x\in E^\com$. For each $y\in E$, let $r_y:=\frac{1}{2}d(x,y)$. Clearly $\{B_{r_y}(y)\}_{y\in E}$ is an open cover of $E$ because each point in $E$ is a center of an open ball. By the assumption, $E$ is compact, so there is a finite subcover $\{B_{r_{y_1}}(y_1),\dots,B_{r_{y_n}}(y_n)\}$ such that $E\subseteq\bigcup_{i=1}^{n}B_{r_{y_i}}(y_i)$. 
		
		Now take $r=\min\{r_{y_1},\dots,r_{y_n}\}$, and hence $B_r(x)\cap(\bigcup_{i=1}^{n}B_{r_{y_i}}(y_i))=\emptyset$. Since $E\subseteq\bigcup_{i=1}^{n}B_{r_{y_i}}(y_i)$, $B_r(x)\cap E=\emptyset\implies B_r(x)\subseteq E^\com$. Thus $x$ is an interior point of $E^\com$, completing the proof.
		\end{proof}
		\begin{rem}
		Non-closed sets are not compact in any metric space. Notice open set does not mean non-closed.
		\end{rem}
		
		\begin{thm}
		Closed subsets of compact sets are compact.
		\end{thm}
		\begin{proof}
		See worksheet 7.
		\end{proof}
		\begin{cor}
		If $\{K_\alpha\}_{\alpha\in A}$ is a collection of compact sets, then $\bigcap_{\alpha\in\mathcal{A}}K_\alpha$ is compact.
		\end{cor}
		\begin{proof}
		Since compact sets are closed, $\bigcap_{\alpha\in\mathcal{A}}K_\alpha$ is the intersection of closed sets, which is also closed. Since $\bigcap_{\alpha\in\mathcal{A}}K_\alpha$ is a subset of compact sets $U_\alpha$, it is compact.
		\end{proof}
		\begin{rem}
		Finite union of compact sets in $X$ is compact.
		\end{rem}
		
		\begin{thm}
		Every sequence in a compact set has a convergent subsequence.
		\end{thm}
		\begin{proof}
			See worksheet 7.
		\end{proof}
		
		\begin{thm}[Compact Set]
		A set $E\subseteq X$ is compact if and only if every sequence in $E$ has a convergent subsequence converging to a point in $E$.
		\end{thm}
		
		\begin{thm}[Nested Compact Sets Property]
		Let $(F_n)$ be a sequence of closed, bounded, nonempty sets in $\R^k$ such that $F_1\supseteq F_2\supseteq\cdots$, then $F=\bigcap_{n=1}^{\infty}F_n\neq\emptyset$ and $F$ is closed and bounded.
		\end{thm}
		
		\begin{thm}
		Suppose $\{E_\alpha\}_{\alpha\in\mathcal{A}}$ is a collection of compact sets such that $\bigcap_{\alpha\in\mathcal{B}}E_\alpha\neq\emptyset$ for any finite $\mathcal{B}\subseteq \mathcal{A}$. Then $\bigcap_{\alpha\in\mathcal{A}}E_\alpha\neq\emptyset$.
		\end{thm}
		
		\begin{dfn}[K-cell]
		A K-cell is a subset of $\R^k$ of the form $[a_1,b_1]\times[a_2,b_2]\times\cdots\times[a_k,b_k]$.
		\end{dfn}
		
		\begin{thm}
		Every k-cell $F$ in $\mathbb{R}^k$ is compact.
		\end{thm}
		\begin{proof}
		TODO
		\end{proof}
			
		\begin{thm}
		A subset $E$ of $\mathbb{R}^k$ is compact if and only if it is closed and bounded.
		\end{thm}
		\begin{proof}
			TODO
		\end{proof}
		\begin{rem}
			The forward direction is true in any metric space.
		\end{rem}
		
		\begin{tcolorbox}[title=\textbf{Characterization of compact sets}]
		(1) and (2) are equivalent in any metric space. Forward direction of (3) is true in any metric space. All of three are equivalent in $\R^k$.
		\begin{enumerate}
			\item Every open cover of $E$ has a finite subcover.
			\item A set $E\subseteq X$ is compact if and only if every sequence in $E$ has a convergent subsequence converging to a point in $E$.
			\item A subset $E$ of $\mathbb{R}^k$ is compact if and only if it is closed and bounded.
		\end{enumerate}
		\end{tcolorbox}
		\lline
		
		\subsection*{Cantor Set}
		\begin{dfn}[Cantor Set]
		Let $\mathcal{C}_0$ be $[0,1]$. Then define $\mathcal{C}_1$ as the union of $2^1$ interval $[0,\frac{1}{3}]\cup[\frac{2}{3},1]$. Each time delete the middle $\frac{1}{3}$ of intervals. Thus $\mathcal{C}_2$ is the union of $2^2$ intervals which is $[0,\frac{1}{9}]\cup[\frac{2}{9},\frac{1}{3}]\cup[\frac{2}{3},\frac{7}{9}]\cup[\frac{8}{9},1]$.\smallskip
		
		In short, $C_n$ is the union of $2^n$ disjoint closed intervals of which length is $(\frac{1}{3})^n$. Then define Cantor Set
		\begin{displaymath}
			\mathcal{C}=\bigcap_{i=0}^{\infty}\mathcal{C}_i.
		\end{displaymath}
		\end{dfn}
		\begin{thm}
		Here are some facts/properties about the Cantor set $\mathcal{C}$:
		\begin{itemize}
			\item $\mathcal{C}$ is compact.
			\item $\mathcal{C}$ does not contain any intervals.
			\item $\mathcal{C}$ does not have any interior points.
			\item Every point in $\mathcal{C}$ is a limit point of $\mathcal{C}$.
			\item Every point in $\mathcal{C}$ is a limit point of $\mathcal{C}^\com$.
		\end{itemize}
		\end{thm}
		\newpage	
			
	\section{Series}
		For an infinite series $\sum_{n=m}^{\infty}a_n$, we say it \emph{converge} provided the sequence $(s_n)$ of 
		partial sums
		\begin{displaymath}
			s_n=a_m+a_{m+1}+\cdots+a_n=\sum_{k=m}^{n}a_k
		\end{displaymath}
		also converges to a real number $S$. i.e.
		\begin{displaymath}
			\sum_{n=m}^{\infty}a_n=S\quad\text{means}\quad\lim s_n=S\quad\text{or}\quad\lim\limits_{n\rightarrow\infty}\left(\sum_{k=m}^{n}a_k\right)=S
		\end{displaymath}
		A series that does not converge is said to \emph{diverge}, so $\sum_{n=m}^{\infty}a_n$ \emph{diverge} to $+\infty$
		, $\sum_{n=m}^{\infty}a_n=+\infty$, provided $\lim s_n=+\infty$. Similar for diverging to $-\infty$.
		
		If the terms in $\sum a_n$ are all nonnegative, then the corresponding partial sums $(s_n)$ form an increasing sequence, so $\sum a_n$ either converges or diverges to $+\infty$ by \ref{def:bounded monotone seq} and \ref{def:unbounded monotone seq}. In particular, $\sum |a_n|$ is meaningful for any $(s_n)$ whatever. The series
		$\sum a_n$ is said to \emph{converge absolutely} or to be \emph{absolutely convergent} if $\sum |a_n|$ converges.
		
		We use $\sum a_n$ to represent $\sum_{n=m}^{\infty}a_n$
		
		\begin{eg}[Geometric Series]
		A series of the form $\sum_{n=0}^{\infty}ar^n$ for constants $a$ and $r$ is called a geometric series. For $r\neq1$,
		the partial sums $s_n$ are given by
		\begin{displaymath}
			\sum_{k=0}^{n}ar^k=a\frac{1-r^{n+1}}{1-r}.
		\end{displaymath}
		Furthermore, if $|r|<1$, then $\lim_{n\rightarrow \infty}r^{n+1}=0$ and
		\begin{displaymath}
			\sum_{n=0}^{\infty}ar^n=\frac{a}{1-r}
		\end{displaymath}
		If $a\neq0$ and $|r|\geq1$, then $(ar^n)$ does not converge to $0$, so $\sum ar^n$ diverges.
		\end{eg}
		
		\begin{eg}
		\begin{displaymath}
			\sum_{n=1}^{\infty}\frac{1}{n^p}\quad \text{converges if and only if}\quad p>1
		\end{displaymath}
		If $p\leq1$, $\sum1/n^p=+\infty$
		\end{eg}
		
		\setcounter{equation}{0}
		\begin{dfn}\label{def:cauchy criterion}
		We say a series $\sum a_n$ satisfies the \emph{Cauchy criterion} if its sequence $(s_n)$ of partial sums is a
		Cauchy sequence which is:
		\begin{equation}
			\forall\epsilon>0,\ \exists N,\ m,n>N\Rightarrow|s_n-s_m|<\epsilon
		\end{equation}
		which is equivalent to
		\begin{equation}
			\forall\epsilon>0,\ \exists N,\ n\geq m>N\Rightarrow|s_n-s_{m-1}|<\epsilon.
		\end{equation}
		Since $s_n-s_{m-1}=\sum_{k=m}^{n}a_k$, we can write (2) as
		\begin{equation}
			\forall\epsilon>0,\ \exists N,\ n\geq m>N\Rightarrow\left|\sum_{k=m}^{n}a_k\right|<\epsilon
		\end{equation}
		\end{dfn}
		
		\begin{thm}
		A series converges $\iff$ it satisfies the Cauchy criterion. 
		\end{thm}
		\begin{proof}
		By \ref{def:cauchy iff convergent}, we know its partial sum converges, so the series also converges.
		\end{proof}
		\begin{cor}\label{def:If a_n converges then lim(a_n)=0}
		If a series $\sum a_n$ converges, then $\lim a_n=0$
		\end{cor}
		\begin{proof}
		By setting $n=m$ in the condition of \ref{def:cauchy criterion}, we get
		\begin{displaymath}
			\left(\forall\epsilon>0,\ \exists N,\ n>N\Rightarrow\left|a_n\right|<\epsilon\right)\Rightarrow\lim a_n=0
		\end{displaymath}
		\end{proof}
		\begin{rem}
			If $\lim a_n\neq0$, then $\sum a_n$ does not converge.
		\end{rem}
		A useful contrapositive of this corollary is "If $\lim a_n\neq0$, then $\sum a_n$ does not converge."
		
		\setcounter{equation}{0}
		\begin{thm}[Comparison Test]\label{def:Comparison Test}
		Let $\sum a_n$ be a series where $a_n\geq0$ for all $n$.
		\begin{enumerate}[(i)]
			\item If $\sum a_n$ converges and $|b_n|\leq a_n$ for all $n$, then $\sum b_n$ converges.
			\item If $\sum a_n=+\infty$ and $b_n\geq a_n$ for all $n$, then $\sum b_n=+\infty$
		\end{enumerate}
		\end{thm}
		\begin{proof}
		\begin{enumerate}[(i)]
			\item[]
			\item For $n\geq m$ we have
			\begin{equation}
				\left|\sum_{k=m}^{n}b_k\right|\leq\sum_{k=m}^{n}|b_k|\leq\sum_{k=m}^{n}a_k
			\end{equation} 
			Since $\sum a_n$ converges, it satisfies \ref{def:cauchy criterion}(1). Then from (1)  we can see $\sum b_n$
			also satisfies the Cauchy criterion in \ref{def:cauchy criterion}(3), and hence $\sum b_n$ converges.
			\item Since $b_n\geq a_n$ for all $n$, obviously we have $\sum_{k=m}^{n}b_k\geq\sum_{k=m}^{n}a_k$. Since $\lim\sum_{k=m}^{n}b_k=+\infty$, $\lim\sum_{k=m}^{n}a_k=+\infty$.			
		\end{enumerate}
		\end{proof}
		\begin{cor}
		Absolutely convergent series are convergent.
		\end{cor}
		\begin{proof}
		Suppose $\sum b_n$ is absolutely convergent. This means $\sum a_n$ converges where $a_n=|b_n|$ for all $n$. Then
		$|b_n|\leq a_n$ and $\sum b_n$ converges trivially from \ref{def:Comparison Test}.
		\end{proof}
		
		\begin{thm}[Root Test]\label{def:Root Test}
		Let	$\sum a_n$ be a series and let $\alpha=\lim\sup|a_n|^{1/n}$. The series $\sum a_n$
		\begin{enumerate}[(i)]
			\item converges absolutely if $\alpha<1$
			\item diverges if $\alpha>1$
			\item Otherwise the test does not provide any useful information.
		\end{enumerate}
		\end{thm}
		\begin{proof}
		\begin{enumerate}[(i)]
			\item Suppose $\alpha<1$, and select $\epsilon>0$ so that $\alpha+\epsilon<1$. Then
			\begin{displaymath}
				\alpha-\epsilon<\sup\{|a_n|^{1/n}: n>N\}<\alpha+\epsilon
			\end{displaymath}
			so
			\begin{displaymath}
				|a_n|<(a+\epsilon)^n\quad\text{for}\quad n>N.
			\end{displaymath}
			Since $0<\alpha+\epsilon<1$, $\sum_{n=N+1}^{\infty}(\alpha+\epsilon)^n$ converges and \ref{def:Comparison Test}(i) tells $\sum_{n=N+1}^{\infty}a_n$ converges. Then clearly $\sum a_n$ converges.
			\item If $\alpha>1$, then for each $\alpha-\epsilon>1$, there exists $N\in\N$ such that $\sup\{|a_n|^\frac{1}{n}: n\geq N\}>1$, i.e., $\exists N_1\geq N$ such that $|a_{N_1}|^\frac{1}{N_1}|>1\implies |a_{N_1}|>1$ since $1$ is smaller than the supremum. This means $|a_n|>1$ for infinitely many choices of $n$. In particular, $(a_n)$
			cannot possibly converge to $0$, so $\sum a_n$ cannot converge by the contrapositive of \ref{def:If a_n converges then lim(a_n)=0}.
			\item Example: $\sum\frac{1}{n}$ diverges but $\sum\frac{1}{n^2}$ converges.
		\end{enumerate}
		\end{proof}
	
		\begin{thm}[Ratio Test]\label{def:Ratio Test}
 		A series $\sum a_n$ of nonzero terms
 		\begin{enumerate}[(i)]
 			\item converges absolutely if $\lim\sup\left|\frac{a_{n+1}}{a_n}\right|<1$,
 			\item diverges if $\lim\inf\left|\frac{a_{n+1}}{a_n}\right|>1$.
 			\item Otherwise $\lim\inf\left|\frac{a_{n+1}}{a_n}\right|\leq1\leq\lim\sup\left|\frac{a_{n+1}}{a_n}\right|$ and
 			the test gives no information.
 		\end{enumerate}
	 	\end{thm}
	 	\begin{proof}
	 	let $\alpha=\lim\sup|a_n|^{1/n}$. By \ref{def:12.2} we have
	 	\begin{displaymath}
	 		\lim\inf\left|\frac{a_{n+1}}{a_n}\right|\leq\alpha\leq\lim\sup\left|\frac{a_{n+1}}{a_n}\right|.
	 	\end{displaymath}
 		\begin{enumerate}[(i)]
 			\item If $\lim\sup\left|\frac{a_{n+1}}{a_n}\right|<1$, then $\alpha<1$ and the series converges by \ref{def:Root Test}.
 			\item If $\lim\inf\left|\frac{a_{n+1}}{a_n}\right|>1$, then $\alpha>1$ and the series diverges by \ref{def:Root Test}.
 			\item If $\alpha=1$, then same reasoning as the proof in \ref{def:Root Test}(iii).
 		\end{enumerate}
	 	\end{proof}
 		\bigskip
		If the terms $a^n$ are nonzero and if $\lim\left|\frac{a_{n+1}}{a_n}\right|=1$, then $\alpha=\lim\sup|a_n|^{1/n}=1$
		by \ref{def:12.3}, so neither the Ratio Test nor the Root Test gives information about the convergence of $\sum a_n$.
		\newpage
		
	\section{Alternating Series and Integral Tests}
		Sometimes we can try to check convergence or divergence of series by comparing the partial sums with familiar
		integrals. By drawing the function $a^n$ and the of rectangles corresponding to the series on a same picture and comparing the areas under the function and the sum of areas of these rectangles, we may get the information about the convergence of the series. For example, if all rectangles are below the function and the integral of the function is finite, then the series converge.
		\begin{thm}
		$\sum\frac{1}{n^p}$ converges $\iff$ $p>1$.
		\end{thm}
		\begin{proof}
		By drawing the function $\frac{1}{n^p}$ and the of rectangles corresponding to the series on a same picture, we can
		get
		\begin{displaymath}
			\sum_{k=1}^{n}\frac{1}{k^p}\leq1+\int_{1}^{n}\frac{1}{x^p}dx=1+\frac{1}{p-1}\left(1-\frac{1}{n^{p-1}}\right)<1+\frac{1}{p-1}=\frac{p}{p-1}
		\end{displaymath}
		Thus $\sum_{n=1}^{\infty}\frac{1}{n^p}\leq\frac{p}{p-1}<+\infty$
		
		Suppose $0<p\leq1$. Then $\frac{1}{n^p}\geq\frac{1}{n}$ for all n, so $\sum\frac{1}{n^p}$ diverges when $\sum\frac{1}{n}$ diverges by \ref{def:Comparison Test}.
		\end{proof}
		
		\begin{thm}
		Here are the conditions under which an integral test is advisable:
		\begin{enumerate}[(a)]
			\item All comparison, root, and ratio tests do not apply.
			\item The terms $a_n$ of the series are nonnegative.
			\item There is a nice decreasing function $f$ on $[1,\infty)$ such that $f(n)=a_n$ for all $n$.
			\item The integral of $f$ is easy to calculate or estimate.
		\end{enumerate}
		If $\lim_{n\rightarrow\infty}\int_{1}^{n}f(x)dx=+\infty$, then the series diverges. If $\lim_{n\rightarrow\infty}\int_{1}^{n}f(x)dx<+\infty$, then the series will converge.
		\end{thm}
		
		\begin{thm}[Alternating Series Theorem]\label{def:alternating series}
		If $a_1\geq a_2\geq\cdots\geq a_n\geq\cdots\geq0$ and $\lim a_n=0$, then the alternating series $\sum(-1)^{n+1}a_n$
		converges. Moreover, the partial sums $s_n=\sum_{k=1}^{n}(-1)^{k+1}a_k$ satisfy $|s-s_n|\leq a_n$ for all $n$.
		\end{thm}
		\begin{proof}
		To prove the series converge we need to show the partial sum $(s_n)$ also converges. Note that the subsequence
		$(s_{2n})$ is increasing (accumulative sum of positive $a_n$) and the subsequence $(s_{2n-1})$ is decreasing (accumulative sum of negative $a_n$). We claim
		\begin{equation}
			s_{2m}\leq s_{2n+1}\quad\text{for all}\quad m,n\in\mathbb{N}
		\end{equation}		
		Since $s_{2n+1}-s_{2n}=a_{2n+1}\geq0$, we have $s_{2n}\leq s_{2n+1}$ for all $n$. Thus if $m\leq n$ in (1) then (1)
		holds because $s_{2m}\leq s_{2n}\leq s_{2n+1}$, when $(s_{2n})$ is increasing. If $m\geq n$ in (1), then (1) also
		holds because $s_{2n+1}\geq s_{2m+1}\geq s_{2m}$ when $(s_{2n+1})$ is decreasing. Therefore, by (1) we can see that
		the subsequence $(s_{2n})$ is bounded above by every odd partial sum, and the subsequence $(s_{2n+1})$ is a 
		bounded below by each even partial sum. Then by \ref{def:bounded monotone seq} $(s_{2n})$ and $(s_{2n+1})$ converge
		to some $s$ and $t$. Now we have
		\begin{displaymath}
			t-s=\lim\limits_{n\rightarrow\infty}(s_{2n+1}-2_{2n})=\lim\limits_{n\rightarrow\infty}a_{2n+1}=0
		\end{displaymath}
		so $s=t$ and $\lim_ns_n=s$. (\textbf{WHY???} Is it because $s=sup S$ and $t=\inf S$ where $S$ is the set of subsequential limits.)
		
		To check the last claim, note that $s_{2k}\leq s\leq s_{2k+1}$, so both $s_{2k+1}-s$ and $s-s_{2k}$ are bounded by
		$s_{2k+1}-s_{2k}=a_{2k+1}\leq a_{2k}$ (\textbf{WHY????}). So whether $n$ is even or odd, we have $|s-s_n|\leq a_n$. 
		\end{proof}
	
	\chapter{Continuity}
		\newpage
		\section{Continuous Functions}
			In this book/note, we will be concerned with functions $f$ such that $\dom(f)\subseteq\R$ and such that $f$ is
			a real-valued function. We consider the \emph{natural domain} as "the largest subset of $\R$ on which the function is a well defined real-valued function. 
			
			\begin{dfn}\label{def:continuous}
			The function $f$ is \emph{continuous} at $x_0$ in $\dom(f)$ if, for every sequence $(x_n)$ in $\dom(f)$ converging to $x_0$, we have $\lim_nf(x_n)=f(x_0)$. If $f$ is continuous at each point of a set $S\subseteq\dom(f)$, then $f$ is said to be \emph{continuous} on $S$. The function $f$ is said to be
			\emph{continuous} if it is continuous on $\dom(f)$.
			\end{dfn}
		
			\setcounter{equation}{0}
			\begin{thm}\label{def:delta-epsilon property}
				$f$ is continuous at $x_0$ in $\dom(f)$ if and only if
				\begin{equation}
					\forall\epsilon>0,\ \exists\delta>0\quad\text{such that}\quad(x\in\dom(f))\land(|x-x_0|<\delta)\implies|f(x)-f(x_0)|<\epsilon
				\end{equation}
			\end{thm}
			\begin{proof}
			\begin{itemize}
				\item[]
				\item[$\implies$]: Suppose $f$ is continuous at $x_0$ but (1) does not hold. In other words, there exists
				$\epsilon>0$ so that
				\begin{displaymath}
					(x\in\dom(f))\land(|x-x_0|<\delta)\implies|f(x)-f(x_0)|<\epsilon
				\end{displaymath}
				fails for each $\delta>0$. In particular the implication
				\begin{displaymath}
					(x\in\dom(f))\land(|x-x_0|<\frac{1}{n})\implies|f(x)-f(x_0)|<\epsilon
				\end{displaymath}
				fails for each $n\in\N$. Thus for each $n\in\N$ there exists $x_n$ in $\dom(f)$ such that $|x_n-x_0|<\frac{1}{n}$ and yet $|f(x_0)-f(x_n)|\geq\epsilon$. Hence we have $|f(x_0)-f(x_n)|\geq\epsilon\implies\lim f(x_n)\neq f(x_0)$. This contradicts to the definition of continuity \ref{def:continuous}.								
				\item[$\impliedby$]: Suppose (1) holds and consider a $(x_n)$ in $\dom(f)$ such that $\lim x_n=x_0$. Let
				$\epsilon>0$. By (1) there exists $\delta>0$ such that
				\begin{displaymath}
					(x\in\dom(f))\land(|x-x_0|<\delta)\implies|f(x)-f(x_0)|<\epsilon
				\end{displaymath}
				Since $\lim x_n=x_0$ we have
				\begin{displaymath}
					n>N\implies|x_n-x_0|<\delta\implies|f(x_n)-f(x_0)|<\epsilon
				\end{displaymath}
				Thus $\lim f(x_n)=f(x_0)$
			\end{itemize}
			\end{proof}
			The condition $(x\in\dom(f))\land(|x-x_0|<\delta)\implies|f(x)-f(x_0)|<\epsilon$ in the book is a little bit
			confusing. In other words, it means
			\begin{displaymath}
				\forall x\in\dom(f),\ |x-x_0|<\delta\implies|f(x)-f(x_0)|<\epsilon
			\end{displaymath}
			
			To use $\epsilon\text{-}\delta$ property to prove the discontinuity, we need to show that
			\begin{displaymath}
				\exists\epsilon>0\quad\text{such that}\quad\forall\delta>0,\ \exists x\in\dom(f)\quad\text{such that}\quad |x-x_0|<\delta\quad\text{but}\quad|f(x)-f(x_0)|\geq\epsilon
			\end{displaymath}
			
			
			\begin{thm}\label{def:17.3}
			If $f$ is continuous at $x_0$ in $\dom(f)$, then $|f|$ and $kf$, for $k\in\R$, are continuous at $x_0$.
			\end{thm}
			\begin{proof}
			Since $f$ is continuous at $x_0$, we have $\lim f(x_n)=f(x_0)$. Since $\lim kf(x_n)=k\lim f(x_n)=kf(x_0)$, this
			proves $kf$ is continuous at $x_0$.
			Since $\lim f(x_n)=f(x_0)$, we have 
			\begin{displaymath}
				n>N\implies|f(x_n)-f(x_0)|<\epsilon
			\end{displaymath}
			Since $\abs{\abs{f(x_n)}-\abs{f(x_0)}}\leq\abs{f(x_n)-f(x_0)}$, we have 
			\begin{displaymath}
				n>N\implies\big|\abs{f(x_n)}-\abs{f(x_0)}\big|\leq\abs{f(x_n)-f(x_0)}<\epsilon
			\end{displaymath}
			so $\lim \abs{f(x_n)}=\abs{f(x_0)}$
			\end{proof}
			
			\begin{thm}\label{def:17.4}
			Let $f$ and $g$ be real-valued fucntions that are continuous at $x_0$ in $\R$. Then
			\begin{enumerate}[(i)]
				\item $f+g$ is continuous at $x_0$;
				\item $fg$ is continuous at $x_0$;
				\item $f/g$ is continuous ar $x_0$ if $g(x_0)\neq0$.
			\end{enumerate}
			\end{thm}
			\begin{proof}
			We use the basic definition of continuity \ref{def:continuous} and the basic theorems of limit.
			\end{proof}
			
			\begin{thm}\label{def:17.5}
			If $f$ is continuous at $x_0$ and $g$ is continuous at $f(x_0)$, then the composite function $g\circ f$ is
			continuous at $x_0$.
			\end{thm}
			\begin{proof}
			Given that $x_0\in\dom(f)$ and $f(x_0)\in\dom(g)$, let $(x_n)$ be a sequence in $\{x\in\dom(f): f(x)\in\dom(g)\}$
			converging to $x_0$. Since $f$ is continuous at $x_0$, we have $\lim f(x_n)=f(x_0)$. Since the sequence $(f(x_n))$ converges to $f(x_0)$ and $g$ is continuous at $f(x_0)$, we also have $\lim g(f(x_n))=g(f(x_0))$ which
			is $\lim g\circ f(x_n)=g\circ f(x_0)$.
			\end{proof}
			\newpage
		\section{Properties of Continuous Functions}
			A real-valued function $f$ is said to be \emph{bounded} if $\{f(x): x\in\dom(f)\}$ is a bounded set. i.e. if
			there exists a real number $M$ such that $|f(x)|\leq M$ for all $x\in\dom(f)$.
			
			\begin{thm}\label{def:18.1}
			Let $f$ be a continuous real-valued function on a \emph{closed} interval $[a,b]$. Then $f$ is a bounded function.
			Moreover, $f$ assume its maximum and minimum values on $[a,b]$; that is there exist $x_0,y_0$ in $[a,b]$
			such that $f(x_0)\leq f(x)\leq f(y_0)$ for all $x\in[a,b]$.
			\end{thm}
			\begin{proof}
			First assume $f$ is not bounded on $[a,b]$. Then for each $n\in\N$ there corresponds an $x_n\in[a,b]$ such that
			$|f(x_n)|>n \implies \lim_{k\rightarrow\infty}|f(x_{n_k})|=+\infty$. By \ref{def:B-W}, since $(x_n)$ is bounded by $[a,b]$ it has a subsequence $(x_{n_k})$ that
			converges to some real number $x_0\in[a,b]$. Since $f$ is continuous, we have $\lim f(x_{n_k})=f(x_0)<\infty$,
			which is a contradiction. Thus, $f$ is bounded.
			
			Not since $f$ is bounded, $M=\sup\{f(x): x\in[a,b]\}$ is finite. For each $n\in\N$ there exists $y_n\in[a,b]$
			such that $M-\frac{1}{n}<f(y_n)\leq M$. Hence we have $\lim f(y_n)=M$ by Squezze formula. By \ref{def:B-W} there
			is a subsequence $(y_{n_k})$ of $(y_n)$ converging to some limit $y_0\in[a,b]$. Since $y$ is continuous at $y_0$, we have $\lim_{k\rightarrow\infty}f(y_{n_k})=f(y_0)$. Since $(f(y_{n_k}))$ is also a subsequence
			of $(f(y_n))$, by \ref{def:subsequence converges to the same limit} $\lim_{k\rightarrow\infty}f(y_{n_k})=
			\lim_{n\rightarrow\infty}f(y_n)=M$. Thus $f(y_0)=M$ meaning that $f$ achieves its maximum at $y_0$.
			
			Apply the same method to $-f$, and we get $-f$ achieves its maximum at some $x_0\in[a,b]$. In other words, $f$
			achieves its minimum at $x_0$.
			\end{proof}
			
			\begin{thm}[Intermediate Value Theorem]\label{def:18.2}
			If $f$ is a continuous real-valued function on an interval $I$, then $f$ has the intermediate value property on I: Whenever $a,b\in I$, if $a<b$ and $y$ lies between $f(a)$ and $f(b)$ [i.e. $f(a)<y<f(b)$ or $f(b)<y<f(a)$], then there exists at least one $x$ in
			$(a,b)$ such that $f(x)=y$.
			\end{thm}
			\begin{proof}
			Let's focus on the case that $f(a)<y<f(b)$ since the other case is similar. Let $S=\{x\in[a,b]: f(x)<y\}$. Since
			$a\in S$ the set $S$ is nonempty, and $x_0=\sup S$ represents a number in $[a,b]$. For each $n\in\N$, $x_0-\frac{1}{n}$ is not an upper bound for $S$, so there exists $s_n\in S$ such that $x_0-\frac{1}{n}<s_n\leq x_0$. Thus $\lim s_n=x_0$ and since $f(s_n)<y$ for all $n$, we have 
			\begin{displaymath}
				f(x_0)=\lim f(s_n)\leq y
			\end{displaymath} 
			because $f$ is continuous at $x_0$. Let $t_n=\min\{b, x_0+\frac{1}{n}\}$. Since $x_0<t_n\leq x_0+\frac{1}{n}$ we have $\lim t_n=x_0$. Each $t_n$ belongs to $[a,b]$ but not to $S$, so $f(t_n)\geq y$ for all $n$. Therefore,
			\begin{displaymath}
				f(x_0)=\lim f(t_n)\geq y
			\end{displaymath}
			because $f$ is continuous at $x_0$. Thus $f(x_0)=y$. 
			\end{proof}
			
			\begin{cor}\label{def:18.3}
			If $f$ is a continuous real-valued function on an interval $I$, then the set $f(I)=\{f(x): x\in I\}$ is also
			an interval or a single point.
			\end{cor}
			\begin{proof}
			By \ref{def:18.2} the set $J=f(I)$ has the property:
			\begin{displaymath}
				(y_0,y_1\in J\quad\text{and}\quad y_0<y<y_1)\implies y\in J
			\end{displaymath}
			If $\inf J<\sup J$, then such a set $J$ will be an interval. Consider $\inf J<y<\sup J$. Then there exist $y_0,y_1\in J$ so that $y_0<y<y_1$, so $y\in J$ by the above property. We showed that
			\begin{displaymath}
				\inf J<y<\sup J \implies y\in J
			\end{displaymath}
			so $J$ is an interval with endpoints $\inf J$ and $\sup J$
			\end{proof}
					
			\begin{thm}\label{def:18.5}
			Let $g$ be a strictly increasing function on an interval $J$ such that $g(J)$ is an interval $I$. Then $g$ is
			continuous on $J$.
			\end{thm}
			\begin{proof}
			Consider an non-endpoint $x_0$ of $J$. Since $g$ is strictly increasing, $g(x_0)$ is also not an endpoint of $I$, so $\exists \epsilon_0>0$ such that $(g(x_0)-\epsilon_0,g(x_0)+\epsilon_0)\subseteq I$.
			
			Let $\epsilon>0$ and we can assume $\epsilon<\epsilon_0$ (\textbf{WHY???}). Then $\exists x_1,x_1\in J$ such that $g(x_1)=g(x_0)-\epsilon$ and $g(x_2)=g(x_0)+\epsilon$. This means $x_1<x_0<x_2$ because $g$ is increasing.
			Also if $x_1<x<x_2$, then $g(x_1)<g(x)<g(x_2)$, hence $g(x_0)-\epsilon<g(x)<g(x_0)+\epsilon$, and hence
			$|g(x)-g(0)|<\epsilon$. Now set $\delta=\min\{x_2-x_0,x_0-x_1\}$, then 
			\begin{displaymath}
				|x-x_0|<\delta\implies x_1<x<x_2\implies|g(x)-g(x_0)|<\epsilon
			\end{displaymath}
			Thus $g$ is continuous on $J$.
			\end{proof}
		
			\begin{thm}\label{def:18.4}
			Let $f$ be a continuous strictly increasing function on some interval $I$. Then $f(I)$ is an interval $J$ by \ref{def:18.3} and $f^{-1}$ represents a function with domain $J$. The function $f^{-1}$ is a continuous strictly increasing function on $J$.
			\end{thm}
			\begin{proof}
			Obviously $f^{-1}$ is still strictly increasing. Since $f^{-1}$ maps $J$ onto $I$, by \ref{def:18.5} $f^{-1}$ is continuous.
			\end{proof}
			
			\setcounter{equation}{0}
			\begin{thm}
			Let $f$ be a \emph{one-to-one} continuous function on an interval $I$. Then $f$ is strictly increasing or strictly decreasing.
			\end{thm}
			\begin{proof}
			Firstly we want to show that
			\begin{equation}
				\text{if $a<b<c$ in $I$, then $f(b)$ lies between $f(a)$ and $f(c)$}
			\end{equation}
			Assume it's false so $f(b)>\max\{f(a),f(c)\}$. Select $y$ so that $f(b)>y>\max\{f(a),f(c)\}$. By \ref{def:18.2} applied to $[a,b]$ and $[b,c]$, $\exists x_1\in(a,b)$ and $x_2\in(b,c)$ such that $f(x_1)=f(x_2)=y$. This contradicts to the one-to-one property of $f$. 
			
			Now select any $a_0<b_0$ in $I$ and suppose, say, that $f(a_0)<f(b_0)$. We need to show $f$ is strictly increasing on $I$. By (1) we have
			\begin{align*}
				f(x)<f(a_0)\quad&\text{for}\quad x<a_0\\
				f(a_0)<f(x)<f(b_0)\quad&\text{for}\quad a_0<x<b_0\\
				f(b_0)<f(x)\quad&\text{for}\quad x>b_0
			\end{align*}
			In particular,
			\begin{align}
				f(x)<f(a_0)\quad&\text{for all}\quad x<a_0\\
				f(a_0)<f(x)\quad&\text{for all}\quad x>a_0
			\end{align}
			Now consider any $x_1<x_2$ in $I$.
			\begin{align*}
				x_1\leq a_0\leq x_2 &\implies f(x_1)<f(x_2)\quad\text{by (2) and (3)}\\
				x_1<x_2<a_0&\implies f(x_1)<f(a_0)\quad\text{by (2)} \implies f(x_1)<f(x_2)\quad\text{by (1)}\\
				a_0<x_1<x_2&\implies f(a_0)<f(x_2)\quad\text{by (2)} \implies f(x_1)<f(x_2)\quad\text{by (1)}
			\end{align*}
			\end{proof}
			\newpage
		\section{Uniform Continuity}
			Sometimes we want to know when the $\delta$ in \ref{def:delta-epsilon property} can be chosen to depend only on
			$\epsilon>0$ and $S$, so that $\delta$ does not depend on the particular point $x_0$.
			\begin{dfn}
			Let $f$ be a real-valued function defined on a set $S\subseteq\R$. Then $f$ is uniformly ocntinuous on $S$ if
			\begin{displaymath}
				\text{for each $\epsilon>0$ there exists $\delta>0$ such that $\forall x,y\in S,\ |x-y|<\delta\implies|f(x)-f(y)|<\epsilon.$}
			\end{displaymath}
			We will say $f$ is uniformly continuous if $f$ is uniformly continuous on $\dom f$.
			\end{dfn}
			
			\begin{thm}\label{def:19.2}
			If $f$ is continuous on a closed interval $[a,b]$, then $f$ is uniformly continuous on $[a,b]$.
			\end{thm}
			\begin{proof}
			Assume $f$ is not uniformly continuous on $[a,b]$, then there exists $\epsilon>0$ such that for each $\delta>0$,
			there exists $x,y\in[a,b]$ such that $|x-y|<\delta$ but $|f(x)-f(y)|\geq\epsilon$. Thus for each $n\in\N$, since
			$\frac{1}{n}>0$, there exist $x_n,y_n\in[a,b]$ such that $|x_n-y_n|<\frac{1}{n}$ but $|f(x_n)\geq f(y_n)|\geq\epsilon$. By \ref{def:B-W}, since $(x_n)$ is bounded in $[a,b]$, it has a subsequence $(x_{n_k})$
			converging to $x_0\in[a,b]$. Clearly we can also have a subsequence $(y_{n_k})$ converging to $x_0$. Because $f$
			is continuous at $x_0$, we have 
			\begin{displaymath}
				f(x_0)=\lim\limits_{x\rightarrow\infty}f(x_{n_k})=\lim_{k\rightarrow\infty}f(y_{n_k})
			\end{displaymath}
			so
			\begin{displaymath}
				\lim_{k\rightarrow\infty}[f(x_{n_k})-f(y_{n_k})]=0
			\end{displaymath} 
			However, since $|f(x_{n_k})-f(y_{n_k})|\geq\epsilon>0$ for all $k$, we have a contradiction.
			\end{proof}
			
			\begin{thm}\label{def:19.4}
			If $f$ is uniformly continuous on a set $S$ and $(s_n)$ is a Cauchy sequence in $S$, then $(f(s_n))$ is a 
			Cauchy sequence.
			\end{thm}
			\begin{proof}
			Let $(s_n)$ be a Cauchy sequence in $S$ and let $\epsilon>0$. Since $f$ is uniformly continuous on $S$, there
			exists $\delta>0$ so that
			\begin{displaymath}
				\forall x,y\in S,\ |x-y|<\delta\implies|f(x)-f(y)|<\epsilon.
			\end{displaymath}
			Since $(s_n)$ is a Cauchy sequence, there exists $N$ so that
			\begin{displaymath}
				m,n>N\implies|s_n-s_m|<\delta\implies|f(s_n)-f(s_m)|<\epsilon.
			\end{displaymath}
			Thus $(f(s_n))$ is indeed a Cauchy sequence.
			\end{proof}
			The contrapositive of the above them is that "If $(s_n)$ is a Cauchy sequence in $S$, but $(f(s_n))$ is not a Cauchy sequence, then $f$ is not uniformly continuous on a set $S$.
			
			\setcounter{equation}{0}
			\begin{thm}\label{def:19.5}
			A real-valued function $f$ on $(a,b)$ is uniformly continuous on $(a,b)$ if and only if it can be extended to
			a continuous function $\tilde{f}$ on $[a,b]$.
			\end{thm}
			\begin{proof}
			\begin{itemize}
				\item[]
				\item[$\implies$]: Suppose $(s_n)$ is a sequence in $(a,b)$ converging to $a$, then $(s_n)$ is a Cauchy
				sequence by \ref{def:cauchy iff convergent}, so $(f(s_n))$ is also a Cauchy sequence by \ref{def:19.4} and 
				converging again by \ref{def:cauchy iff convergent}. Hence we have the following claim:
				\begin{equation}
					\text{if $(s_n)$ is a sequence in $(a,b)$ converging to $a$, then $(f(s_n))$ converges}
				\end{equation}
			
				Create a sequence $(u_n)=(s_1, t_1,s_2,t_2,\dots)$ where $(t_n)$ is also a sequence converging to $a$.
				Clearly $\lim u_n=a$ and $\lim f(u_n)$ exists due to (1). Therefore, $(f(s_n))$ and $(f(t_n))$ are 
				subsequences of $(f(u_n))$ both converge to $\lim f(u_n)$ by \ref{def:subsequence converges to the same limit}, so $\lim f(s_n)=\lim f(t_n)$. Hence we have the following claim:
				\begin{equation}
					\text{if $(s_n)$ and $(t_n)$ are sequences in $(a,b)$ converging to $a$, then $\lim f(s_n)=\lim f(t_n)$}
				\end{equation}
				Now we define
				\begin{equation}
					\text{$\tilde{f}(a)=\lim f(s_n)$ for any sequence $(s_n)$ in $(a,b)$ converging to $a$}
				\end{equation}
				(1) guarantees $\lim f(s_n)$ exists, and (2) guarantees $\tilde{f}(a)$ is not ambiguous. Thus, $\tilde{f}$
				is continuous at $a$. Similar method for $\tilde{f}(b)$.
				\item[$\impliedby$]: Since $\tilde{f}$ is continuous on $[a,b]$, $\tilde{f}$ is also uniformly continuous
				on $[a,b]$ by \ref{def:19.2}, so clearly $f$ is uniformly continuous on $(a,b)$.
			\end{itemize}
			\end{proof}
			
			\begin{thm}
			Let $f$ be a continuous function on an interval $I$ [$I$ may be bounded or unbounded]. Let $I^\circ$ be the
			interval obtained by removing from $I$ any endpoints that happen to be in $I$. If $f$ is differentiable on
			$I^\circ$ and if $f^\prime$ is bounded on $I^\circ$, then $f$ is uniformly continuous on $I$.
			\end{thm}
			\newpage
		\section{Limits of Functions}
			\begin{dfn}
			Let $S\subset\R$ and $a\in\R$ or a symbol $\infty$ or $-\infty$ that is the limit of some sequence in $S$, and let $L$ be a real number or symbol $+\infty$ or $-\infty$. We write $\lim_{x\rightarrow a^S} f(x)=L$ if
			\begin{displaymath}
				\text{$f$ is a function defined on $S$,}
			\end{displaymath}
			and
			\begin{displaymath}
				\text{for every sequence $(x_n)$ in $S$ with limit $a$, we have $\lim_{n\rightarrow\infty}f(x_n)=L$}.
			\end{displaymath}
			\end{dfn}
			Recall the definition of continuity, now we can say that a function $f$ is continuous at $a$ in $\dom(f)=S \iff \lim_{x\rightarrow a^S} f(x)=f(a)$. Also notice that when limits exist, they are unique. In other words, there
			is only one $L$ equals to $\lim_{x\rightarrow a^S} f(x)$.  
			
			Now let's define the various standard limit concepts for functions.
			\begin{dfn}
			\begin{enumerate}[(a)]
				\item[]
				\item For $a\in\R$ and a function $f$ we write $\lim_{x\rightarrow a}f(x)=L$ provided $\lim_{x\rightarrow a^S}f(x)=L$ for some set $S=J\backslash\{a\}$ where $J$ is an open interval containing $a$. Such a limit 
				$\lim_{x\rightarrow a^S}$ is called the \emph{[two-sided]} limit of $f$ at $a$. Note that neither $f(a)$ needs to be defined or $\lim_{x\rightarrow a}f(x)$ needs to be equal $f(a)$, unless we want to say $f$ is continuous at $a$.
				\item For $a\in\R$ and a function $f$ we write $\lim_{x\rightarrow a^+}f(x)=L$ provided $\lim_{x\rightarrow a^S}f(x)=L$ for some open interval $S=(a,b)$. This is called the \emph{[right-hand]} limit. Again $f$ need not
				be defined at $a$.
				\item For $a\in\R$ and a function $f$ we write $\lim_{x\rightarrow a^-}f(x)=L$ provided $\lim_{x\rightarrow a^S}f(x)=L$ for some open interval $S=(c,a)$. This is called the \emph{[left-hand]} limit.
				\item For a function $f$ we write $\lim_{x\rightarrow\infty}f(x)=L$ provided $\lim_{x\rightarrow\infty^S}f(x)=L$ for some interval $S=(c,\infty)$. Likewise, For a function $f$ we write $\lim_{x\rightarrow-\infty}f(x)=L$ provided $\lim_{x\rightarrow-\infty^S}f(x)=L$ for some interval $S=(-\infty,b)$
			\end{enumerate} 
			\end{dfn}
			
			\begin{thm}\label{def:20.4}
			Let $f_1$ and $f_2$ be functions for which the limits $L_1=\lim_{x\rightarrow a^S}f_1(x)$ and $L_2=\lim_{x\rightarrow a^S}f_2(x)$ exist and are finite. Then
			\begin{enumerate}[(i)]
				\item $\lim_{x\rightarrow a^S}(f_1+f_2)(x)$ exists and equals $L_1+L_2$;
				\item $\lim_{x\rightarrow a^S}(f_1f_2)(x)$ exists and equals $L_1L_2$;
				\item $\lim_{x\rightarrow a^S}(f_1/f_2)(x)$ exists and equals $L_1/L_2$ provided $L_2\neq0$ and $f_2(x)\neq0$ for $x\in S$.
			\end{enumerate}
			\end{thm}
			\begin{proof}
			By the assumption since both $f_1$ and $f_2$ are defined on $S$ and $a$ is the limit of some sequence in $S$,
			clearly the functions $f_1+f_2$ and $f_1f_2$ are defined on$S$ and so is $f_1/f_2$ if $f_2(x)\neq0$ for $x\in S$.
			
			By the assumption, for every sequence $(x_n)$ in $S$ with limit $a$, we have $L_1=\lim_{n\rightarrow\infty}f_1(x_n)$ and $L_2=\lim_{n\rightarrow\infty}f_2(x_n)$. Now by the basic theorems of the limits, we have
			\begin{displaymath}
				\lim\limits_{n\rightarrow\infty}(f_1+f_2)(x_n)=	\lim\limits_{n\rightarrow\infty}f_1(x_n) + \lim\limits_{n\rightarrow\infty}f_2(x_n) = L_1+L_2
			\end{displaymath}
			and
			\begin{displaymath}
				\lim\limits_{n\rightarrow\infty}(f_1f_2)(x_n)=\left[\lim\limits_{n\rightarrow\infty}f_1(x_n)\right] \cdot \left[\lim\limits_{n\rightarrow\infty}f_2(x_n)\right] = L_1L_2
			\end{displaymath}
		    Similar proof for (iii).
			\end{proof}
			
			\begin{thm}\label{def:20.5}
				Let $f$ be a function for which the limit $L=\lim_{x\rightarrow a^S}f(x)$ exists and is finite. If $g$ is a function defined on $\{f(x): x\in S\}\cup\{L\}$ that is continuous at $L$, then $\lim_{x\rightarrow a^S} g\circ f(x)$ exists and equals $g(L)$.
			\end{thm}
			\begin{proof}
			First note that $g\circ f$ is defined on $S$ by our assumptions. Consider a sequence $(x_n)$ in $S$ with limit
			$a$. Then we have $L=\lim_{n\rightarrow \infty}f(x_n)$. Since $g$ is continuous at $L$, it follows that
			\begin{displaymath}
				g(L)=\lim_{n\rightarrow\infty}g(f(x_n))=\lim_{n\rightarrow\infty}g\circ f(x_n)
			\end{displaymath}
			Hence $\lim_{n\rightarrow a^S}g\circ f(x_n)=g(L)$.
			\end{proof}
			Be careful that for this theorem to work, $g$ needs to be \textbf{continuous} at $L$.
			
			\setcounter{equation}{0}	
			\begin{thm}\label{def:20.6}
			Let $f$ be a function defined on a subset $S$ of $\R$, let a be a real number that is the limit of some sequence
			in $S$, and let $L$ be a real number, then $\lim_{x\rightarrow a^S}f(x)=L$ if and only if
			\begin{equation}
				\text{for each $\epsilon>0$ there exists $\delta>0$ such that $x\in S$ and $|x-a|<\delta$ imply $|f(x)-L|<\epsilon$}
			\end{equation}
			\end{thm}
			\begin{proof}
			\begin{itemize}
				\item[]
				\item[$\implies$]: Suppose $\lim_{n\rightarrow a^S}f(x)=L$ but (1) does not hold. This means there exists
				$\epsilon>0$ such that for each $\delta>0$ and $n\in\N$\textbf{WHY???}, there exists $x_n\in S$ such that $|x_n-a|<\delta$ but $|f(x)-L|\geq\epsilon$. Hence $x_n$ is a sequence in $S$ with limit $a$ but $\lim_{n\rightarrow\infty}f(x_n)=L$ fails. This is a contradiction.
				\item[$\impliedby$]: Consider an arbitrary sequence $(x_n)$ in $S$ such that $\lim_{n\rightarrow \infty}x_n=a$. Thus, choose $\epsilon=\delta$ and there exists $N$ such that 
				\begin{displaymath}
				n>N\implies|x_n-a|<\delta\implies|f(x_n)-L|<\epsilon.
				\end{displaymath}  
				The last implication comes from the assumption, so $\lim_{n\rightarrow a^S}f(x)=L$.
			\end{itemize}
			\end{proof}

			\setcounter{equation}{0}			
			\begin{cor}\label{def:20.7}
			Let $f$ be a function defined on $J\backslash\{a\}$ for some open interval $J$ containing $a$, and let $L$ be a
			real number. Then  $\lim_{x\rightarrow a^S}f(x)=L$ if and only if
			\begin{equation}
				\text{for each $\epsilon>0$ there exists $\delta>0$ such that $0<|x-a|<\delta\implies|f(x)-L|<\epsilon$}
			\end{equation}
			\end{cor}

			\setcounter{equation}{0}			
			\begin{cor}\label{def:20.8}
			Let $f$ be a function defined on some interval $(a,b)$, and let $L$ be a real number. Then $\lim_{x\rightarrow a^+}f(x)=L$ if and only if
			\begin{equation}
				\text{for each $\epsilon>0$ there exists $\delta>0$ such that $a<x<a+\delta\implies|f(x)-L|<\epsilon$}
			\end{equation} 
			\end{cor}
			
			Now let's give some general conditions for the limit of function in different situations: $\lim_{x\rightarrow s}f(x)=L \iff$
			\begin{itemize}
				\item $L$ is finite:
				\begin{itemize}
					\item $s=a$: for each $\epsilon>0$ there exists $\delta>0$ such that $0<|x-a|<\delta$ implies $|f(x)-L|<\epsilon$.
					\item $s=a^+$: for each $\epsilon>0$ there exists $\delta>0$ such that $a<x<a+\delta$ implies $|f(x)-L|<\epsilon$.
					\item $s=a^-$: for each $\epsilon>0$ there exists $\delta>0$ such that $a-\delta<x<a$ implies $|f(x)-L|<\epsilon$.
					\item $s=\infty$: for each $\epsilon>0$ there exists $\alpha<\infty$ such that $x>\alpha$ implies $|f(x)-L|<\epsilon$.
					\item $s=-\infty$: for each $\epsilon>0$ there exists $\alpha>-\infty$ such that $x<\alpha$ implies $|f(x)-L|<\epsilon$.
				\end{itemize}
				\item $L=+\infty$:
				\begin{itemize}
					\item $s=a$: for each $M>0$ there exists $\delta>0$ such that $0<|x-a|<\delta$ implies $f(x)>M$.
					\item $s=a^+$: for each $M>0$ there exists $\delta>0$ such that $a<x<a+\delta$ implies $f(x)>M$.
					\item $s=a^-$: for each $M>0$ there exists $\delta>0$ such that $a-\delta<x<a$ implies $f(x)>M$.
					\item $s=\infty$: for each $M>0$ there exists $\alpha<\infty$ such that $x>\alpha$ implies $f(x)>M$.
					\item $s=-\infty$: for each $M>0$ there exists $\alpha>-\infty$ such that $x<\alpha$ implies $f(x)>M$.
				\end{itemize}
				\item $L=-\infty$:
				\begin{itemize}
					\item $s=a$: for each $M<0$ there exists $\delta>0$ such that $0<|x-a|<\delta$ implies $f(x)<M$.
					\item $s=a^+$: for each $M<0$ there exists $\delta>0$ such that $a<x<a+\delta$ implies $f(x)<M$.
					\item $s=a^-$: for each $M<0$ there exists $\delta>0$ such that $a-\delta<x<a$ implies $f(x)<M$.
					\item $s=\infty$: for each $N<0$ there exists $\alpha<\infty$ such that $x>\alpha$ implies $f(x)<N$.
					\item $s=-\infty$: for each $N<0$ there exists $\alpha>-\infty$ such that $x<\alpha$ implies $f(x)<N$. 
				\end{itemize}
			\end{itemize}
			
			\begin{thm}\label{def:20.10}
			Let $f$ be a function defined on $J\backslash\{a\}$ for some open interval $J$ containing $a$. Then $\lim_{x\rightarrow a}f(x)$ exists $\iff$ the limits $\lim_{x\rightarrow a^+}f(x)$ and $\lim_{x\rightarrow a^-}f(x)$ both exist and are equal to each other, thereby all three limits are equal.
			\end{thm}
			\begin{proof}
			\begin{itemize}
				\item[]
				\item[$\implies$]: Suppose $L$ is finite. Since $\lim_{x\rightarrow a}f(x)=L$, (1) in \ref{def:20.7} holds, and then (1) in \ref{def:20.8} also holds. Thus we have $\lim_{x\rightarrow a^+}f(x)=L$; similarly for $\lim_{x\rightarrow a^-}f(x)=L$.
				
				If $L$ is infinite, say $+\infty$, then consider an arbitrary $M>0$, there exists $\delta>0$ such that
				\begin{equation}
					0<|x-a|<\delta\implies f(x)>M
				\end{equation}
				Then clearly
				\begin{equation}
					a<x<a+\delta\implies f(x)>M
				\end{equation}
				and
				\begin{equation}
					a-\delta<x<a\implies f(x)>M
				\end{equation}
				so $\lim_{x\rightarrow a^+}f(x)=\lim_{x\rightarrow a^-}f(x)=+\infty$.
				\item[$\impliedby$]: Suppose $L$ is finite and $\lim_{x\rightarrow a^+}f(x)=\lim_{x\rightarrow a^-}f(x)=L$. Consider $\epsilon>0$, then we apply \ref{def:20.8} and its analogue for $a^-$ to
				get $\delta_1>0$ and $\delta_2>0$ such that
				\begin{displaymath}
					a<x<a+\delta_1\implies|f(x)-L|<\epsilon
				\end{displaymath}
				and
				\begin{displaymath}
					a-\delta_2<x<a\implies|f(x)-L|<\epsilon.
				\end{displaymath}
				If $\delta=\min\{\delta_1,\delta_2\}$, then
				\begin{displaymath}
					0<|x-a|<\delta\implies|f(x)-L|<\epsilon,
				\end{displaymath}
				so $\lim_{x\rightarrow a}f(x)=L$ by \ref{def:20.7}.
				
				If $L$ is infinite, say $+\infty$, so $\lim_{x\rightarrow a^+}f(x)=\lim_{x\rightarrow a^-}f(x)=+\infty$ then for each $M>0$ there exists $\delta_1>0$ such that (2) holds, and there exists $\delta_2>0$ so that (3) holds. Then $(1)$ holds with $\delta=\min\{\delta_1,\delta_2\}$. We conclude $\lim_{x\rightarrow a}f(x)=+\infty$. 
			\end{itemize}	
			\end{proof}
	\chapter{Sequences and Series of Functions}
		\newpage
		\section{Power Series}
			\begin{dfn}
			Given a sequence $(a_n)_{n=0}^{\infty}$ of real numbers, the series $\sum_{n=0}^{\infty}a_nx^n$ is called 
			a power series, which is a function of $x$ provided it converges for some or all $x$. One of the following
			holds for a power series with coefficients $(a_n)$:
			\begin{enumerate}[(a)]
				\item The power series converge for all $x\in\R$;
				\item The power series converges only for $x=0$;
				\item The power series converges for all $x$ in some bounded interval centered at $0$; the interval may be
				open, half-open, or closed.
			\end{enumerate}
			\end{dfn}
			
			\begin{thm}\label{def:23.1}
			For the power series $\sum a_nx^n$, let
			\begin{displaymath}
				\beta=\lim\sup |a_n|^{1/n}\quad\text{and}\quad R=\frac{1}{\beta}
			\end{displaymath}
			[If $\beta=0$ we set $R=+\infty$, and if $\beta=+\infty$ we set $R=0$.] Then
			\begin{enumerate}[(i)]
				\item The power series converges for $|x|<R$;
				\item The power series diverges for $|x|>R$. 
			\end{enumerate}
			We call $R$ the \emph{radius of convergence} for the power series. For $|x|=R$ we need some extra care.
			\end{thm}
			\begin{proof}
			The proof follows easily from the Root Test \ref{def:Root Test}. Define $\alpha_x=\lim\sup|a_nx^n|^{1/n}$, then
			we have 
			\begin{displaymath}
				\alpha_x=\lim\sup|a_nx^n|^{1/n}=\lim\sup|x||a_n|^{1/n}=|x|\cdot\lim\sup|a_n|^{1/n}=\beta|x|
			\end{displaymath}
			Now we need to consider three different cases:
			\begin{enumerate}
				\item Suppose $0<R<+\infty$. Then $\alpha_x=\frac{|x|}{R}$. If $|x|<R$, then $\alpha_x<1$, so the series
				converge by the root test. Likewise, if $|x|>R$, then $\alpha_x>1$ and the series diverges.
				\item Suppose $R=+\infty$. Then $\beta=0$ and $\alpha_x=0$ no matter what $x$ is. Hence the power series
				converges for all $x$.
				\item Suppose $R=0$. Then $\beta=+\infty$ and $\alpha_x=+\infty$ for $x\neq0$. Thus the series diverges for
				$x\neq0$ by the root test.
			\end{enumerate}
			\end{proof}
			Note that if $\lim\left|\frac{a_{n+1}}{a_n}\right|$ exists, then it equals $\beta$ by \ref{def:12.3}. This limit
			is often easier to calculate than $\lim\sup|a_n|^{1/n}$.
			\newpage
		\section{Uniform Convergence}
			\begin{dfn}
			Let $(f_n)$ be a sequence of real-valued functions defined on a set $S\subseteq\R$. The sequence $(f_n)$ \emph{converges pointwise} to a function $f$ defined on $S$ if
			\begin{displaymath}
				\lim\limits_{n\rightarrow\infty}f_n(x)=f(x)\quad\text{for all}\quad x\in S.
			\end{displaymath}
			We often write $\lim f_n=f$ \emph{pointwise [on S]} or $f_n\rightarrow f$ \emph{pointwise [on S]}
			\end{dfn}
			Observe that saying $f_n\rightarrow f$ \emph{pointwise [on S]} is equivalent to the following:
			\begin{displaymath}
				\text{for each $\epsilon>0$ and $x$ in $S$ there exists $N$ such that $|f_n(x)-f(x)|<\epsilon$ for $n>N$.}
			\end{displaymath}
			
			\begin{dfn}
			Let $(f_n)$ be a sequence of real-valued functions defined on a set $S\subseteq\R$. The sequence $(f_n)$ \emph{converges uniformly} on $S$ to a function $f$ defined on $S$ if
			\begin{align*}
				\text{for each $\epsilon>0$ there exists a number $N$ such that}\\
				\text{$|f_n(x)-f(x)|<\epsilon$ for all $x\in S$ and all $n>N$.}
			\end{align*}
			We write $\lim f_n=f$ \emph{uniformly [on S]} or $f_n\rightarrow f$ \emph{uniformly [on S]}
			\end{dfn}
			
			\setcounter{equation}{0}
			\begin{thm}\label{thm:24.3}
			The uniform limit of continuous functions is continuous. More precisely, let $(f_n)$ be a sequence of functions
			on a set $S\subseteq\R$, suppose $f_n\rightarrow f$ uniformly on $S$, and suppose $S=\dom(f)$. If each $f_n$ is
			continuous at $x_0$ in $S$, then $f$ is continuous at $x_0$. [so if each $f_n$ is continuous on $S$, then $f$ is continuous on $S$.]
			\end{thm}
			\begin{proof}
			The critical inequality for this proof is
			\begin{equation}
				|f(x)-f(x_0)|\leq|f(x)-f_n(x)|+|f_n(x)-f_n(x_0)|+|f_n(x_0)-f(x_0)|
			\end{equation}
			Let $\epsilon>0$. There exists $N\in\N$ such that
			\begin{displaymath}
				n>N\implies \forall x\in S,\ |f_n(x)-f(x)|<\frac{\epsilon}{3}
			\end{displaymath}
			since $f_n\rightarrow f$ uniformly on $S$. In particular,
			\begin{equation}
				\forall x\in S,\ |f_{N+1}(x)-f(x)|<\frac{\epsilon}{3}
			\end{equation}
			Since $f_{N+1}$ is continuous at $x_0$ there is a $\delta>0$ such that
			\begin{equation}
				\forall x\in S,\ |x-x_0|<\delta\implies|f_{N+1}(x)-f_{N+1}(x_0)|<\frac{\epsilon}{3};
			\end{equation}
			Now apply (1) with $n=N+1$, (2) twice and (3) once to conclude
			\begin{displaymath}
				\forall x\in S,\ |x-x_0|<\delta\implies|f(x)-f(x_0)|<3\cdot\frac{\epsilon}{3}=\epsilon;
			\end{displaymath}
		 	Thus $f$ is continuous at $x_0$.
			\end{proof}
			The contrapositive of this theorem is useful to show $f_n$ does not uniformly converge to $f$ on $S$:
			If $f$ is not continuous at $x_0\in S$ but $f_n$ is continuous at $x_0$, then the statement that "$f_n\rightarrow f$ uniformly on $S$" is \textbf{incorrect}.
		
			It is worthy to mention that the uniform convergence can be reformulated as follows: \emph{A sequence $(f_n)$ of functions on a set $\subseteq\R$ converges uniformly to a function $f$ on $S$ $\iff$}
			\begin{displaymath}
				\lim\limits_{n\rightarrow\infty}\sup\{|f(x)-f_n(x)|: x\in S\}=0
			\end{displaymath}
			Thus we can decide whether a sequence $(f_n)$ converges uniformly to $f$ by calculating $\sup\{|f(x)-f_n(x)|: x\in S\}$ for each $n$. If $f-f_n$ is differentiable, we may use calculus to find these suprema.
			\newpage
		\section{More on Uniform Convergence}
			Here are two important facts about integration
			\begin{enumerate}
				\item If $g$ and $h$ are integrable on $[a,b]$ and if $g(x)\leq h(x)$ for all $x\in[a,b]$, then $\int_{a}^{b}g(x)dx\leq\int_{a}^{b}h(x)dx$.
				\item If $g$ is integrable on $[a,b]$, then
				\begin{displaymath}
					\abs{\int_{a}^{b}g(x)dx}\leq\int_{a}^{b}\abs{g(x)}dx.
				\end{displaymath}
			\end{enumerate}
			Also continuous functions on closed intervals are integrable.
			
			\begin{thm}\label{thm:25.2}
			Let $(f_n)$ be a sequence of continuous functions on $[a,b]$, and suppose $f_n\rightarrow f$ uniformly on $[a,b]$. Then
			\begin{displaymath}
				\lim\limits_{n\rightarrow\infty}\int_{a}^{b}f_n(x)dx=\int_{a}^{b}f(x)dx.
			\end{displaymath}
			\begin{proof}
			By \ref{def:24.3} $f$is continuous, so $f_n-f$ are all integrable on $[a,b]$. Let $\epsilon>0$. Since $f_n\rightarrow f$ uniformly on $[a,b]$, there exists $N$ such that $\abs{f_n(x)-f(x)}<\frac{\epsilon}{b-a}$ for all $x\in[a,b]$ and all $n>N$. Consequently $n>N$ implies
			\begin{align*}
				\abs{\int_{a}^{b}f_n(x)dx-\int_{a}^{b}f(x)dx}&=\abs{\int_{a}^{b}[f_n(x)-f(x)]dx}\\
															 &\leq\int_{a}^{b}\abs{f_n(x)-f(x)}dx\\
															 &\leq\int_{a}^{b}\frac{\epsilon}{b-a}dx\\
															 &=\epsilon
			\end{align*}
			Thus $\lim\limits_{n\rightarrow\infty}\int_{a}^{b}f_n(x)dx=\int_{a}^{b}f(x)dx$.
			\end{proof}
			\end{thm}
			
			\begin{dfn}
			A sequence $(f_n)$ of functions defined on a set $S\subseteq\R$ is \emph{uniformly Cauchy on S} if
			\begin{align*}
				\text{for each $\epsilon>0$ there exists a number $N$ such that}\\
				\text{$\abs{f_n(x)-f_m(x)}<\epsilon$ for all $x\in S$ and all $m,n>N$.}
			\end{align*}
			\end{dfn}
			Notice that uniformly convergent sequences of functions are uniformly Cauchy.
			
			\setcounter{equation}{0}
			\begin{thm}\label{thm:25.4}
			Let $(f_n)$ be a sequence of functions defined and uniformly Cauchy on a set $S\subseteq\R$. Then there exists a function $f$ on $S$ such that $f_n\rightarrow f$ uniformly on $S$.
			\end{thm}
			\begin{proof}
			Since $(f_n)$ is uniformly Cauchy, $(f_n(x_0))$ is a Cauchy sequence for each $x_0\in S$. Now for each $x\in S$, define $f(x)=\lim_{n\rightarrow\infty}f_n(x)$. This defines a function $f$ on $S$ such that $f_n\rightarrow f$ pointwise on $S$.
			
			Let $\epsilon>0$. Since $(f_n)$ is uniformly Cauchy, there is a number $N$ such that 
			\begin{equation}
				\abs{f_n(x)-f_m(x)}<\frac{\epsilon}{2}\quad\text{for all}\quad x\in S\quad\text{and all}\quad m,n<N
			\end{equation}
			Consider $m>N$ and $x\in S$. (1) tells us that $f_n(x)$ lies in $(f_m(x)-\frac{\epsilon}{2},f_m(x)+\frac{\epsilon}{2})$ for all $n>N$. Therefore, $f(x)=\lim_{n\rightarrow \infty}f_n(x)$ lies in $[f_m(x)-\frac{\epsilon}{2},f_m(x)+\frac{\epsilon}{2}]$. In other words,
			\begin{displaymath}
				\abs{f(x)-f_m(x)}\leq\frac{\epsilon}{2}\quad\text{for all}\quad x\in S\quad\text{and}\quad m>N
			\end{displaymath}
			Then of course
			\begin{displaymath}
				\abs{f(x)-f_m(x)}\leq\epsilon\quad\text{for all}\quad x\in S\quad\text{and}\quad m>N
			\end{displaymath}
			Thus $f_m\rightarrow f$ uniformly on $S$, as desired.
			\end{proof}
			
			A \emph{series of functions} is an expression $\sum_{k=0}^{\infty}g_k$ or $\sum_{k=0}^{\infty}g_k(x)$ which makes sense provided the sequence of partial sums $\sum_{k=0}^{n}g_k$ converges, or diverges to $+\infty$ or $-\infty$ pointwise. If the sequence of partial sums converges uniformly on a set $S$ to $\sum_{k=0}^{\infty}$, then we say the \emph{series is uniformly convergent} on $S$.
			
			\begin{thm}\label{thm:25.5}
			Consider a series $\sum_{k=0}^{\infty}g_k$ of functions on a set $S\subseteq\R$. Suppose each $g_k$ is continuous on $S$ and the series converges uniformly on $S$. Then the series $\sum_{k=0}^{\infty}g_k$ represents a continuous function on $S$.
			\end{thm}
			\begin{proof}
			Each partial sum $f_n=\sum_{k=1}^{n}g_k$ is continuous and the sequence $(f_n)$ converges uniformly on $S$. Hence the limit function is continuous by \ref{def:24.3}.
			\end{proof}
			There is an analogue between the Cauchy criterion for a normal series $\sum a_k$ and the one for a series of functions $\sum g_k$: \emph{The sequence of partial sums of a series $\sum_{k=0}^{\infty}g_k$ of functions is uniformly Cauchy on a set $S$ $\iff$ the series satisfies the \emph{Cauchy criterion}[uniformly on $S$]}:
			\begin{align*}
				\text{for each $\epsilon>0$ there exists a number $N$ such that}\\
				\text{$n\geq m>N$ implies $\abs{\sum_{k=m}^{n}g_k(x)}<\epsilon$ for all $x\in S$.}
			\end{align*}
			
			\begin{thm}\label{thm:25.6}
			If a series $\sum_{k=0}^{\infty}g_k$ of functions satisfies the Cauchy criterion uniformly on a set $S$, then the
			series converges uniformly on $S$.
			\end{thm}
			\begin{proof}
			Let $f_n=\sum_{k=0}^{n}g_k$. The sequence $(f_n)$ of partial sums is uniformly Cauchy on $S$, so $(f_n)$ converges uniformly on $S$ by \ref{def:25.4}.
			\end{proof}
			
			\begin{thm}[Weierstrass M-test]\label{thm:25.7}
			Let $(M_k)$ be a sequence of nonnegative real numbers where $\sum M_k<\infty$. If $|g_k(x)|\leq M_k$ for all $x$ in a set $S$, then $\sum g_k$ converges uniformly on $S$.
			\end{thm}
			\begin{proof}
			We want to verify the Cauchy criterion of such $\sum g_k$ on $S$. Let $\epsilon>0$. Since the series $\sum M_k$ converges, it satisfies the Cauchy criterion in \ref{def:cauchy criterion}. So there exists a number $N$ such that
			\begin{displaymath}
				n\geq m>N\implies\sum_{k=m}^{n}M_k<\epsilon.
			\end{displaymath}
			Hence if $n\geq m>N$ and $x$ is in $S$, then
			\begin{displaymath}
				\abs{\sum_{k=m}^{n}g_k(x)}\leq\sum_{k=m}^{n}\abs{g_k(x)}\leq\sum_{k=m}^{n}M_k<\epsilon. 
			\end{displaymath}
			Thus the series $\sum g_k$ satisfies the Cauchy criterion uniformly on $S$, and \ref{def:25.6} shows that it converges uniformly on $S$.
			\end{proof}
			
			If the series $\sum g_n$ converges uniformly on a set $S$, then
			\begin{displaymath}
				\lim\limits_{n\rightarrow\infty}\sup\{|g_n(x)|: x\in S\}=0.
			\end{displaymath}
			\newpage
		\section{Differentiation and Integration of Power Series}
			\begin{thm}\label{thm:26.1}
			Let $\sum_{n=0}^{\infty}a_nx^n$ be a power series with radius of convergence $R>0$ [possibly $R=+\infty$]. If $0<R_1<R$, then the power series converges uniformly on $[-R_1, R_1]$ to a continuous function.
			\end{thm}
			\begin{proof}
			Consider $0<R_1<R$. First observe that the series $\sum a_nx^n$ and $\sum|a_n|x^n$ have the same radius of convergence. Since $|R_1|<R$, we have $\sum a_nR_1^n<\infty$. Clearly we have $|a_nx^n|\leq|a_n|R_1^n$ for all $x\in[-R_1,R_1]$, so the series $\sum a_nx^n$ converges uniformly on $[-R_1,R_1]$ by \ref{thm:25.7}. Then by \ref{def:25.5} since each $a_nx^n$ is continuous, the limit of the series of the functions is also continuous.
			\end{proof}
			
			\begin{cor}\label{cor:26.2}
			The power series $\sum a_nx^n$ converges to a continuous function on the open interval $(-R,R)$.
			\end{cor}
			\begin{proof}
			If $x_0\in(-R,R)$, then $x_0\in(-R_1,R_1)$ for some $R_1<R$. The theorem shows the limit of the series is continuous at $x_0$. 
			\end{proof}
			
			\begin{lem}\label{lem:26.3}
			If the power series $\sum_{n=0}^{\infty}a_nx^n$ has radius of convergence $R$, then the power series
			\begin{displaymath}
				\sum_{n=1}^{\infty}na_nx^{n-1}\quad\text{and}\quad\sum_{n=0}^{\infty}\frac{a_n}{n+1}x^{n+1}
			\end{displaymath} 
			also have radius of convergence $R$.
			\end{lem}
			\begin{proof}
			First observe the series $\sum na_nx^{n-1}$ and $\sum na_nx^n$ have the same radius of convergence. Same for $\sum\frac{a_n}{n+1}x^{n+1}$ and $\sum\frac{a_n}{n+1}x^n$. It is because one is $x$ multiple of the other.
			
			For the series $\sum na_nx^n$, because $\lim\sup(n|a_n|)^{1/n}=\lim\sup n^{1/n}|a_n|^{1/n}$ and $\lim n^{1/n}=1$, so $\lim\sup(n|a_n|)^{1/n}=\lim\sup|a_n|^{1/n}=\beta$. Hence $\sum na_nx^n$ has radius of convergence $R$.
			
			Similarly, for the series $\sum\frac{a_n}{n+1}x^n$, because $\lim\sup(\frac{|a_n|}{n+1})^{1/n}=\lim\sup(\frac{1}{n+1})^{1/n}\cdots|a_n|^{1/n}=1\cdot\lim\sup|a_n|^{1/n}=\beta$ by \ref{thm:12.1}, so the series $\sum\frac{a_n}{n+1}x^n$ has radius of convergence $R$.
			\end{proof}
			
			\begin{thm}\label{thm:26.4}
			Suppose $f(x)=\sum_{n=0}^{\infty}a_nx^n$ has radius of convergence $R>0$. Then
			\begin{displaymath}
				\int_{0}^{x}f(t)dx=\sum_{n=0}^{\infty}\frac{a_n}{n+1}x^{n+1}\quad\text{for}\quad |x|<R
			\end{displaymath}
			\end{thm}
			\begin{proof}
			Fix $x<0$. The case $x>0$ is similar. On the interval $[x,0]$, the sequence of partial sums $\sum_{k=0}^{n}a_kt^k$ converges uniformly to $f(t)$ by \ref{thm:26.1}. Consequently, by \ref{thm:25.2} we have
			\begin{align*}
				\int_{x}^{0}f(t)dt&=\lim_{n\rightarrow \infty}\int_{x}^{0}\left(\sum_{k=0}^{n}a_kt^k\right)dt\\
								  &=\lim_{n\rightarrow \infty}\sum_{k=0}^{n}a_k\int_{x}^{0}t^kdt\\
								  &=\lim_{n\rightarrow \infty}\sum_{k=0}^{n}a_k\left[\frac{0^{k+1}-x^{k+1}}{k+1}\right]\\
								  &=-\left(\sum_{k=0}^{\infty}\frac{a_k}{k+1}x^{k+1}\right)\\
								  &=-\left(\int_{0}^{x}f(t)dt\right)
			\end{align*}
			\end{proof}
			The last theorem shows that a power series can be integrated term-by-term inside its interval of convergence. The next theorem shows that term-by-term differentiation is also legal
			
			\begin{thm}\label{thm:26.5}
			Let $f(x)=\sum_{n=0}^{\infty}a_nx^n$ have radius of convergence $R>0$. Then $f$ is differentiable on $(-R,R)$ and \begin{displaymath}
				f^\prime(x)=\sum_{n=1}^{\infty}na_nx^{n-1}\quad\text{for}\quad|x|<R
			\end{displaymath}
			\end{thm}
			\begin{proof}
			We begin with the series $g(x)=\sum_{n=1}^{\infty}na_nx^{n-1}$ and observe this series converges for $|x|<R$ by \ref{lem:26.3}. By \ref{thm:26.4}, we can integrate $g(x)$ term-by-term:
			\begin{displaymath}
				\int_{0}^{x}g(t)dt=\sum_{n=1}^{\infty}a_nx^n=f(x)-a_0\quad\text{for}\quad|x|<R.
			\end{displaymath}
			Thus if $0<R_1<R$, then
			\begin{displaymath}
				f(x)=\int_{-R_1}^{x}g(t)dt+k\quad\text{for}\quad|x|\leq R_1
			\end{displaymath}
			where $k=a_0-\int_{-R_1}^{0}g(t)dt$. Since $g$ is continuous, the Fundamental Theorem of Calculus shows $f$ is differentiable and $f^\prime(x)=g(x)$. Thus
			\begin{displaymath}
				f^\prime(x)=g(x)=\sum_{n=1}^{\infty}na_nx^{n-1}\quad\text{for}\quad|x|\leq R_1
			\end{displaymath}
			\end{proof}
			
			\setcounter{equation}{0}
			\begin{thm}[Abel's Theorem]\label{thm:26.6}
			Let $f(x)=\sum_{n=0}^{\infty}a_nx^n$ be a power series with finite positive radius of convergence $R$. If the series converges at $x=R$, then $f$ is continuous at $x=R$; if the series converges at $x=-R$, then $f$ is continuous at $x=-R$.
			\end{thm}
			\begin{proof}
			\begin{itemize}
				\item[]
				\item[Case] 1. Suppose $f(x)=\sum_{n=0}^{\infty}a_nx^n$ has radius of convergence $1$ and the series converges at $x=1$. We will prove $f$ is continuous on $[0,1]$. We need to show that $f$ is continuous on $[0,1]$. By subtracting a constant from $f$, we may assume $f(1)=\sum_{n=0}^{\infty}a_n=0$. Let $f_n(x)=\sum_{k=0}^{n}a_kx^k$ and $s_n=\sum_{k=0}^{n}a_k=f_n(1)$ for $n=0,1,2,\dots.$. Since $f_n(x)\rightarrow f(x)$ pointwise on $[0,1]$ and each $f_n$ is continuous, by \ref{thm:24.3} it suffices to show $f_n\rightarrow f$ uniformly on $[0,1]$. By \ref{thm:25.4} it suffices to show the convergence is uniformly Cauchy.
				
				For $m<n$, we have
				\begin{align*}
					f_n(x)-f_m(x)&=\sum_{k=m+1}^{n}a_kx^k\\
								 &=\sum_{k=m+1}^{n}(s_k-s_{k-1})x^k\\
								 &=\sum_{k=m+1}^{n}s_kx^k-x\sum_{k=m+1}^{n}s_{k-1}x^{k-1}\\
								 &=\sum_{k=m+1}^{n}s_kx^k-x\sum_{k=m}^{n-1}s_kx^{k}
				\end{align*}
				and therefore
				\begin{equation}
					f_n(x)-f_m(x)=s_nx^n-s_mx^{n+1}+(1-x)\sum_{k=m+1}^{n-1}s_kx^k
				\end{equation}
				By the definition of $s_n$, we have $\lim s_n=\sum_{k=0}^{\infty}a_k=f(1)=0$. Given $\epsilon>0$, there is an integer $N$ so that $|s_n|<\frac{\epsilon}{3}$ for all $n\geq N$. Then for $n>m\geq N$ and $x\in[0,1)$,
				we have
				\begin{align}
					\left|(1-x)\sum_{k=m+1}^{n-1}s_kx^k\right|&\leq\frac{\epsilon}{3}(1-x)\sum_{k=m+1}^{n-1}x^k\notag\\
															  &=\frac{\epsilon}{3}(1-x)x^{m+1}\frac{1-x^{n-m-1}}{1-x}\notag\\
															  &<\frac{\epsilon}{3}
				\end{align}
				Since $\left|(1-x)\sum_{k=m+1}^{n-1}s_kx^k\right|<\frac{\epsilon}{3}$ for $x=1$, combining (1) and (2), for $n>m\geq N$ and $x\in[0,1]$,
				\begin{displaymath}
					|f_n(x)-f_m(x)|\leq|s_n|x^n+|s_m|x^{m+1}+\frac{\epsilon}{3}<\frac{\epsilon}{3}+\frac{\epsilon}{3}+\frac{\epsilon}{3}=\epsilon.
				\end{displaymath}
				Thus the sequence $(f_n)$ is uniformly Cauchy on $[0,1]$, and its limit $f$ is continuous.
				\item[Case] 2. Suppose $f(x)=\sum_{n=0}^{\infty}a_nx^n$ has radius of convergence $R$, $0<R<\infty$, and the series converges at $x=R$. Let $g(x)=f(Rx)$ and note that 
				\begin{displaymath}
					g(x)=\sum_{n=0}^{\infty}a_nR^nx^n\quad\text{for}\quad|x|<1.
				\end{displaymath}
			 	This series has radius of convergence $1$, and it converges at $x=1$. By Case 1, $g$ is continuous at $x=1$. Since $f(x)=g(\frac{x}{R})$, it follows that $f$ is continuous at $x=R$.
			 	\item[Case] 3. Suppose $f(x)=\sum_{n=0}^{\infty}a_nx^n$ has radius of convergence $R$, $0<R<\infty$, and the series converges at $x=-R$. Let $h(x)=f(-x)$ and note that 
			 	\begin{displaymath}
			 		h(x)=\sum_{n=0}^{\infty}(-1)^na_nx^n\quad\text{for}\quad|x|<R.
			 	\end{displaymath}
			 	This series for $h$ converges at $x=R$, so $h$ is continuous at $x=R$ by Case 2. It follows that $f(x)=h(-x)$ is continuous at $x=-R$.
			\end{itemize}
			\end{proof}
	\chapter{Differentiation}
		\newpage
		\section{Basic Properties of the Derivative}
			\begin{dfn}\label{def:28.1}
			Let $f$ be a real-valued function defined on an open interval containing a point $a$. We say $f$ is \emph{differentiable} at $a$, or $f$ has a derivative at $a$, if the limit
			\begin{displaymath}
				\lim_{x\rightarrow a}\frac{f(x)-f(a)}{x-a}
			\end{displaymath}
			exists and is finite. We will write $f^\prime(a)$ for the derivative of $f$ at $a$:
			\begin{displaymath}
				f^\prime(a)=\lim_{x\rightarrow a}\frac{f(x)-f(a)}{x-a}
			\end{displaymath}
			whenever this limit exists and is finite.
			\end{dfn}
			The domain of $f^\prime$ is the set of points at which $f$ is differentiable; thus $\dom(f^\prime)\subseteq\dom(f)$.
			
			\begin{thm}\label{thm:28.2}
			If $f$ is differentiable at a point $a$, then $f$ is continuous at $a$.
			\end{thm}
			\begin{proof}
			We are given $f^\prime(a)=\lim_{x\rightarrow a}\frac{f(x)-f(a)}{x-a}$, and we need to prove $\lim_{x\rightarrow a}f(x)=f(x)$. We have 
			\begin{displaymath}
				f(x)=(x-a)\frac{f(x)-f(a)}{x-a}+f(a)
			\end{displaymath}
			for $x\in\dom(f), x\neq a$. Since $\lim_{x\rightarrow a}(x-a)=0$ and $\lim_{x\rightarrow a}\frac{f(x)-f(a)}{x-a}$ exists and is finite, by \ref{def:20.4}(ii), $\lim (x-a)\frac{f(x)-f(a)}{x-a}=0$. Thus $\lim_{x\rightarrow a}f(x)=f(a)$.
			\end{proof}
			
			\begin{thm}\label{def:28.3}
			Let $f$ and $g$ be functions that are differentiable at the point $a$. Each of the functions $cf$, $f+g$, $fg$, and $f/g$ is also differentiable at $a$, except $f/g$ is $g(a)=0$ since $f/g$ is not defined at $a$ in this case.
			The formulas are:
			\begin{enumerate}[(i)]
				\item $(cf)^\prime(a)=c\cdot f^\prime(a)$;
				\item $(f+g)^\prime(a)=f\prime(a)+g^\prime(a)$;
				\item Product rule: $(fg)^\prime(a)=f(a)g^\prime(a)+f^\prime(a)g(a)$;
				\item Quotient rule: $(f/g)^\prime(a)=\frac{[g(a)f^\prime(a)-f(a)g^\prime(a)]}{g^2(a)}$ if $g(a)\neq0$.
			\end{enumerate}
			\end{thm}
			\begin{proof}
			\begin{enumerate}[(i)]
				\item[]
				\item
				\begin{displaymath}
					(cf)^\prime(a)=\lim_{x\rightarrow a}\frac{(cf)(x)-(cf)(a)}{x-a}=\lim_{x\rightarrow a}c\cdot\frac{f(x)-f(a)}{x-a}=c\cdot f^\prime(a)
				\end{displaymath} 
				\item
				\begin{displaymath}
					\frac{(f+g)(x)-(f+g)(a)}{x-a}=\frac{f(x)-f(a)}{x-a}+\frac{g(x)-g(a)}{x-a}
				\end{displaymath}
				\item See the textbook.
				\item See the textbook.
			\end{enumerate}
			\end{proof}
			
			\begin{thm}[Chain Rule]\label{def:28.4}
			If $f$ is differentiable at $a$ and $g$ is differentiable at $f(a)$, then the composite function $g\circ f$ is differentiable at $a$ and we have $(g\circ f)^\prime(a)=g^\prime(f(a))\cdot f^\prime(a)$.
			\end{thm}
			If $f$ is differentiable on an interval $I$ and if $g$ is differentiable on $\{f(x): x\in I\}$, then $(g\circ f)^\prime$ is exactly $(g^\prime\circ f)\cdot f^\prime$ on $I$.
			
	\chapter{Useful Tricks}
	\begin{enumerate}
		\item Here is one of the most important techniques in real analysis.
		\begin{enumerate}[(a)]
			\item If $a\leq b+\epsilon$ for any $\epsilon>0$, then $a\leq b$.
			\item If $a\geq b-\epsilon$ for any $\epsilon>0$, then $a\geq b$.
			\item If $|a-b|<\epsilon$ for any $\epsilon>0$, then $|a-b|=0$.
		\end{enumerate}
		\item Let $S$ be a bounded nonempty subset of $\R$ and suppose $\sup S\notin S$. Then there is a (strictly) increasing sequence $(s_n)$ of points in $S$ such that $\lim s_n=\sup S$.
		\item A point $x$ is a limit point of a set $E\subseteq X$ if and only if $x=\lim x_n$ for some sequence $x_n$ of points in $E\backslash\{x\}$. 
		\item Let $(s_n)$ be a convergent sequence.
			\begin{itemize}
				\item If $s_n\geq a$ for all but finitely many $n$, then $\lim s_n\geq a$.
				\item If $s_n\leq b$ for all but finitely many $n$, then $\lim s_n\leq b$.
		\end{itemize}
		\item (Squeeze Theorem) If $a_n\leq s_n\leq b_n$ for all $n$ and $\lim a_n=\lim b_n=s$, then $\lim s_n=s$.
		\item Assume all $s_n\neq0$ and that the limit $L=\lim\left|\frac{s_{n+1}}{s_n}\right|$ exists.
		\begin{enumerate}[(a)]
			\item If $L<1$, then $\lim s_n=0$.
			\item If $L>1$, then $\lim |s_n|=+\infty$.
		\end{enumerate} 
		\item The set $\mathbb{Q}$ of rational number can be listed as a sequence $(r_n)$. Given any real number $a$ there exists a subsequence $(r_{n_k})$ of $(r_n)$ converging to $a$.
		\item Given two \textbf{convergent} sequences $(s_n)$ and $t_n$. If there exists $N\in\N$ such that $s_n\leq t_n$ for all $n\geq N_0$, then $\lim s_n\leq\lim t_n$.
		\item In general, if $A\subseteq B$, then $\inf A\geq \inf B$ and $\sup A\leq \sup B$.
  	\end{enumerate}
		
\end{document}